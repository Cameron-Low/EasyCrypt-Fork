

\subsection{Reasoning on the program structure}
\subsubsection*{Empty statements: the \rawec{skip} tactic}
\index{prhl}{\rawec{skip}}

\Syntax \rawec{skip}

\Description Reduces logical program judgements with empty statements
to a higher-order logical goal. The generated subgoals can then be
processed using the ambient logic. The behaviour of the \rawec{skip}
tactic can be briefly described by the following rule:
%
\begin{displaymath}
\infrule{
  \pre \Rightarrow \post
}{
  \Equiv{\Skip}{\Skip}{\pre}{\post}
}
\end{displaymath}
%


\subsubsection*{Random samplings: the \rawec{rnd} tactic}
\index{prhl}{\rawec{rnd}}

\Syntax \rawec{rnd}[\textit{side}] [\textit{form} | \textit{form} \textit{form}]

\Description

The logical rule implemented by the \rawec{rnd} tactic depends on the
the optional parameter \textit{side}. If a left/right side flag is
provided then the one-sided logical rule for random sampling is
applied. If missing, then the two-sided rule for random assignment is
considered.
%

\paragraph*{Two-sided application.} 
When no side flag is provided, then the \rawec{rnd} tactic takes as
parameter a representation of a bijective function.

When two formulae are provided as the bijection parameter,
they are verified to be bijective functions and inverse of each
other. If only one function is provided then this function is verified
to be an involution, and lastly if no argument is given then the
identity function is considered.

The description of the rule below assumes that a bijective function
$f$ and its inverse is provided and generates according verification
conditions. Furthermore, it requires the following type constraints
for some types \rawec{'a} and \rawec{'b}: 
\begin{itemize}
\item $d_1$ \rawec{: 'a Distr.distr}, 
\item $d_2$ \rawec{: 'b Distr.distr}
\item $f$ \rawec{: 'a} $\to$ \rawec{'b},
\item $g$ \rawec{: 'b} $\to$ \rawec{'a}, 
\end{itemize}

\begin{displaymath}
\infrule{
  \Equiv{c_1}{c_2}{\pre} 
  { % \begin{array}{l}
      \forall z\,v,\insupp{z}{d_1}\Rightarrow
      \insupp{v}{d_2}\Rightarrow \mathsf{bij}\,f\,g
       \land
      \post\subst{x}{z}\subst{y}{f\,z}
    % \end{array}
  }
}{
  \Equiv{c_1;\Rand{x}{d_1}}{c_2;\Rand{y}{d_2}}{\pre}{\post}
}\left[\mathec{rnd}\ f\ g\right]
\end{displaymath}
where
$\mathsf{bij}\,f\,g$ stands for
\begin{displaymath}
  \mu\,d_1\,\charfun_{\{z\}}=\mu\,d_2\,\charfun_{\{f\,z\}} 
  \land \insupp{(g\,v)} {d_1} \land
  g\,(f\,z)=z \land f\,(g\,v)=v
\end{displaymath}
    

\paragraph*{One-sided application.} 
If the \emph{side} optional argument is provided, then the \rawec{rnd}
tactic implements the following rule:
%
\begin{displaymath}
\infrule{
  \Equiv{c}{c'}{\pre}{\mathsf{weight}\ d=1 \land  \forall z,\insupp{z}{d} \Rightarrow \post\subst{x}{z}}
}{
  \Equiv{c;\Rand{x}{d}}{c'}{\pre}{\post}
}\left[\mathec{rnd}\{1\}\right]
\end{displaymath}
where $\mathsf{weight}\ d$ stands for the expression
$\mathec{Distr.weight}\ d$ defined as $\mathec{mu}\ d\ \mathec{cpTrue}$.
%
A similar rule holds for an invocation to $\left[\mathec{rnd}\{2\}\right]$.



\subsubsection*{Sequential composition: the \rawec{seq} tactic}
\index{prhl}{\rawec{seq}}


\Syntax
\rawec{seq} \textit{codepos} \textit{form}

\Description
Applies the RHL rule for sequential composition:
$$
\infrule{\Equiv{c_1}{c_2}{\post}{\post'} \quad
         \Equiv{c_1'}{c_2'}{\post'}{\post''}}
        {\Equiv{c_1;c_1'}{c_2;c_2'}{\post}{\post''}}[\textrm{R-Seq}]
$$
The application of tactic $\mathec{app}\ m\ n\ p$ defines $c_1$ as the first
$m$ instructions of the program on the left-hand side and $c_2$ as
the first $n$ instructions of the program on the right-hand side
and $\post'$ as $p$.







\subsubsection*{Conditional statements: the \rawec{condt,condf} tactic}
\index{prhl}{\rawec{condt,condf}}
%
\NotDocumented


\subsubsection*{Conditional statements: the \rawec{if} tactic}
\index{prhl}{\rawec{if}}

\Syntax \rawec{if} [\textit{side}]

\Description Applies the pRHL rule for conditional.
If the \textit{side} argument is given then the corresponding
one-side rule is used, else the two side rule is used.
The \rawec{if} tactic expects a conditional as first instruction. 
\begin{center}
\begin{tabular}{c|c}
Syntax & Rule \\
\hline\\
\mathec{if\{1\}} &
$
\infrule{\Equiv{c_1;c}{c'}{\pre \land e\sidel}{\post}
        \quad \Equiv{c_2;c}{c'}{\pre \land \neg e\sidel}{\post}}
        {\Equiv{\Cond{e}{c_1}{c_2};c}{c'}{\pre}{\post}}
$\\
\\\hline\\
\mathec{if\{2\}} &
$
\infrule{\Equiv{c'}{c_1;c}{\pre \land e\sider}{\post}
        \quad \Equiv{c'}{c_2;c}{\pre \land \neg e\sider}{\post}}
        {\Equiv{c'}{\Cond{e}{c_1}{c_2};c}{\pre}{\post}}
$\\
\\\hline\\
\mathec{if} &
$
\infrule{
 \begin{array}{c}
   \vdash \pre \Rightarrow e\sidel = e'\sider \\
   \Equiv{c_1;c}{c'_1;c'}{\pre \land e\sidel \land e'\sider}{\post}\\
   \Equiv{c_2;c}{c'_2;c'}{\pre \land \neg e\sidel \land \neg e'\sider}{\post}
 \end{array}
}{\Equiv{\Cond{e}{c_1}{c_2};c}
        {\Cond{e'}{c'_1}{c'_2};c'}
        {\pre}{\post}}
$\\
\end{tabular}
\end{center}





\subsubsection*{Deterministic straight-line code: the \rawec{wp} tactic}
\index{prhl}{\rawec{wp}}


\Syntax \rawec{wp} [\textit{codepos}]

\Description The \rawec{wp} tactic computes the weakest-precondition of
deterministic, loop and procedure-call free program fragments
(i.e. deterministic assignments and conditionals).   
The computation of the weakest precondition over a
conditional statement is only possible if its branches do not
contain random samplings, while-loops nor function calls.

The optional code position parameter \textit{pos} restricts the range
of instructions that may be affected by the tactic invocation. 
%
If the optional code position parameter is not provided, The tactic
processes instructions bottom-up until a random sampling, a loop or a
function call is reached.


\subsubsection*{Function call statements: the \rawec{call} tactic}
\index{prhl}{\rawec{call}}



\subsubsection*{While loop statements: the \rawec{while} tactic}
\index{prhl}{\rawec{while}}

\Syntax  \rawec{while} [\textit{side}] \textit{form} [\textit{form}]

\Description This tactic applies the pRHL verification rules for
loops:
\begin{itemize}
\item the optional argument \textit{side} can be either \rawec{{1}} or
  \rawec{{2}} to indicate the application of one-sided versions of the
  rule. If missing, the two-sided rule for loops is considered.
\item the first \textit{form} argument is mandatory and is used as
  loop invariant. It can refer to variables in both the left and right
  programs.
\item the optional parameter \textit{form} is required (and accepted
  only) in the one-sided application of the rule. This parameter
  corresponds to the decreasing variant expression used to prove loop
  termination.
\end{itemize}



\paragraph{Two-sided version.}
%
\Syntax \rawec{while} \textit{form} 
%
\Description Applies the two-sided RHL rule for while loops, using the
\textit{form} parameter as loop invariant. This tactic requires that
the last instruction of both left and right statements are while loops.
In the rule, $M$ refers to the variables that may be modified by the
loop bodies.

\begin{displaymath}
\infrule{ 
  \begin{array}{c}
    \Equiv{c_2}{c'_2}{I \land e\sidel \land e'\sider}{I \land  e\sidel = e'\sider}\\
    \Equiv{c_1}{c'_1}{\pre}{ I \land e\sidel = e'\sider \land 
      \forall M, (I \land \neg e\sidel \land \neg e'\sider \Rightarrow \post)}
  \end{array}
}{
  \Equiv{c_1;\While{e}{c_2}}{c'_1;\While{e'}{c'_1}}{\pre}{\post}
}
\end{displaymath}

\paragraph{One-sided version.}

\Syntax \rawec{while} \textit{side} \textit{form} \textit{form} 

\Description Applies the one-sided pRHL rule for while loops, using
the first parameter \textit{form} as loop invariant and the second
parameter \textit{form} as a decreasing \textit{variant}
expression. The variant is used to verify the loop termination. The
one-sided rule are described below. Only the left (i.e., flagged with
\rawec{\{1\}}) variant is shown; the right (i.e., flagged with
\rawec{\{2\}}) variant is symmetric. The expressions $\forall
X,~\varphi$ and $\exists X,~\varphi$ denote, respectively, universal
and existential quantification over the set of variables $X$ modified
in the loop body $c$. 

\begin{displaymath}
\infrule{
  \begin{array}{c}
    \vdash I \land v \leq b \Rightarrow \neg e  \\
    \Equiv{c}{\Skip}{b=B \land v=C \land e \land I }{b=B \land v<C \land I} \\
    \Equiv{c_1}{c_2}{\pre}{I \land \forall X, (I \land \neg e
      \Rightarrow \post)}
  \end{array}
}{
  \Equiv{c_1;\While{e}{c}}{c_2}{\pre}{\post}
}
\end{displaymath}


\subsection{Abstract adversaries: the \rawec{fun} tactic}
\index{prhl}{\rawec{fun}}

\paragraph*{ }
The formula given as parameter represents the general oracle
invariant. 
%
The tactic implements the following rule:
\begin{displaymath}
\infrule{
  \begin{array}{c}
    \pre \Rightarrow \chi \land \glob_A = \glob_B \land \vec{p}_A=\vec{p}_B
    \\[.5ex]
    \chi\land\glob_A=\glob_B\land\result_A=\result_B\Rightarrow\post
    \\ 
    \Equiv{O_i}{O_i'}{\chi\land
      \vec{p}_{O_i}=\vec{p}_{O'_i}}{\chi\land \result_{o_i}=\result_{o'_i}}
  \end{array}
}{
  \Equiv{A}{B}{\pre}{\post}
} [\mathec{fun}~\chi]
\end{displaymath}
%
where $\vec{p}_f$ represent the formal parameters of a function
(abstract adversary or oracle) $f$, $\result_f$ represents the result of
a function (abstract adversary or oracle) $f$, $\left\{O_i\right\}_{i=0}^k$ and
$\left\{O'_i\right\}_{i=0}^k$ are the oracles of the abstract adversaries $A$ and
$B$, $\glob_A$ and $\glob_B$ represent the global state of the abstract
adversaries $A$ and $B$.

\subsubsection{Verifying invariants upto bad}
\NotDocumented

\subsection{Strengthening goals}
\index{prhl}{\rawec{conseq}}

\subsubsection*{The \rawec{conseq} tactic}

\subsubsection*{The \rawec{conseq} tactic}
\index{prhl}{\rawec{conseq}}

\Syntax \rawec{conseq} (\_ : \textit{formula} \rawec{==>} \textit{formula} )

\begin{displaymath}
\infrule{
  \Equiv{c_1}{c_2}{\pre'}{\post'} \qquad \pre\Rightarrow\pre' \qquad  \post'\Rightarrow\post
}{
  \Equiv{c_1}{c_2}{\pre}{\post}
}\left[\mathec{conseq}~ ( \_ : \pre'~\mathec{==>}~ \post' \right]
\end{displaymath}


\subsubsection*{The \rawec{bypr} tactic}
\index{prhl}{\rawec{bypr}}
%

\begin{displaymath}
\infrule{
  \forall m_1,\ \forall m_2,\ 
  \Prm{f_1}{m_1}{\varphi_1} = \Prm{f_2}{m_2}{\varphi_2}
}{
  \Equiv{f_1}{f_2}{\pre}{\post}
}
\end{displaymath}









\subsection{Probability expressions: \rawec{deno} tactics}

\index{prhl}{\rawec{deno}}

\begin{displaymath}
\infrule{
  \Equiv{c_1}{c_2}{\pre}{\post} 
  \qquad
  \pre
  \qquad
  \post \Rightarrow \chi_1 \Rightarrow \chi_2
}{
  \Prm{c_1}{m_1}{\chi_1} \leq \Prm{c_2}{m_2}{\chi_2}
}\left[\mathec{deno}\ \pre\ \post\right]
\end{displaymath}

\begin{displaymath}
\infrule{
  \Equiv{c_1}{c_2}{\pre}{\post} 
  \qquad
  \pre
  \qquad
  \post \Rightarrow (\chi_1 \Leftrightarrow \chi_2)
}{
  \Prm{c_1}{m_1}{\chi_1} = \Prm{c_2}{m_2}{\chi_2}
}\left[\mathec{deno}\ \pre\ \post\right]
\end{displaymath}



%%% Local Variables: 
%%% mode: latex
%%% TeX-master: "easycrypt"
%%% End: 
