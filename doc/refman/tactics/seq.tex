% --------------------------------------------------------------------
\begin{tactic}{seq}
  \begin{tsyntax}{seq $\ n_1\ n_2$: R}
  Relational sequence rule.

  \paragraph{Examples:}\strut
  
  \begin{cmathpar}
  \texample[\prhl{}]
    {\ec{seq $\ \left|c_1\right|\ \left|c_2\right|$: R}}
    {\pRHL{P}{c_1}{c_2}{R} \\
     \pRHL{R}{c_1'}{c_2'}{Q}}
    {\pRHL{P}{c_1;c_1'}{c_2;c_2'}{Q}}
  \end{cmathpar}
  \end{tsyntax}

  \begin{tsyntax}{seq $\ n$: R}
  Non-relational possibilistic sequence rule.

  \paragraph{Examples:}\strut

  \begin{cmathpar}
  \texample[\hl{}]
    {\ec{seq $\ \left|c\right|$: R}}
    {\HL{P}{c}{R} \\ \HL{R}{c'}{Q}}
    {\HL{P}{c;c'}{Q}}
  \end{cmathpar}
  \end{tsyntax}

  \begin{tsyntax}{seq $\ n$: R $\ \delta_1\ \delta_2\ \delta_3\ \delta_4$ I}
  Non-relational probabilistic sequence rule. Argument \ec{I} is
  optional (and defaults to $\mathsf{true}$). When one of
  $(\delta_1,\delta_2)$ (resp. $(\delta_3,\delta_4)$) is 0, the other
  can be replaced with a wildcard \ec{_}, and the corresponding goal
  is not generated, as it is not relevant to the proof. When none of
  the $\delta$s are given, the following default values are used:
  $\delta_1 = 1$, $\delta_2 = \delta$, $\delta_3 = 0$.

  \paragraph{Examples:}\strut
  
  \begin{cmathpar}
  \texample[\phl{}]
    {\ec{seq $\ \left|c\right|$: R $\ \delta_1$ $\ \delta_2$ $\ \delta_3$ $\ \delta_4$ I}}
    {\HL{P}{c}{I} \\
     \pHL{P}{c}{R}{\diamond}{\delta_1}  \\
     \pHL{R \wedge I}{c'}{Q}{\diamond}{\delta_2} \\
     \pHL{P}{c}{\neg R}{\diamond}{\delta_3} \\
     \pHL{\neg R \wedge I}{c'}{Q}{\diamond}{\delta_4} \\
     \delta_1 \delta_2 + \delta_3 \delta_4 \diamond \delta}
    {\pHL{P}{c;c'}{Q}{\diamond}{\delta}}
  \end{cmathpar}

  \begin{cmathpar}
  \texample[\phl{}]
    {\ec{seq $\ \left|c\right|$: R $\ \delta_1$ $\ \delta_2\ \_\ 0$}}
    {\HL{P}{c}{\mathsf{true}} \\
     \pHL{P}{c}{R}{\diamond}{\delta_1} \\\\
     \pHL{R \wedge I}{c'}{Q}{\diamond}{\delta_2} \\
     \pHL{\neg R \wedge I}{c'}{Q}{\diamond}{0} \\
     \delta_1 \delta_2 \diamond \delta}
    {\pHL{P}{c;c'}{Q}{\diamond}{\delta}}
  \end{cmathpar}

  \textbf{Note:} Since most tactics implicitly apply the \rtactic{seq}
  rule, most \phl tactics take optional final arguments corresponding
  to the $\delta$s and \ec{I}.
  \end{tsyntax}
\end{tactic}
