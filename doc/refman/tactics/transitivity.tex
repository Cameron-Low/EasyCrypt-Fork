% --------------------------------------------------------------------
\begin{tactic}{transitivity}
  \begin{tsyntax}{transitivity c ($P_1$ ==> $\ Q_1$) ($P_2$ ==> $\ Q_2$)}
  In \prhl, applies the transitivity of program equivalence using the
  specified program and specifications. When the goal is a judgment on
  procedures, \ec{c} should be a procedure. When the goal is a
  judgment on statements, \ec{c} should be a statement, and the
  tactic then takes a side argument, used to decide the procedure
  context under which local variables from \ec{c} are evaluated.

  \textbf{Examples:}
  \begin{mathpar}
  \inferrule%%
    {\forall \mem{m_1}\ \mem{m_2}.\, \mem{m_1} \rel{P} \mem{m_2} \Rightarrow
        \exists \mem{m}.\, \mem{m_1} \rel{P_1} \mem{m}
                           \wedge \mem{m} \rel{P_2} \mem{m_2} \\%
     \forall \mem{m_1}\ \mem{m}\ \mem{m_2}.\,
        \mem{m_1} \rel{Q_1} \mem{m} \Rightarrow
        \mem{m}   \rel{Q_2} \mem{m_2} \Rightarrow
        \mem{m_1} \rel{Q}   \mem{m_2} \\%
     \pRHL{P_1}{f_1}{f}{Q_1} \\%
     \pRHL{P_2}{f}{f_2}{Q_2}}%%
    {\pRHL{P}{f_1}{f_2}{Q}}%%
    \quad\mbox{(\prhl)\quad\parbox{200pt}{\ec{transitivity f ($P_1$ ==> $\ Q_1$) ($P_2$ ==> $\ Q_2$)}}} \\
  \inferrule%%
    {\forall \mem{m_1}\ \mem{m_2}.\, \mem{m_1} \rel{P} \mem{m_2} \Rightarrow
        \exists \mem{m}.\, \mem{m_1} \rel{P_1} \mem{m}
                           \wedge \mem{m} \rel{P_2} \mem{m_2} \\%
     \forall \mem{m_1}\ \mem{m}\ \mem{m_2}.\,
        \mem{m_1} \rel{Q_1} \mem{m} \Rightarrow
        \mem{m}   \rel{Q_2} \mem{m_2} \Rightarrow
        \mem{m_1} \rel{Q}   \mem{m_2} \\%
     \pRHL{P_1}{s_1}{s}{Q_1} \\%
     \pRHL{P_2}{s}{s_2}{Q_2}}%%
    {\pRHL{P}{s_1}{s_2}{Q}}%%
    \quad\mbox{(\prhl)\quad\parbox{200pt}{\ec{transitivity$\{$1$\}$ $\ \{$ s $\ \}$ ($P_1$ ==> $\ Q_1$) ($P_2$ ==> $\ Q_2$)}}} \\
  \end{mathpar}

  \textbf{Note:} In practice, the existential quantification over
  memory $\mem{m}$ in the first generated subgoal is replaced with an
  existential quantification over the program variables appearing in $P$,
  $P_1$, ot $P_2$.
  \end{tsyntax}
\end{tactic}
