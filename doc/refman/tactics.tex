% --------------------------------------------------------------------
\section{Tactics}

\EasyCrypt comes with a proof engine that allows to state and prove properties
about programs written in the various languages.
%
Proofs are built interactively by applying tactics, that transform a
current proof goal into a (possibly empty) set of subgoals such that the
conjunction of the subgoals implies the original goal.
%
This process is repeated, starting from the theorem statement, up to the point
where all the subgoals correspond to general axioms or premises of the theorem.

In this chapter, we describe this proof engine in general before listing and
describing the existing tactics for various fragments of \EasyCrypt's
underlying logic.

\section{The proof engine}

The proof engine deals with judgments or goals of the form
$\epsilon; \Gamma \vdash \phi$ where $\epsilon$ is the (global) environment,
$\Gamma$ is the context (a set of local facts) and $\phi$ is the
formula we want to prove. Here is an example of such a judgment:

\begin{center}
$\mathsf{Int}; x, y , z: \tint, x \le y \vdash x + z \le y + z$.
\end{center}

It states that in the environment ($\mathsf{Int}$) solely composed of the
theory of integers, having three local variables $x, y, z$ of type
$\mathsf{int}$ along with the fact $x \le y$ (the context $\Gamma$), we are
interested in proving $x + z \le y + z$.

\medskip

On top on this, a set of deduction rules is given. They describe how one can
derive a judgment $\epsilon; \Gamma \vdash \phi$ given that a set of prerequisites
(or premises) are fulfilled. The general form of such a rule is given
as follow:

\begin{mathpar}
 \inferrule{A_1 \cdots A_n}{\epsilon; \Gamma \vdash \phi}
\end{mathpar}

It has to be read as: \emph{given that $A_1 \cdots A_n$ are derivable, then
so is $\epsilon, \Gamma \vdash \phi$}. We give three examples of such deduction
rules:

\begin{cmathpar}
 \inferrule*[left=MP]
         {\epsilon; \Gamma \vdash \phi_1 \quad
          \epsilon; \Gamma \vdash \phi_1 \Rightarrow \phi_2}
         {\epsilon; \Gamma \vdash \phi_2}

 \inferrule*[left=$\Rightarrow$-I]
         {\epsilon; \Gamma, \phi_1 \vdash \phi_2}
         {\epsilon; \Gamma \vdash \phi_1 \Rightarrow \phi_2}

 \inferrule*[left=Ax]{ }{\epsilon; \Gamma, \phi, \Delta \vdash \phi}
\end{cmathpar}

The first, the \emph{modus ponens}, states that one can derive
$\epsilon; \Gamma \vdash \phi_2$ given that $\epsilon; \Gamma \vdash \phi_1
\Rightarrow \phi_2$ and $\epsilon; \Gamma \vdash \phi_1$ are derivable.
%
The next provides a way for deriving $\phi_1 \Rightarrow \phi_2$ from
a derivation of $\phi_2$, but with a context augmented by $\phi_1$.
%
The last states that $\epsilon; \Gamma, \phi, \Delta \vdash \phi$ is derivable as-is.

Combining these deduction rules, it is possible to build a tree rooted by
a judgment $\epsilon; \Gamma \vdash \phi$ and with leaves composed of deduction
rules with no premises (as the third one in the previous example). Such a
tree forms a proof of $\epsilon; \Gamma \vdash \phi$.
%
For instance, Figure~\ref{fig:LJproof} gives a proof of
%
\[\epsilon; b_1, b_2 : \tbool \vdash (b_1 \Rightarrow b_2) \Rightarrow b_1 \Rightarrow b_2\]

\begin{figure}
  \begin{mathpar}
    \inferrule
      {\inferrule{ }{\epsilon; b_1, b_2 : \tbool, b_1 \Rightarrow b_2, b_1 \vdash b_1 \Rightarrow b_2} \quad
       \inferrule{ }{\epsilon; b_1, b_2 : \tbool, b_1 \Rightarrow b_2, b_1 \vdash b_1}}
      {\inferrule
        {\epsilon; b_1, b_2 : \tbool, b_1 \Rightarrow b_2, b_1 \vdash b_2}
        {\inferrule
           {\epsilon; b_1, b_2 : \tbool, b_1 \Rightarrow b_2 \vdash b_1 \Rightarrow b_2}
           {\epsilon; b_1, b_2 : \tbool \vdash (b_1 \Rightarrow b_2) \Rightarrow b_1 \Rightarrow b_2}}}
  \end{mathpar}

  \caption{\label{fig:LJproof} Proof tree of
    $\epsilon; b_1, b_2 : \tbool \vdash
        (b_1 \Rightarrow b_2) \Rightarrow b_1 \Rightarrow b_2$}
\end{figure}

The \EasyCrypt proof engine helps the user build such proofs. At each step
of the proof building, the system presents to the user the set of goals
that have to be proved. The user can then \emph{apply} a tactic to one of
them, each tactic corresponding to a deduction rule. If the conclusion
of the rule corresponding to the applied tactic matches the goal to which
it is applied, the proof engine replaces it with the set of the
premises of the applied rule - the subgoals. This application may generate
no, one or several subgoals depending on the rule. This process is repeated
iteratively, up to the point where no goals remain.

\subsection{Ambient logic}

% --------------------------------------------------------------------
\begin{tactic}{idtac}
  \begin{tsyntax}[empty]{idtac}
    Does nothing, i.e. keep the goal unchanged.
  \end{tsyntax}
\end{tactic}

% --------------------------------------------------------------------
\begin{tactic}[move | move: $\;\pi_1 \cdots \pi_n$]{move}
  \begin{tsyntax}{move}
     Does nothing, equivalent to \rtactic{idtac}. This form is mainly
     used in conjonction with an introduction pattern (see
     Section~\ref{s:intro-pattern}), e.g. \ls!move=> $\iota_1 \cdots \iota_n$!.
  \end{tsyntax}

  \begin{tsyntax}{move: $\;\pi_1 \cdots \pi_n$}
    Generalize the patterns $\pi_1, \cdots, \pi_n$, starting from
    $\pi_n$ and going back. See Section~\ref{s:gen-pattern} for more
    information on the generalization mechanism.
  \end{tsyntax}
\end{tactic}

% --------------------------------------------------------------------
\begin{tactic}{clear}
\end{tactic}

% --------------------------------------------------------------------
\begin{tactic}{done}
\end{tactic}

% --------------------------------------------------------------------
\begin{tactic}{apply}
\end{tactic}

% --------------------------------------------------------------------
\begin{tactic}{exact (p : proofterm)}
  \begin{tsyntax}[empty]{exact}
  Equivalent to \ec{by apply (p : proofterm)}, i.e. apply the given
  proof-term and the try to close the goals with \ec{trivial} - failing
  if not all goals can be closed.
  \end{tsyntax}
\end{tactic}

% --------------------------------------------------------------------
\begin{tactic}{assumption}
\end{tactic}

% --------------------------------------------------------------------
\begin{tactic}[pose x := $\;\pi$]{pose}
  \begin{tsyntax}[empty]{pose}
  Search for the first subterm \ec{p} of the goal matching $\pi$ and
  leading to the full instantiation of the pattern. Then introduce,
  after instantiation, the local definition \ec{x := p} and abstract
  all occurrences of \ec{p} in the goal as \ec{x}. An occurence
  selector can be used (see \rtactic{rewrite}).
  \end{tsyntax}
\end{tactic}

% --------------------------------------------------------------------
\begin{tactic}[cut $\;\iota$: $\;\phi$]{cut}
  \begin{tsyntax}[empty]{cut}
  Logical cut. Generate two subgoals: one for the cut formula $\phi$,
  and one for $\phi \Rightarrow G$ where $G$ is the current goal. Moreover,
  the intro-pattern \tct{$\iota$} is applied to the second subgoal.
  \end{tsyntax}
\end{tactic}


% --------------------------------------------------------------------
\begin{tactic}[rewrite $\;\pi_1 \cdots \pi_n$]{rewrite}
  \begin{tsyntax}[empty]{rewrite}
  Rewrite the rewrite-pattern $\pi_1 \cdots \pi_n$ from left to right,
  where the $\pi_i$ can be of the following form:
  \begin{itemize}
  \item one of \ec{//}, \ec{/=}, \ec{//=},
  \item a proof-term, or
  \item a pattern prefixed by \ec{/} (slash).
  \end{itemize}
  The two last forms can be prefixed by a direction indicator (the sign
  \ec{-}), followed by an occurrence selector (\ec{\{i1 ... in\}}),
  followed (for proof-terms only) by a repetition marker
  (\ec{!}, \ec{?}, \ec{n!} or \ec{n?}). All these prefixes are optional.

  Depending on the form of $\pi$, \ec{rewrite $\;\pi$} does the following:
    \begin{itemize}
    \item For \ec{//}, \ec{/=}, and \ec{//=}, see \ec{intros}.
    \item If \ec{rw} is a proof-term for the pattern
      \begin{center}
	\ec{forall (x1 : t1) ... (xn : tn), A1 -> ... -> An -> f1 = f2}
      \end{center}
      \noindent then \ec{rewrite} searches for the first subterm of the goal
      matching \ec{f1} and resulting in the full instantiation of the pattern.
      It then replaces, after instantiation of the pattern, all the occurrences
      of \ec{f1} by \ec{f2} in the goal, and creates $n$ new subgoals for the
      \ec{Ai}'s. If no subterms of the goal match \ec{f1} or if the pattern
      cannot be fully instantiated by matching, the tactic fails.
      The tactic works the same if the pattern ends by \ec{f1 <=> f2}. If the
      direction indicator \ec{-} is given, \ec{rewrite} works in the reverse
      direction, searching for a match of \ec{f2} and then replacing all
      occurrences of \ec{f2} by \ec{f1}.
    \item If \ec{rw} is a \ec{/}-prefixed pattern of the form \ec{(o p1 ... pn)},
      with \ec{o} a defined symbol, then \ec{rewrite} searches for the first subterm
      of the goal matching \ec{(o p1 ... pn)} and resulting in the full instantiation
      of the pattern. It then replaces, after instantiation of the pattern, all
      the occurrences of \ec{(o p1 ... pn)} by the $\beta\delta$ head-normal form
      of \ec{(o p1 ... pn)}, where the $\delta$-reduction is restricted to subterms
      headed by the symbol \ec{o}. If no subterms of the goal match \ec{(o p1 ... pn)} or
      if the pattern cannot be fully instantiated by matching, the tactic fails. If the
      direction indicator \ec{-} is given, \ec{rewrite} works in the reverse
      direction, searching for a match of the $\beta\delta_{\rm o}$ head-normal
      of \ec{(o p1 ... pn)} and then replacing all occurrences of this head-normal
      form with \ec{(o p1 ... pn)}.
    \end{itemize}
    
    \smallskip
    
    The occurrence selector \ec{\{i1 ... in\}} restricts which occurrences
    of the matching pattern are replaced in the goal. If given, only the
    \ec{i1}-th, ..., \ec{in}-th ones are replaced (considering that the goal is
    traversed in DFS mode). Note that this selection applies after the matching has
    been done.
    
    \medskip
    
    Repetition markers allow the repetition of the same rewriting. For instance,
    \ec{rewrite $\;\pi$} leads to \ec{do! rewrite $\;\pi$}. See \ec{do} for
    more information.
    
    \medskip

    Last, \ec{rewrite} \ec{rw}${}_1$ ... \ec{rw}${}_n$ is equivalent to
    \ec{rewrite} \ec{rw}${}_1$; ...; \ec{rewrite} \ec{rw}${}_n$.
  \end{tsyntax}
\end{tactic}

% --------------------------------------------------------------------
\begin{tactic}[subst | subst x]{subst}
  \begin{tsyntax}[empty]{subst}
  Search for the first equation of the form \tct{x = f} or \tct{f = x} in the context
  and replace all the occurrences of \tct{x} by \tct{f} everywhere in the context and the
  goal before clearing it. If no identifier is given, repeatedly apply the tactic to
  all identifiers for which such an equation exists.
  \end{tsyntax}
\end{tactic}


% --------------------------------------------------------------------
\begin{tactic}{split}
  \begin{tsyntax}[empty]{split}
  Break an intrinsically conjunctive goal into its component subgoals.
  For instance, it can:
  \begin{itemize}
    \item close any goal that is convertible to \tct{true} or provable by \tct{reflexivity},
    \item replace a logical equivalence by the direct and indirect implication,
    \item replace a goal of the form \tct{f1 /\\ f2} by the two subgoals for \tct{f1} an
          \tct{f2}. The same applies for a goal of the form \tct{f1 && f2},
    \item replace an equality between $n$-tuples by $n$ equalities
          on their components.
  \end{itemize}
  \end{tsyntax}
\end{tactic}

% --------------------------------------------------------------------
\begin{tactic}{left}
  \begin{tsyntax}[empty]{left}
  \fix{Missing description of left}.
  \end{tsyntax}
\end{tactic}

% --------------------------------------------------------------------
\begin{tactic}{right}
  \begin{tsyntax}[empty]{right}
  Reduce a disjunctive goal to its left member.
  \end{tsyntax}
\end{tactic}


% --------------------------------------------------------------------
\begin{tactic}{case}
\end{tactic}

% --------------------------------------------------------------------
\begin{tactic}[elim/$\phi$ $\;\;\pi_1 \cdots \pi_n$]{elim}
  \begin{tsyntax}[empty]{elim}
  Apply the elimination principle $\phi$ to the top assumption after
  having generalizing $\pi_1 \cdots \pi_n$.
  \end{tsyntax}
\end{tactic}


% --------------------------------------------------------------------
\begin{tactic}{simplify}
\end{tactic}

% --------------------------------------------------------------------
\begin{tactic}[change $\;\phi$]{change}
  \begin{tsyntax}[empty]{change $\;\phi$}
  Change the current goal for $\phi$ -- $\phi$ must be \emph{convertible}
  to the current goal.
  \end{tsyntax}
\end{tactic}

% --------------------------------------------------------------------
\begin{tactic}[progress | progress $\;\tau$]{progress}
  \begin{tsyntax}[empty]{progress}
  Break the goal into multiple \emph{simpler} ones by repeatedly applying
  \ec{split}, \ec{subst} and \ec{move=>}. The tactic $\tau$ given to
  \ec{progress} is tentatively applied after each step.
  \end{tsyntax}

  \fixme{Describe \ec{progress} options.}
\end{tactic}


% --------------------------------------------------------------------
\begin{tactic}{reflexivity}
  \begin{tsyntax}[empty]{reflexivity}
  Solve goals of the form \ec{b = b} (up to computation).
  \end{tsyntax}
\end{tactic}

% --------------------------------------------------------------------
\begin{tactic}{congr}
  \begin{tsyntax}[empty]{congr}
  Replace a goal of the form \tct{f t1 ... tn = f u1 ... un} with the subgoals
  \tct{ti = ui} for all \tct{i}. Subgoals solvable by \tct{reflexivity} are
  automatically closed.
  \end{tsyntax}
\end{tactic}

% --------------------------------------------------------------------
\begin{tactic}{algebra}
\end{tactic}


% --------------------------------------------------------------------
\begin{tactic}{trivial}
  \begin{tsyntax}[empty]{trivial}
  \fix{Missing description of trivial}.
  \end{tsyntax}
\end{tactic}

% --------------------------------------------------------------------
\begin{tactic}{smt}
  \begin{tsyntax}[empty]{smt}
  \fix{Missing description of smt}.
  \end{tsyntax}
\end{tactic}


% --------------------------------------------------------------------
\begin{tactic}{admit}
  \begin{tsyntax}[empty]{admit}
  Close the current goal by admitting it.
  \end{tsyntax}
\end{tactic}


\subsection{Tacticals}

Tactics can be combined together, composed and modified by tacticals. Tacticals do not
correspond to any deduction rule but make the proof process smoother, and sometimes permit
the reuse of proofs with similar patterns, but where the fine minutiae might differ.

\begin{tactic}[t1; t2]{sequence}
  \begin{tsyntax}[empty]{t1; t2}
  Execute \ec{t1} and then \ec{t2} on all the subgoals generated by \ec{t1}.
  \end{tsyntax}
\end{tactic}

\begin{tactic}[try t]{failure recovery}
  \begin{tsyntax}[empty]{try t}
  Execute the tactic \ec{t} if it succeeds; do nothing if it fails.

  \paragraph{Remark}
  By default, \EasyCrypt proofs are run in \ec{strict} mode. In this mode,
  \ec{smt} failures cannot be caught using \ec{try}. This allows \EasyCrypt
  to always build the proof tree correctly, even in weak check mode, where
  \ec{smt} calls are assumed to succeed. Inside a strict proof, weak check mode
  can be turned on and off at will, allowing for the fast replay of proof
  sections during development. In any event, we recommend \emph{never} using
  \ec{try smt}: a little thought is much more cost-effective than a bunch of
  \ec{smt}.
  \end{tsyntax}
\end{tactic}

\begin{tactic}[do! t]{tactic repetition}
  \begin{tsyntax}[empty]{do! t}
  Apply \ec{t} to the current goal, then repeatedly apply it to all subgoals,
  stopping only when it fails. An error is produced it \ec{t} does not apply to
  the current goal.
  \end{tsyntax}

  \paragraph{Variants}\strut

  \begin{tabularx}{\textwidth}{@{}ll@{}}
  {\ec{do ?t}} & apply {\ec{t}} 0 or more times, until it fails\\
  {\ec{do n !t}} & apply {\ec{t}} with depth exactly {\ec{n}}\\
  {\ec{do n ?t}} & apply {\ec{t}} with depth at most {\ec{n}}
  \end{tabularx}
\end{tactic}

\begin{tactic}[t1; first t2]{goal selection}
  \begin{tsyntax}[empty]{t1; first t2}
  Apply the tactic \ec{t1}, then apply \ec{t2} on the first subgoal
  generated by \ec{t1}. An error is produced if no subgoals have been
  generated by \ec{t1}.

  \paragraph{Variants}\strut

  \noindent\begin{tabularx}{\textwidth}{@{}ll@{}}
  {\ec{t1; first n t2}} & apply {\ec{t2}} on the first {\ec{n}} subgoals
    generated by {\ec{t1}}\\
  {\ec{t1; last t2}} & apply {\ec{t2}} on the last subgoal
    generated by {\ec{t1}}\\
  {\ec{t1; last n t2}} & apply {\ec{t2}} on the last {\ec{n}} subgoals
    generated by {\ec{t1}}\\
  {\ec{t; first n last}} & \parbox{200pt}{reorder the subgoals generated by {\ec{t}}, moving
    the first n to the end of the list}
  \end{tabularx}
  \end{tsyntax}
\end{tactic}

\begin{tactic}[by t]{closing goals}
  \begin{tsyntax}[empty]{by t}
  Apply the tactic \ec{t} and try to close all the generated subgoals using
  \rtactic{trivial}. Fail if not all subgoals can be closed.
  \end{tsyntax}
\end{tactic}

\subsection{Program Logics}

Judgments in the program logics may refer to procedures or
statements. Whenever the context allows both, we use $c$ (or \ec{c})
to denote programs, using $f$ (or \ec{f}) when only judgments on
procedures are allowed by the context, and $s$ (or \ec{s}) when only
judgments on statements are allowed.

\EasyCrypt includes three different program logics:
\begin{itemize}
\item \prhl, or probabilistic relational Hoare logic, with judgments of the form
%%
$$\pRHL{P}{c_1}{c_2}{Q}$$
%%
where $c_1$ and $c_2$ are programs, and $P$ and $Q$
are relations on memories.
\item \phl, or probabilistic Hoare logic, with judgments of the form
%%
$$\pHL{P}{c}{Q}{\diamond}{\delta}$$
%%
where $c$ is a program, $P$ and $Q$ are predicates on memories,
$\diamond\in\{\leq,\geq,=\}$ is a comparison relation and $\delta$ is
a real-valued expression, evaluated in the initial memory.
\item \hl, or (possibilistic) Hoare logic, with judgments of the form
%%
$$\HL{P}{c}{Q}$$
%%
where $c$ is a program, and $P$ and $Q$ are predicates on memories.
\end{itemize}

When $c$ is a procedure, preconditions ($P$ above) operate on memories
extended with a special $\Arg$ location that refers to the procedure's
arguments, and postcondition ($Q$ above) operate on memories extended
with a special $\Res$ location that refers to the procedure's return
value.

In the following, given a relation $R$, we denote with $\invrel{R}$
its inverse relation (that is,
%%
$m_1 \rel{R} m_2 \Leftrightarrow m_2 \rel{R^-1} m_1$).

We denote with $\diamond^{\uparrow}$ the function defined by
$$
\diamond^{\uparrow} =
\left\{\begin{array}{l l}
=    & \textit{if }\ \diamond = \Leftrightarrow \\
\leq & \textit{if }\ \diamond = \Leftarrow      \\
\geq & \textit{if }\ \diamond = \Rightarrow
\end{array}\right.
$$

Given a predicate $P$, we denote with $\inmem{P}{1}$
(resp. $\inmem{P}{2}$) the relation defined by
%%
$m_1 \rel{\inmem{P}{1}} m_2 \Leftrightarrow P~m_1$
(resp. $m_1 \rel{\inmem{P}{2}} m_2 \Leftrightarrow P~m_2$).
We lift logical connectors to predicates and relations over memories
in the natural way.

TODO: define \ec{<spec>}, \ec{<lemma>}, \ec{<prhl>}, \ec{<phl>},
\ec{<hl>}.

\paragraph{Reasoning on Specifications}
% --------------------------------------------------------------------
\begin{tactic}{symmetry}
\end{tactic}

% --------------------------------------------------------------------
\begin{tactic}{transitivity}
  \begin{tsyntax}{transitivity c ($P_1$ ==> $\ Q_1$) ($P_2$ ==> $\ Q_2$)}
  In \prhl, applies the transitivity of program equivalence using the
  specified program and specifications. When the goal is a judgment on
  procedures, \ec{c} should be a procedure. When the goal is a
  judgment on statements, \ec{c} should be a statement, and the
  tactic then takes a side argument, used to decide the procedure
  context under which local variables from \ec{c} are evaluated.

  \textbf{Examples:}
  \begin{mathpar}
  \inferrule%%
    {\forall \mem{m_1}\ \mem{m_2}.\, \Rel{P}{\mem{m_1}}{\mem{m_2}} \Rightarrow
        \exists \mem{m}.\, \Rel{P_1}{\mem{m_1}}{\mem{m}}
                           \wedge \Rel{P_2}{\mem{m}}{\mem{m_2}} \\%
     \forall \mem{m_1}\ \mem{m}\ \mem{m_2}.\,
        \Rel{Q_1}{\mem{m_1}}{\mem{m}} \Rightarrow
        \Rel{Q_2}{\mem{m}}{\mem{m_2}} \Rightarrow
        \Rel{Q}{\mem{m_1}}{\mem{m_2}} \\%
     \pRHL{P_1}{f_1}{f}{Q_1} \\%
     \pRHL{P_2}{f}{f_2}{Q_2}}%%
    {\pRHL{P}{f_1}{f_2}{Q}}%%
    \quad\mbox{(\prhl)\quad\parbox{200pt}{\ec{transitivity f ($P_1$ ==> $\ Q_1$) ($P_2$ ==> $\ Q_2$)}}} \\
  \inferrule%%
    {\forall \mem{m_1}\ \mem{m_2}.\, \Rel{P}{\mem{m_1}}{\mem{m_2}} \Rightarrow
        \exists \mem{m}.\, \Rel{P_1}{\mem{m_1}}{\mem{m}}
                           \wedge \Rel{P_2}{\mem{m}}{\mem{m_2}} \\%
     \forall \mem{m_1}\ \mem{m}\ \mem{m_2}.\,
        \Rel{Q_1}{\mem{m_1}}{\mem{m}} \Rightarrow
        \Rel{Q_2}{\mem{m}}{\mem{m_2}} \Rightarrow
        \Rel{Q}{\mem{m_1}}{\mem{m_2}} \\%
     \pRHL{P_1}{s_1}{s}{Q_1} \\%
     \pRHL{P_2}{s}{s_2}{Q_2}}%%
    {\pRHL{P}{s_1}{s_2}{Q}}%%
    \quad\mbox{(\prhl)\quad\parbox{200pt}{\ec{transitivity$\{$1$\}$ $\ \{$ s $\ \}$ ($P_1$ ==> $\ Q_1$) ($P_2$ ==> $\ Q_2$)}}} \\
  \end{mathpar}

  \textbf{Note:} In practice, the existential quantification over
  memory $\mem{m}$ in the first generated subgoal is replaced with an
  existential quantification over the program variables appearing in $P$,
  $P_1$, ot $P_2$.
  \end{tsyntax}
\end{tactic}

% --------------------------------------------------------------------
\begin{tactic}{conseq}
  \begin{tsyntax}{conseq <specification>}
  Rule of consequence. Proves a specification by weakening of a
  stronger result. Any one of the specification places can be filled
  with a wildcard \tct{_} to keep the value it contains in the current
  goal and trivially discharge the corresponding subgoal.

  \textbf{Examples:} In the following, $\leq^\uparrow$ (resp. $=^\uparrow$,
  $\geq^\uparrow$) is $\Leftarrow$ (resp. $\Leftrightarrow$ and
  $\Rightarrow$).
  \begin{mathpar}
  \inferrule*[left=(\prhl),rightskip=10em]%%
    {P' \Rightarrow P \\%
     Q \Rightarrow Q' \\%
     \pRHL{P}{c}{c'}{Q}}%%
    {\pRHL{P'}{c}{c'}{Q'}}%%
    \quad\raisebox{.7em}{\tct{conseq (_: P ==> Q)}} \\
  \inferrule*[left=(\prhl),rightskip=10em]%%
    {Q \Rightarrow Q' \\%
     \pRHL{P'}{c}{c'}{Q}}%%
    {\pRHL{P'}{c}{c'}{Q'}}%%
    \quad\raisebox{.7em}{\tct{conseq (_: _ ==> Q)}} \\
  \inferrule*[left=(\phl),rightskip=10em]%%
    {P' \Rightarrow \delta \mathrel{\diamond} \delta' \\%
     P' \Rightarrow P \\%
     Q \mathrel{\diamond^\uparrow} Q' \\%
     \pHL{P}{c}{Q}{\diamond}{\delta}}%%
    {\pHL{P'}{c}{Q'}{\diamond}{\delta'}}%%
    \quad\raisebox{.7em}{\tct{conseq (_: P ==> Q: $\delta$)}} \\
  \inferrule*[left=(\hl),rightskip=10em]%%
    {P' \Rightarrow P \\%
     Q \Rightarrow Q' \\%
     \HL{P}{c}{Q}}%%
    {\HL{P'}{c}{Q'}}%%
    \quad\raisebox{.7em}{\tct{conseq (_: P ==> Q)}} \\
  \end{mathpar}
  \end{tsyntax}

  \begin{tsyntax}{conseq <lemma>}
  Only applies to judgments on procedures. Same as \tct{conseq
  <specification>}, but the specification to use is inferred from
  the lemma provided. Raises an error if the lemma does not refer to
  the expected procedure(s). All variants of \tct{conseq} may take
  lemmas in place of explicit specifications with the same effect, in
  which case they must be applied to judgments on procedures.
  \end{tsyntax}

  \begin{tsyntax}{conseq* <spec>}
  Same as \tct{conseq}, but the subgoal corresponding to the
  postcondition is refined by a ``may modify'' analysis. All variants
  of \tct{conseq} can be refined using the \tct{*}, with the same
  effect.
  \end{tsyntax}

  \begin{tsyntax}{conseq <prhl> <hl> <hl>}
  Combine relational and non-relational specifications to prove a
  relational specification. Either one of the Hoare logic
  specifications can be replaced with a wildcard.

  \textbf{Examples:}
  \begin{mathpar}
  \inferrule*[left=(\prhl),rightskip=5em]%%
    {P' \Rightarrow P \wedge P_1\{1\} \wedge P_2\{2\} \\%
     Q \wedge Q_1\{1\} \wedge Q_2\{2\} \Rightarrow Q' \\%
     \HL{P_1}{c_1}{Q_1} \\%
     \HL{P_2}{c_2}{Q_2} \\%
     \pRHL{P}{c_1}{c_2}{Q}}%%
    {\pRHL{P'}{c_1}{c_2}{Q'}}%%
    \quad\raisebox{.7em}{\tct{conseq (_: P ==> Q) (_: P$_1$ ==> Q$_1$) (_: P$_2$ ==> Q$_2$)}}
  \end{mathpar}
  \end{tsyntax}

  \fix{Missing descriptions of combining variants of conseq}.
\end{tactic}

% --------------------------------------------------------------------
\begin{tactic}[case]{case-pl}
  \begin{tsyntax}{case e}
  Split the current \prhl, \phl or \hl goal by doing a case analysis
  in the precondition.

  \textbf{Example:}
  \begin{mathpar}
    \inferrule%%
      {\pRHL{P \wedge C}{c_1}{c_2}{Q} \\%
       \pRHL{P \wedge \neg C}{c_1}{c_2}{Q}}%%
      {\pRHL{P}{c_1}{c_2}{Q}}%%
      \quad\mbox{(\prhl)\quad\parbox{200pt}{\ec{case C}}} \\%%
    \inferrule%%
      {\pHL{P \wedge C}{c}{Q}{\diamond}{\delta} \\%
       \pHL{P \wedge \neg C}{c}{Q}{\diamond}{\delta}}%%
      {\pHL{P}{c}{Q}{\diamond}{\delta}}%%
      \quad\mbox{(\phl)\quad\parbox{200pt}{\ec{case C}}} \\%%
    \inferrule%%
      {\HL{P \wedge C}{c}{Q} \\%
       \HL{P \wedge \neg C}{c}{Q}}%%
      {\HL{P}{c}{Q}}%%
      \quad\mbox{(\hl)\quad\parbox{200pt}{\ec{case C}}} \\%%
  \end{mathpar}
  \end{tsyntax}
\end{tactic}

% --------------------------------------------------------------------
\begin{tactic}{phoare split}
  \begin{tsyntax}{phoare split $\delta_{A}$ $\delta_{B}$ $\delta_{AB}$}
  Splits a \phl judgment whose postcondition is a conjunction or
  disjunction into three \phl judgments following the definition of
  the probability of a disjunction of events.

  \paragraph{Examples:}\strut

  \begin{cmathpar}
  \texample[\phl{}]
    {\ec{phoare split $\ \delta_{A}\ \delta_{B}\ \delta_{AB}$}}
    {\delta_{A} + \delta_{B} - \delta_{AB} \diamond \delta \\
     \pHL{P}{c}{A}{\diamond}{\delta_{A}} \\
     \pHL{P}{c}{B}{\diamond}{\delta_{B}} \\
     \pHL{P}{c}{A \wedge B}{\invrel{\diamond}}{\delta_{AB}}}
    {\pHL{P}{c}{A \vee B}{\diamond}{\delta}}

  \texample[\phl{}]
    {\ec{phoare split $\ \delta_{A}\ \delta_{B}\ \delta_{AB}$}}
    {\delta_{A} + \delta_{B} - \delta_{AB} \diamond \delta \\
     \pHL{P}{c}{A}{\diamond}{\delta_{A}} \\
     \pHL{P}{c}{B}{\diamond}{\delta_{B}} \\
     \pHL{P}{c}{A \vee B}{\invrel{\diamond}}{\delta_{AB}}}
    {\pHL{P}{c}{A \wedge B}{\diamond}{\delta}}
  \end{cmathpar}
  \end{tsyntax}

  \begin{tsyntax}{phoare split $\ {!}$ $\ \delta_{\top}$ $\ \delta_{!}$}
  Splits a \phl judgment into two judgments whose postcondition are
  true and the negation of the original postcondition, respectively.

  \paragraph{Examples:}\strut

  \begin{cmathpar}
  \texample[\phl{}]
    {\ec{phoare split ! $\ \delta_{\top}\ \delta_{!}$}}
    {\delta_{\top} - \delta_{!} \diamond \delta \\
     \pHL{P}{c}{\mathsf{true}}{\diamond}{\delta_{\top}} \\
     \pHL{P}{c}{!Q}{\invrel{\diamond}}{\delta_{!}}}
    {\pHL{P}{c}{Q}{\diamond}{\delta}}
  \end{cmathpar}
  \end{tsyntax}

  \begin{tsyntax}{phoare split $\delta_{A}$ $\delta_{!A}$: A}
  Splits a \phl judgment following an event $A$.

  \paragraph{Examples:}\strut

  \begin{cmathpar}
  \texample[\phl{}]
    {\ec{phoare split $\ \delta_{A}\ \delta_{!A}$: A}}
    {\delta_{A} + \delta_{!A} \diamond \delta \\
     \pHL{P}{c}{Q \wedge A}{\diamond}{\delta_{A}} \\
     \pHL{P}{c}{Q \wedge \neg A}{\diamond}{\delta_{!A}}}
    {\pHL{P}{c}{Q}{\diamond}{\delta}}
  \end{cmathpar}
  \end{tsyntax}  
\end{tactic}

% --------------------------------------------------------------------
\begin{tactic}{byequiv}

  \begin{tsyntax}{byequiv [option]? <specification>}
  Derives probability relation from \prhl judgements. 
  Only applies to judgments on procedures.
 
  \textbf{Examples:}
  \begin{mathpar}
    \inferrule*[left=(\prhl),rightskip=5em]%%
    { \pRHL{P}{f_1}{f_2}{Q} \\%
      P~\vec{a}_1~m_1~\vec{a}_2~m_2 \\%
      Q \Rightarrow E_1\{1\}  \Leftrightarrow E_2\{2\} }%%
    { \PR{f_1}{\vec{a}_1}{\mem{m_1}}{E_1} = \PR{f_2}{\vec{a}_2}{m_2}{E_2} }%%
    \quad\raisebox{.7em}{\tct{byequiv (: P ==> Q)} } \\
    \inferrule*[left=(\prhl),rightskip=5em]%%
    { \pRHL{P}{f_1}{f_2}{Q} \\% 
      P~\vec{a}_1~m_1~\vec{a}_2~m_2 \\%
      Q \Rightarrow E_1\{1\}  \Rightarrow E_2\{2\} }%%
    { \PR{f_1}{\vec{a}_1}{\mem{m_1}}{E_1} \leq \PR{f_2}{\vec{a}_2}{m_2}{E_2} }%%
    \quad\raisebox{.7em}{\tct{byequiv (: P ==> Q)} } \\
    \inferrule*[left=(\prhl),rightskip=5em]%%
    { \pRHL{P}{f_1}{f_2}{Q} \\%
      P~\vec{a}_1~m_1~\vec{a}_2~m_2 \\%
      Q \Rightarrow E_2\{2\}  \Rightarrow E_1\{1\} } %%
    { \PR{f_1}{\vec{a}_1}{\mem{m_1}}{E_1} \geq \PR{f_2}{\vec{a}_2}{m_2}{E_2} }%%
    \raisebox{.7em}{\tct{byequiv (: P ==> Q)} } 
  \end{mathpar}
 
 \end{tsyntax}

  Possible options are \tct{-eq} or \tct{eq}.
  Any one of the specification places can be filled
  with a wildcard \tct{_}. It that case the corresponding argument 
  is automatically inferred. Some time the infered postcondition  
  is stronger than necessary, in that case use the option \tct{-eq}.

  \fix{Missing description of byequiv for upto}.
  
  \begin{tsyntax}{byequiv <lemma>}
  Same as \tct{byequiv <specification>}, but the specification to use is 
  inferred from the lemma provided. Raises an error if the lemma does 
  not refer to the expected procedures. All variants of \tct{byequiv} 
  may take lemmas in place of explicit specifications with the same effect.
  \end{tsyntax}


\end{tactic}

% --------------------------------------------------------------------
\begin{tactic}{byphoare}
  \begin{tsyntax}{byphoare [option]? <spec>}
  Derives a probability relation from a \phl judgement on the
  procedure involved. \ec{<spec>} can include wildcards when the
  tactic should infer the pre or postcondition.

  \textbf{Options:} By default, (\ec{eq} option) specification
  inference attempts to infer a conjunction of equalities sufficient
  to imply the desired relation. Passing the \ec{-eq} option
  overrides this behaviour, instead using the trivial relation on
  events.

  \paragraph{Examples:}\strut

  \begin{cmathpar}
    \texample
      {\ec{byphoare (_: P ==> Q)}}
      {\pHL{P}{f}{Q}{=}{\delta} \\
       P\ m[\Arg\mapsto\vec{a}] \\
       \forall \mem{m'}.\, Q\ m' \Leftrightarrow E\ m'}
      {\PR{f}{\vec{a}}{\mem{m}}{E} = \delta}
  \end{cmathpar}
  \end{tsyntax}

  \begin{tsyntax}{byphoare <lemma>}
  Same as \ec{byphoare <spec>}, but the specification to use is
  inferred from the lemma provided. Raises an error if the lemma does
  not refer to the expected procedure. Inference options have no
  effect in this setting.
  \end{tsyntax}
\end{tactic}

% --------------------------------------------------------------------
\begin{tactic}{hoare}
  \begin{tsyntax}[empty]{hoare}
  \fix{Missing description of hoare}.
  \end{tsyntax}
\end{tactic}

% --------------------------------------------------------------------
\begin{tactic}{bypr}
  \begin{tsyntax}[empty]{bypr}
  \fix{Missing description of bypr}.
  \end{tsyntax}
\end{tactic}

% --------------------------------------------------------------------
\begin{tactic}{exists*}
  \begin{tsyntax}[empty]{exists*}

  \end{tsyntax}
\end{tactic}

% --------------------------------------------------------------------
\begin{tactic}{elim*}
  \begin{tsyntax}[empty]{elim*}
  Destruct existential quantifications at the head of a
  precondition. Such existential quantifications may be introduced by
  \rtactic{sp} or \rtactic{exists*}.

  \paragraph{Examples:}\strut

  \begin{cmathpar}
  \texample[\prhl{}]{\ec{elim*}}%%
    {\forall x.\, \pRHL{x = \inmem{M.x}{1} \wedge P}{c_1}{c_2}{Q}}%%
    {\pRHL{\exists x, x = \inmem{M.x}{1} \wedge P}{c_1}{c_2}{Q}}

  \texample[\phl{}]{\ec{elim*}}%%
    {\forall x.\, \pHL{x = M.x \wedge P}{c}{Q}{\diamond}{\delta}}%%
    {\pHL{\exists x, x = M.x \wedge P}{c}{Q}{\diamond}{\delta}}

  \texample[\hl{}]{\ec{elim*}}%%
    {\forall x.\, \HL{x = M.x \wedge P}{c}{Q}}%%
    {\HL{\exists x, x = M.x \wedge P}{c}{Q}}  
  \end{cmathpar}

  \end{tsyntax}
\end{tactic}

% --------------------------------------------------------------------
\begin{tactic}{exfalso}
  \begin{tsyntax}{exfalso}
  Combines \rtactic{conseq}, \rtactic{byequiv}, \rtactic{byphoare},
  \rtactic{hoare} and \rtactic{bypr} to strengthen the precondition
  into $\mathsf{false}$ and discharge the resulting trivial goal.

  \paragraph{Examples:}\strut
  
  \begin{cmathpar}
  \texample[\prhl{}]
    {\ec{exfalso}}
    {P \Rightarrow \mathsf{false}}
    {\pRHL{P}{c}{c'}{Q}}

  \texample[\phl{}]
    {\ec{exfalso}}
    {P \Rightarrow \mathsf{false}}
    {\pHL{P}{c}{Q}{\diamond}{\delta}}

  \texample[\hl{}]
    {\ec{exfalso}}
    {P \Rightarrow \mathsf{false}}
    {\HL{P}{c}{Q}}
  \end{cmathpar}
  \fix{Move \ec{exfalso} to automatic tactics}?
  \end{tsyntax}
\end{tactic}

%% --------------------------------------------------------------------
\begin{tactic}{pr\_bounded}
  \begin{tsyntax}[empty]{pr\_bounded}
  \fix{Missing description of pr\_bounded}.
  \end{tsyntax}
\end{tactic}


\paragraph{Reasoning on Programs}
Unless specified, the following program logic tactics operate on a
program's last instruction. Although we describe these tactics as if
they operated on single instructions, their practical implementation
automatically and implicitly applies tactic \rtactic{seq} to deal with
context when necessary.

For simple proofs, it is often enough to simply apply the program
tactic corresponding to the last instruction in the program and let
\ec{smt} deal with the verification condition once the program has
been exhausted.

Most of the program reasoning tactics discussed in this paragraph have
two modes when used on \prhl proof obligations. Their default mode is
to operate on both programs at once. When a side is specified (using
\ec{<tactic>\{1\}} or \ec{<tactic>\{2\}}), a one-sided variant is
used. Apart from the \rtactic{if} tactic, the one-sided variant is in
fact a combination of the \phl tactic and \rtactic{conseq}.

\medskip

% --------------------------------------------------------------------
\begin{tactic}{skip}
  \begin{tsyntax}{skip} Hoare logic rules for empty statement.

  \paragraph{Examples:}\strut

  \begin{cmathpar}
  \texample[\hl{}]
    {\ec{skip}}
    {P \Rightarrow Q}
    {\pRHL{P}{\Skip}{\Skip}{Q}}

  \texample[\phl{}]
    {\ec{skip}}
    {P \Rightarrow Q}
    {\pHL{P'}{\Skip}{Q'}{\diamond}{1}}

  \texample[\hl{}]
    {\ec{skip}}
    {P \Rightarrow Q}
    {\HL{P}{\Skip}{Q}}
  \end{cmathpar}

  \textbf{Note:} Note that the \phl rule forces the bound of the goal
  to be 1. If you end up with an empty program and a bound other than
  1, you might want to use \rtactic{hoare} or \rtactic{conseq}. If
  neither of these work, you should probably have used \rtactic{seq}
  or \rtactic{phoare split} earlier on in your proof.
  \end{tsyntax}
\end{tactic}

% --------------------------------------------------------------------
\begin{tactic}{seq}
  \begin{tsyntax}{seq $\ n_1\ n_2$: R}
  Relational sequence rule.

  \paragraph{Examples:}\strut
  
  \begin{cmathpar}
  \texample[\prhl{}]
    {\ec{seq $\ \left|c_1\right|\ \left|c_2\right|$: R}}
    {\pRHL{P}{c_1}{c_2}{R} \\
     \pRHL{R}{c_1'}{c_2'}{Q}}
    {\pRHL{P}{c_1;c_1'}{c_2;c_2'}{Q}}
  \end{cmathpar}
  \end{tsyntax}

  \begin{tsyntax}{seq $\ n$: R}
  Non-relational possibilistic sequence rule.

  \paragraph{Examples:}\strut

  \begin{cmathpar}
  \texample[\hl{}]
    {\ec{seq $\ \left|c\right|$: R}}
    {\HL{P}{c}{R} \\ \HL{R}{c'}{Q}}
    {\HL{P}{c;c'}{Q}}
  \end{cmathpar}
  \end{tsyntax}

  \begin{tsyntax}{seq $\ n$: R $\ \delta_1\ \delta_2\ \delta_3\ \delta_4$ I}
  Non-relational probabilistic sequence rule. Argument \ec{I} is
  optional (and defaults to $\mathsf{true}$). When one of
  $(\delta_1,\delta_2)$ (resp. $(\delta_3,\delta_4)$) is 0, the other
  can be replaced with a wildcard \ec{_}, and the corresponding goal
  is not generated, as it is not relevant to the proof. When none of
  the $\delta$s are given, the following default values are used:
  $\delta_1 = 1$, $\delta_2 = \delta$, $\delta_3 = 0$.

  \paragraph{Examples:}\strut
  
  \begin{cmathpar}
  \texample[\phl{}]
    {\ec{seq $\ \left|c\right|$: R $\ \delta_1$ $\ \delta_2$ $\ \delta_3$ $\ \delta_4$ I}}
    {\HL{P}{c}{I} \\
     \pHL{P}{c}{R}{\diamond}{\delta_1}  \\
     \pHL{R \wedge I}{c'}{Q}{\diamond}{\delta_2} \\
     \pHL{P}{c}{\neg R}{\diamond}{\delta_3} \\
     \pHL{\neg R \wedge I}{c'}{Q}{\diamond}{\delta_4} \\
     \delta_1 \delta_2 + \delta_3 \delta_4 \diamond \delta}
    {\pHL{P}{c;c'}{Q}{\diamond}{\delta}}
  \end{cmathpar}

  \begin{cmathpar}
  \texample[\phl{}]
    {\ec{seq $\ \left|c\right|$: R $\ \delta_1$ $\ \delta_2\ \_\ 0$}}
    {\HL{P}{c}{\mathsf{true}} \\
     \pHL{P}{c}{R}{\diamond}{\delta_1} \\\\
     \pHL{R \wedge I}{c'}{Q}{\diamond}{\delta_2} \\
     \pHL{\neg R \wedge I}{c'}{Q}{\diamond}{0} \\
     \delta_1 \delta_2 \diamond \delta}
    {\pHL{P}{c;c'}{Q}{\diamond}{\delta}}
  \end{cmathpar}

  \textbf{Note:} Since most tactics implicitly apply the \rtactic{seq}
  rule, most \phl tactics take optional final arguments corresponding
  to the $\delta$s and \ec{I}.
  \end{tsyntax}
\end{tactic}

% --------------------------------------------------------------------
\begin{tactic}{sp}
  \begin{tsyntax}{sp}
  Computes the strongest postcondition of a straightline deterministic
  prefix of the program(s) that is implied by the current
  precondition. \tct{sp} also consumes deterministic \tct{if}
  statements (when both branches are deterministic straightline code
  without procedure calls).
  \end{tsyntax}

  \begin{tsyntax}{sp $\ n_1$ $\ n_2$}
  In \prhl, let \tct{sp} consume \emph{exactly} $n_1$ statements of
  the left program and $n_2$ statements of the right program.
  \end{tsyntax}

  \begin{tsyntax}{sp $\ n$}
  In \phl and \hl, let \tct{sp} consume \emph{exactly} $n$ statements
  of the program.
  \end{tsyntax}
\end{tactic}

% --------------------------------------------------------------------
\begin{tactic}{wp}
  \begin{tsyntax}{wp}
  Computes the weakest precondition of a straightline deterministic
  suffix of the program(s) that implies the current
  postcondition. \ec{wp} also consumes deterministic \ec{if}
  statements (when both branches are deterministic straightline code
  without procedure calls).
  \end{tsyntax}

  \begin{tsyntax}{wp $\ n_1$ $\ n_2$}
  In \prhl, let \ec{wp} consume \emph{exactly} $n_1$ statements of
  the left program and $n_2$ statements of the right program.
  \end{tsyntax}

  \begin{tsyntax}{wp $\ n$}
  In \phl and \hl, let \ec{wp} consume \emph{exactly} $n$ statements
  of the program.
  \end{tsyntax}
\end{tactic}

% --------------------------------------------------------------------
\begin{tactic}{rnd}
  \begin{tsyntax}[empty]{rnd}
  \fix{Missing description of rnd}.
  \end{tsyntax}
\end{tactic}

% --------------------------------------------------------------------
\begin{tactic}{if}
  \begin{tsyntax}[empty]{if}
  \fix{Missing description of if}.
  \end{tsyntax}
\end{tactic}

% --------------------------------------------------------------------
\begin{tactic}{while}
  \begin{tsyntax}[empty]{while}
  \fix{Missing description of while}.
  \end{tsyntax}
\end{tactic}

% --------------------------------------------------------------------
\begin{tactic}{call}
  All variants of the \ec{call} tactic implicitly make use of a frame
  rule, based on a ``may modify'' analysis.

  \begin{tsyntax}{call (_: P ==> Q)}
  Compute the precondition of a procedure call using the given
  specification for the procedure. As a side-goal, prove that the
  procedure fulfills the given specification.

  As with other tactics, the specification \ec{(_: P ==> Q)} can be
  replaced with a lemma from which the specification is inferred.
  \end{tsyntax}

  \begin{tsyntax}{call (_: I)}
  Uses invariant \ec{I} to infer a specification for use with
  \ec{tactic}.

  In \prhl, and if \ec{A} is the abstract module, \ec{call (_: I)}
  is equivalent to
  \ec{call (_: =$\{\Arg,glob A\}$ /\\ I ==> =$\{\Res,glob A\}$ /\\ I); first proc I.}

  In \phl and \hl, \ec{call (_: I)} is equivalent to
  \ec{call (_: I ==> I); first proc I.}
  \end{tsyntax}

  \begin{tsyntax}{call (_: B, I)}
  On \prhl abstract procedures only.
  If \ec{A} is the abstract module, \ec{call (_: B, I)} is equivalent to
  \ec{call (_: $\neg$B /\\ =$\{\Arg,glob A\}$ /\\ I ==> $\neg$B => =$\{\Res, glob A\}$ /\\ I); first proc B I.}
  \end{tsyntax}

  \begin{tsyntax}{call (_: B, I, I')}
  On \prhl abstract procedures only.
  If \ec{A} is the abstract module, \ec{call (_: B, I)} is equivalent to
  \ec{call (_: $\neg$B /\\ =$\{\Arg,glob A\}$ /\\ I ==> if $\neg$B then =$\{\Res,glob A\}$ /\\ I else I')); first proc B I I'.}
  \end{tsyntax}

  \textbf{Note:} When using the invariant-based variants of
  \ec{call}, error messages may be originating from the underlying
  application of \rtactic{proc}. In particular, when using them to
  deal with abstract procedure calls, the invariant \emph{should not}
  refer to memory locations the abstract procedure may modify.
\end{tactic}

% --------------------------------------------------------------------
\begin{tactic}{proc}
  \begin{tsyntax}[empty]{proc}
  \fix{Missing description of proc}.
  \end{tsyntax}
\end{tactic}



\paragraph{Transforming Programs}
% --------------------------------------------------------------------
\begin{tactic}{swap}
All versions of the tactic work for \prhl (an optional side can be given),
\phl and \hl.

\begin{tsyntax}{swap\ $p_1$\ $p_2$\ $p_3$}
  Swaps the code between positions $p_1$ and $p_2$ with the code between 
  positions $p_2$ and $p_3$. That is, assuming that $c_1$ and $c_2$ are 
  syntactically independent, that $c_1$ is between positions $p_1$ and $p_2$ 
  and that $c_2$ is between positions $p_2$ and $p_3$, the tactic implements 
  the following rule:

\begin{cmathpar}
\texample
  {\ec{swap} $\ p_1\ p_2\ p_3$}
  {\HL{P}{c;c_2;c_1;c_3}{Q}}
  {\HL{P}{c;c_1;c_2;c_3}{Q}}
\end{cmathpar}
\end{tsyntax}

\begin{tsyntax}{swap\ $k$}
If $k$ is positive (negative) then [\ec{swap} $k$] moves the first
(last) instruction $k$ positions forwards (backwards). 
\end{tsyntax}

\begin{tsyntax}{swap\ $p$\ $k$}
Moves the $p^{th}$ instruction forwards or backwards.
\end{tsyntax}

\begin{tsyntax}{swap\ [$p_1$..$p_2$]\ $k$}
Moves the instructions between positions $p_1$ and $p_2$ forwards or backwards.
\end{tsyntax}









 %%  \begin{tsyntax}[empty]{swap [$p_1$..$p_2$] n }
%%    swap 
%%   \end{tsyntax}
%% \textbf{where:} 
%% \begin{tabular}[t]{l}
%%   \textit{swap\_pos} ::= 
%%   \textit{n} \textit{n} \textit{n} $\mid$ \textit{n} \textit{z} $\mid$ [\textit{n}..\textit{n}] \textit{z}
%%   \\
%%   $n$ a natural number
%%   \\
%%   $z$ an integer number
%% \end{tabular}
  

%% The tactic [\rawec{swap} $p_1$ $p_2$ $p_3$] swaps the code between
%% positions $p_1$ and $p_2$ with the code between positions $p_2$ and
%% $p_3$. That is, assuming that $c_1$ and $c_2$ are syntactically
%% independent, that $c_1$ is between positions $p_1$ and $p_2$ and that
%% $c_2$ is between positions $p_2$ and $p_3$, the tactic implements the
%% following rule:
%% \begin{displaymath}
%% \infrule{
%%   \Hoare{c;c_2;c_1;c_3}{\pre}{\post}
%% }{
%%   \Hoare{c;c_1;c_2;c_3}{\pre}{\post}
%% } [\mathec{swap}\ p_1\ p_2\ p_3]
%% \end{displaymath}

%% If $k$ is positive (negative) then [\rawec{swap} $k$] moves the first
%% (last) instruction $k$ positions forwards (backwards). Similarly,
%% [\rawec{swap} $i$ $k$] moves the $i^{th}$ instruction forwards or
%% backwards, and [\rawec{swap} $[i_1:i_2]$ $k$] moves the instructions
%% between positions $i_1$ and $i_2$.

%%   \fix{Missing description of swap}.
%%   \end{tsyntax}
\end{tactic}

% --------------------------------------------------------------------
\begin{tactic}{inline}
  \begin{tsyntax}[empty]{inline}
  \fix{Missing description of inline}.
  \end{tsyntax}
\end{tactic}


% --------------------------------------------------------------------
\begin{tactic}{rcondf}
  \begin{tsyntax}[empty]{rcondf}
  \fix{Missing description of rcondf}.
  \end{tsyntax}
\end{tactic}

% --------------------------------------------------------------------
\begin{tactic}{rcondt}
  \begin{tsyntax}[empty]{rcondt}
  \fix{Missing description of rcondt}.
  \end{tsyntax}
\end{tactic}


% --------------------------------------------------------------------
\begin{tactic}{splitwhile}
  \begin{tsyntax}[empty]{splitwhile}
  \fix{Missing description of splitwhile}.
  \end{tsyntax}
\end{tactic}

% --------------------------------------------------------------------
\begin{tactic}{unroll}
\end{tactic}

% --------------------------------------------------------------------
\begin{tactic}{fission}
\end{tactic}

% --------------------------------------------------------------------
\begin{tactic}{fusion}
\end{tactic}


% --------------------------------------------------------------------
\begin{tactic}{alias}
\end{tactic}

% --------------------------------------------------------------------
\begin{tactic}{cfold}
\end{tactic}

% --------------------------------------------------------------------
\begin{tactic}{kill}
\end{tactic}

% --------------------------------------------------------------------
\begin{tactic}{modpath}
  \begin{tsyntax}[empty]{modpath}
  \fix{Missing description of modpath}.
  \end{tsyntax}
\end{tactic}


\paragraph{Automated Tactics}
% --------------------------------------------------------------------
\begin{tactic}{auto}
\end{tactic}

% --------------------------------------------------------------------
\begin{tactic}{sim}
  \begin{tsyntax}[empty]{sim}
  \fix{Missing description of sim}.
  \end{tsyntax}
\end{tactic}


\paragraph{Advanced Tactics}
% --------------------------------------------------------------------
\begin{tactic}{eager}
  \begin{tsyntax}[empty]{eager}
  \fix{Missing description of eager}.
  \end{tsyntax}
\end{tactic}

% --------------------------------------------------------------------
\begin{tactic}{fel}
\end{tactic}

