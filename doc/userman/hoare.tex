

\subsection{Reasoning on the program structure}

\subsubsection*{Empty statements: the \rawec{skip} tactic}
\index{hoare}{\rawec{skip}}

\Syntax \rawec{skip}

\Description Reduces logical program judgements with empty statements
to a higher-order logical goal. The generated subgoals can then be
processed using the ambient logic. The behaviour of the \rawec{skip}
tactic can be briefly described by the following rule:
%
\begin{displaymath}
\infrule{
  \pre \Rightarrow \post
}{
  \Hoare{\Skip}{\pre}{\post}
}
\end{displaymath}
%


\subsubsection*{Random samplings: the \rawec{rnd} tactic}
\index{hoare}{\rawec{rnd}}

This tactic allows dealing with random samplings for arbitrary
distribution expressions. It is supported by the Hoare Logic, the
Probabilistic Hoare Logic, and the Probabilistic Relational Logic. It
accepts different parameters and provides varied functionalities
depending on the goal under verification.

\Description

The application of this tactic expects a goal of the form:
\begin{displaymath}
  \Hoare{c;\Rand{x}{d}}{\pre}{\post}
\end{displaymath}
where $d$ is of type \rawec{'a Distr.distr} for some type \rawec{'a}.

Invocations to this tactic expect no arguments and applies the
following rule:
\begin{displaymath}
\infrule{
  \Hoare{c}{\pre}{\forall z,\insupp{z}{d} \Rightarrow \post\subst{x}{z}}
}{
  \Hoare{c;\Rand{x}{d}}{\pre}{\post}
}
\end{displaymath}
where $\insupp{z}{d}$ stands for the predicate $\ECinsupp{z}{d}$.


\subsubsection*{Sequential composition: the \rawec{seq} tactic}
\index{hoare}{\rawec{seq}}
%

\Syntax 
\rawec{seq} \textit{codepos} \textit{formula} 

\Description
Applies the Hoare Logic rule for sequential composition:
$$
\infrule{\Hoare{c}{\post}{\post'} \quad
         \Hoare{c'}{\post'}{\post''}}
        {\Hoare{c;c'}{\post}{\post''}}
$$
The application of tactic $\mathec{seq}\ k\ p$ defines $c$ as the first
$k$ instructions of the statement $c;c'$ and $\post'$ as
$p$.

\subsubsection*{Conditional statements: the \rawec{condt,condf} tactic}
\index{hoare}{\rawec{condt,condf}}
%
\NotDocumented


\subsubsection*{Conditional statements: the \rawec{if} tactic}
\index{hoare}{\rawec{if}}
%
Applies the following rule for conditional statements. It expects a
conditional statement at the first program position.
\begin{displaymath}
\begin{array}{c}
  \infrule{
    \Hoare{c_1;c}{\pre \land b}{\post}\qquad
    \Hoare{c_2;c}{\pre \land \neg b}{\post}
  }{
    \Hoare{\Cond{b}{c_1}{c_2};c}{\pre}{\post}
  }\left[\mathec{if} \right] 
\\[4ex]
\end{array}
\end{displaymath}

\subsubsection*{Deterministic straight-line code: the \rawec{wp} tactic}
\index{hoare}{\rawec{wp}}

\Syntax \rawec{wp} [\textit{codepos}]

\Description The \rawec{wp} tactic computes the weakest-precondition of
deterministic, loop and procedure-call free program fragments
(i.e. deterministic assignments and conditionals).   
The computation of the weakest precondition over a
conditional statement is only possible if its branches do not
contain random samplings, while-loops nor function calls.

The optional code position parameter \textit{pos} restricts the range
of instructions that may be affected by the tactic invocation. 
%
If the optional code position parameter is not provided, The tactic
processes instructions bottom-up until a random sampling, a loop or a
function call is reached.

\subsubsection*{Function call statements: the \rawec{call} tactic}
\index{hoare}{\rawec{call}}
%
\Syntax \rawec{call} \textit{formula} \textit{formula}
\Description

Let $p$ stand for the formal parameters of function $f$, $\result_f$
the result variable of function $f$, and $\vec{m}$ the set of
variables modifiable by $f$.
\begin{displaymath}
  \infrule{
    \begin{array}{c}
      \Hoare{c}{\pre}{\pre_f\subst{\vec{p}}{\vec{y}} \land
        \forall v.~ \forall \vec{z}.~ 
        \post_f\subst{\result_f}{v}\subst{\vec{m}}{\vec{z}}
        \Rightarrow \post\subst{x}{v}\subst{\vec{m}}{\vec{z}}
      }
      \\[.5ex]
      \Hoare{f}{\pre_f}{\post_f}
    \end{array}
  }{
    \Hoare{c;\Call{x}{f}{\vec{y}}}{\pre}{\post}
  }\left[\mathec{call}~ \pre_f~ \post_f \right]
\end{displaymath}



\subsubsection*{While loop statements: the \rawec{while} tactic}
\index{hoare}{\rawec{while}}

\Syntax

\Description


\subsection*{Abstract adversaries: the \rawec{fun} tactic}
\index{hoare}{\rawec{fun}}

\paragraph*{Verifying invariants}

\begin{displaymath}
\infrule{
    \pre \Rightarrow \chi  \qquad \chi \Rightarrow\post
    \qquad
    \Hoare{O_i}{\chi}{\chi}
}{
  \Hoare{A}{\pre}{\post}
} [\mathec{fun}~\chi]
\end{displaymath}


% \subsubsection{Verifying invariants upto bad}
% \NotDocumented



\subsection{Strengthening goals: the \rawec{conseq} tactic}
\index{hoare}{\rawec{conseq}}

\Syntax \rawec{conseq} (\_ : \textit{formula} \rawec{==>} \textit{formula} )

\begin{displaymath}
\infrule{
  \Hoare{c}{\pre'}{\post'} \qquad \pre\Rightarrow\pre' \qquad  \post'\Rightarrow\post
}{
  \Hoare{c}{\pre}{\post}
}\left[\mathec{conseq}~ \pre'~ \post' \right]
\end{displaymath}



\subsection{Possibilistic or probabilistic: the \rawec{hoare,hoare\_bd} tactic}
\index{hoare}{\rawec{hoare,hoare_db}}

\Syntax \mathec{hoare}, \mathec{hoare_bd}
allows to switch between possibilistic and probabilistic logics
according to these rules:
\begin{displaymath}
\infrule{
  \HoareEq{c}{\pre}{\neg\post}{0}
}{
  \Hoare{c}{\pre}{\post}
}
\end{displaymath}
