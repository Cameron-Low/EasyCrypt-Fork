
\chapter{\EasyCrypt language}


\section{Basic declarations}
\paragraph*{Types, constants, operators.}
\index{types}\index{constants}\index{operators}

\EasyCrypt provides native basic types such as \verb|unit|,
\verb|bool|, \verb|int|, \verb|real|, \verb|bitstring| as well as
polymorphic lists \verb|list|, polymorphic maps \verb|map|, product
types \verb!*! (infix notation), and \verb|option| types.
%
Abstract types can be declared with statements of the form 
\verb+type+~\textit{type_ident}, as in the following example:
\begin{verbatim} 
type secret_key.
type group.
\end{verbatim} 
Parametric type declarations are also supported. Type variables start
with a \verb|'| symbol:
\begin{verbatim}
type 'a list.
\end{verbatim}
%
Types synonyms can be declared with declaration of the form 
\verb+type+~\textit{type_ident}~\verb+=+~\textit{type_exp},
where \textit{type_exp} is built from basic types, type instantiation,
and other user-declared types, as in the following example:
\begin{verbatim} 
type secret_key = int.
type pkey = group.
type ciphertext = group * group. 
\end{verbatim} 

Constants are introduced with declarations of the form
\verb+cnst+~\textit{ident}\verb+:+~\textit{type_exp}~[\textit{exp}],
where \textit{exp} is an optional expression defining the constant.
For example, the following declarations introduce constants with
identifiers {\tt g} and {\tt empty_map} of types {\tt group} and {\tt
  ('a, 'b) map}, respectively:
\begin{verbatim}
cnst g : group.
cnst empty_map : ('a, 'b) map.
\end{verbatim}

Operators are introduced with declarations of the form
\verb+op+~\textit{op_ident}~\verb+:+~\textit{fun_type}~[\verb+as+~\textit{id}]
where the operator \textit{op_ident} can be either an alpha-numerical
identifier or a binary operator ---which may include extra symbols
such as \verb'=', \verb'<', \verb'~', \verb'+', \verb'%', and \verb'^'
for example--- enclosed in square backets.  The identifier
\textit{gt_int} is required when defining a binary operator enclosed
in brackets, and is used as an internal identifier following the
syntactic conventions of the tools in which \EasyCrypt relies.  The
signature \textit{fun_type} is defined with the syntax
\textit{type_exp}~\verb+->+~\textit{type_exp}, or
\verb+(+\textit{type_exp}${}_1$\verb+,+...\verb+,+\textit{type_exp}${}_k$\verb+)+~\verb+->+~\textit{type_exp},
where \textit{type_exp} stands for type expressions and 
\textit{type_exp}${}_1$\verb+,+...\verb+,+\textit{type_exp}${}_k$ is a
possibly empty list of type expressions.
%
For example:
\begin{verbatim}
op exp : real -> real
\end{verbatim}
The first operator is declared as infix and denoted by the symbol
\verb|>|. The operator \verb|exp| is a prefix operator. 
%
The definition of polymorphic operators is also allowed by the use of
type variables, e.g., the \verb+hd+ operator defined in the
\EasyCrypt prelude:
\begin{verbatim}
op hd : 'a list -> 'a.
\end{verbatim}
%
As well as constants, operators can be defined by an expression using
the following syntax:
\\
\verb+op+~\textit{op_ident}\verb+(+\textit{params}\verb+) = +\textit{exp}~[\verb+as+~\textit{id}]
\\
\noindent
notice that the result type is not required in this case.
The following are examples of operators defined in the \EasyCrypt prelude:
\begin{verbatim}
op fst(c : 'a * 'b) = let a,b = c in a.
op [>] (x,y:int) = y < x as gt_int.
\end{verbatim}


\paragraph*{Probabilistic operators.}\index{probabilistic operators}
Probability distributions (see random samplings in the definition of
probabilistic statements) can be defined by declaring operators with
the syntax \verb+pop+~\textit{ident}~\verb+:+~\textit{fun_type} where,
as well as in the definition of deterministic operators, the function
signature \textit{fun_type} is defined with the syntax
\textit{type_exp}~\verb+->+~\textit{type_exp}, or
\verb+(+\textit{type_exp}${}_1$\verb+,+...\verb+,+\textit{type_exp}${}_k$\verb+)+~\verb+->+~\textit{type_exp},
where \textit{type_exp} stands for type expressions and
\textit{type_exp}${}_1$\verb+,+...\verb+,+\textit{type_exp}${}_k$ is a
possibly empty list of type expressions.  For example:
\begin{verbatim}
pop gen_secret_key : int -> secret_key.
\end{verbatim}

\paragraph*{Logical formulae.}
Formulae are built from boolean expressions, standard logical
connectives, defined predicates, and logical variable
quantification. Boolean expressions are built by the application of
native or user-defined operators.

Logical formulae must be closed with respect to logical variables. The
syntax for universal quantification is of the form:
\begin{verbatim}
 forall (x,y:int,z:real), p(x,y,z)
\end{verbatim}
where \verb|p| is a first-order formula and \verb|x,y,z| are logical
variables, and similarly with existential quantification (\verb+exists+).

In addition to logical variables, in some contexts, predicates may
contain program variables tagged with a \verb|{1}| or \verb|{2}|
flag. A formula defining an axiom must contain only logical variables,
whereas formulae describing pre and postconditions on a relational
judgment (discussed below) usually refers to tagged program variables.


The special notation to specify that the states on the left and right
are equal over a subset of variables. For example, one can write
\verb|={x,y,z}| to denote the equivalent relational predicate
\begin{verbatim}
x{1}=x{2} && y{1}=y{2} && z{1}=z{2}
\end{verbatim}
% (the special keyword \verb|res| refers to function return value) 



\paragraph*{Predicates.}\index{predicates}
Predicates are introduced with the syntax
\verb+pred+~\textit{ident}\verb+(+\textit{params}\verb+)=+~\textit{p}
where \textit{params} is a list of formal argument declarations and
\textit{p} is a first-order non-relational formula. For example:
\begin{verbatim}
pred injective(T:('a, 'b) map) = 
  forall (x,y:'a), in_dom(x,T) => in_dom(y,T) => T[x] = T[y] => x = y.
\end{verbatim}



\paragraph*{Axioms and Lemmas.}\index{axioms}\index{lemmas}

Axioms are used to describe properties of abstract operators and
types, or to introduce hypotheses over declared constants. Axioms are
defined by a declaration of the form
\verb+lemma+~\textit{ident}~\verb+:+~\textit{p}, where \textit{ident}
is a valid identifier and $p$ is a first-order non-relational formula.
For example:
\begin{verbatim}
axiom head_def : forall (a: 'a, l: 'a list),  hd(a::l) = a.

axiom empty_in_dom : forall (a:'a), !in_dom(a, empty_map).
\end{verbatim}
%
The axiom \verb+head_def+ defines the list operator \verb+hd+. 
The axiom \verb+empty_in_dom+ characterizes \verb+empty_map+ as a
map with an empty domain.

Lemmas can also be introduced to facilitate the verification of later
goals. The syntax is similar to the one of axioms:
\verb+lemma+~\textit{ident}~\verb+:+~\textit{p}, where $p$ is a
first-order non-relational formula. 
%
When a lemma statement is found, \EasyCrypt proves it by calling the
available provers/SMT provers through the Why3 tool.


\section{Game declarations}
Games are defined by three components: variables describing the global
state, defined procedures and abstract adversary declarations.


\subsection{Probabilistic statements.}

Statements are defined as a list, possible empty, of basic
instructions (assignments and function calls) ending on a semicolon,
or composed instructions (conditional and while loops). No semicolon
is accepted after a conditional or loop
statement. Conditional statements follow the syntax 
%
\verb+if (+\textit{b}\verb+) {+ \textit{stmt} \verb+}+ where \textit{stmt} is
a probabilistic statement and \textit{b} is a boolean guard.  While
loop statements follow the syntax \verb+while (+\textit{b}\verb+) {+
  \textit{stmt} \verb+}+. Curly brackets are not required when
\textit{stmt} contains a single instruction.

Probabilistic assignments are of the form 
%
\verb+ident+ \verb+=+ \textit{d_exp}
%
where \textit{d_exp} is a probability expression, such as uniform
distributions over booleans (\verb+{0,1}+), integer intervals
\verb+[i..j]+, and bitstrings of arbitrary length
(\verb+{0,1}^k+), or distributions defined in terms of probabilistic
operators.  Assume \verb|gen_secret_key : int -> secret_key| is a
defined probabilistic operator, the following are valid probabilistic
assignments:
\begin{verbatim}
x = {0,1}
x = [0..q-1]
x = {0,1}^k
x = gen_secret_key(0)
\end{verbatim}


\subsection{Function Definition.}

Functions are defined either by a function body containing variable
declarations and probabilistic statements or as synonyms of functions
of already defined games.

\begin{itemize}
\item 
\verb+fun+ \textit{fun_ident} \verb+(+%
\textit{typed_args}\verb+) : +\textit{ret_type} \verb+ = { +%
\textit{fun_body} \verb+}+

\textit{fun_ident} is a valid function identifier, a list of typed
formal parameters \textit{typed_args}, the return type
\textit{ret_type} and its body \textit{fun_body}. The function body is
defined as a list of local variable declarations of the form
%
\verb+var+ \textit{ident} \verb+:+ \textit{type}\verb+;+, a
%
probabilistic statement, and a return instruction of the form
\verb+return+ \textit{exp}, where \textit{exp} is a deterministic
expression.

\item
\verb+fun+ \textit{fun_ident} \verb+=+ \textit{game_ident}\verb+.+\textit{fun_ident}

The resulting function has the same formal parameters and function
body than the function on the right.
\end{itemize}


\subsection{Adversary Signature and Declaration.}
Adversary signatures are defined outside a game declaration with a
syntax of the form:

\verb+adversary+
\textit{adv_sign_ident}\verb+(+\textit{typed_args}\verb+) :+
\textit{res_type} \verb+{+\textit{o_sign}${}_1$\verb+,+...\verb+,+\textit{o_sign}${}_k$\verb+}.+

\noindent
where \textit{res_type} is a type expression specifying the return
type and \textit{o_sign}${}_1$\verb+,+...\verb+,+\textit{o_sign}${}_k$
is a list (possibly empty), of oracle signatures.
In the following example
\begin{verbatim}
adversary A1_sign(pk:pkey)  : message * message { group -> message}.
adversary A2_sign(c:cipher) : bool              { group -> message}.
\end{verbatim}
the type expressions \verb|message*message| and \verb|bool| indicate
the return type. A list of signatures in square brackets indicates the
signature of the oracles that can be invoked by adversaries with these
signatures. In this particular example both signatures belong to
adversaries that can invoke a single oracle with type
\verb|group -> message|.

As well as function definition, adversaries are either declared abstractly
or as adversary synonyms.
Abstract declarations follow the syntax:\\
\verb+abs+ \textit{adv_ident} \verb+=+ \textit{adv_sign_ident}
\verb+{+ \textit{ident}${}_1$\verb+,+...\verb+,+\textit{ident}${}_k$\verb+}+

\noindent
For the adversary signature above we can write for example:
\begin{verbatim}
  abs A1 = A1_sign {H_A}
  abs A2 = A2_sign {H_A}
\end{verbatim}
where \verb|H_A| is a defined function representing an
oracle. Clearly, \EasyCrypt requires the function \verb|H_A| to have
the signature \verb|group -> message|.

Adversary synonyms follow a similar syntax to function synonyms:

\verb+fun+ \textit{adv_ident} \verb+=+ \textit{game_ident}\verb+.+\textit{adv_ident}

\noindent
The result of this declaration is, however, not necessarily an
abstract adversary.

\subsection{Game definition}


\begin{itemize}
\item A game can be defined by the following syntax:  
  \Syntax
  \verb+game+ \textit{ident} \verb+=+ \verb+{+\textit{game_body}\verb+}+
  % 
  The body of a game \textit{game_body} is composed of a global
  variable declaration, function definitions and abstract adversary
  declarations. The declaration of global variables consists of a list
  of statements of the form
  % 
  \verb+var+ \textit{ident} \verb+:+ \textit{type} 
  % 
  as in the definition of function local variables, except that they
  are not separated by a semicolon.
\item Alternatively, one can redefine a game by removing or adding
  variables, and redefining functions from an already defined game.
  \Syntax \verb+game+ \textit{ident} \verb+=+
  \textit{g_ident} \textit{var_modifs} \\
  \verb+           where+ \textit{ident${}_1$} \verb+= {+ \textit{fun_body} \verb+} and+ ...
  \verb+and+ \textit{ident${}_k$} \verb+= {+ \textit{fun_body} \verb+}+.

  The \textit{g_ident} identifier refers to an existing game,
  \textit{var_modifs} consists of an optional statement of the form
  \verb+remove+
  \textit{ident${}_1$}\verb+,+..\verb+,+\textit{ident${}_k$} and a
  possible empty list of new variable declarations. Finally, a list of
  function redefinitions is given separated by the \verb+and+ keyword.
\end{itemize}





\chapter{Probabilistic Relational Hoare Logic}

\section{Foundations}
Probabilistic Relational Hoare Logic (pRHL) judgments are quadruples
of the form:
%
$$ \Equiv{c_1}{c_2}{\Pre}{\Post} $$
%
where $c_1, c_2$ are programs and $\Pre, \Post$ are first-order
relational formulae. Relational formulae are first-order formulae over
logical variables and program variables tagged with either \verb|{1}|
or \verb|{2}| to denote their interpretation in the left or right-hand
side program. The special keyword \verb|res| denotes the return value
of a procedure and can be used in the place of a program variable. One
can also write \verb|e{i}| for the expression |e| in which all program
variables are tagged with \verb|{i}|. A relational formula is
interpreted as a relation on program memories.  See the related
articles~\cite{Barthe:2009} for more information on this logic.

\section{Judgements}
In \EasyCrypt, pRHL judgments are introduced with judgments
of the form
\begin{verbatim}
equiv Fact : Game1.f1 ~ Game2.f2 : Pre ==> Post.
\end{verbatim}
where \verb|Fact| is a judgment identifier, \verb!Game1! and
\verb!Game2! are games, \verb!f1! and \verb!f2! are identifiers for
procedures in \verb!Game1! and \verb!Game2! respectively. The
procedures \verb!f1! and \verb!f2! may be abstract or concrete;
however, judgments between two abstract procedures can only be defined
only if the two abstract procedures correspond to the same adversary.

The pre-condition \verb!Pre! and post-condition \verb!Post! are
relational formulae, and define relations between the parameters and
the global variables of the two procedures, the post-condition is a
relation between the global variables and a special variable named
\verb+res+, representing the return value of the procedures.  More
precisely \verb+res{1}+ stands for return value of the left procedure
and \verb+res{2}+ stands for the return value of the right
procedure. For convenience, \EasyCrypt also allows pre-conditions and
post-conditions to include sub-formulae of the form
\verb!={x1, ..., xn}! stating that the values of \verb!x1 ... xn!
  coincide in the left and right memories. That is,
  \verb!={x1, ..., xn}! is a shorthand for
  \verb!x1{1}=x1{2} && ... && xn{1}=xn{2}!.

\EasyCrypt also supports judgments of the form:
\begin{verbatim}
equiv Fact : Game1.f1 ~ Game2.f2 : (Inv).
\end{verbatim}
as a shorthand for 
\begin{verbatim}
equiv Fact : Game1.f1 ~ Game2.f2 : ={params} && Inv  ==>  ={res} && Inv.
\end{verbatim}
where \verb!params! is the list of parameters of \verb!f1! and
\verb!f2!. Note that in order for the judgment to be meaningful, the
procedures must have the same return type and the same signature type.






\section{Proof process}
A statement of the form 
\begin{verbatim}
equiv Fact : G1.f1 ~ G2.f2 : Pre ==> Post.
\end{verbatim}
opens a verification process, provided \verb!f1! and \verb!f2! are
both abstract procedures, or both concrete procedures. 


In case \verb!f1! and \verb!f2! are both abstract procedures, the only
available tactic is \verb!auto!. Note that, since abstract procedures
are allowed to call concrete procedures, it is sometimes useful to
prove invariants on the latter prior to proving equivalence properties
on \verb!f1! and \verb!f2!.


In case both procedures \verb!f1!  and \verb!f2! are concrete,
\EasyCrypt automatically transforms the judgment into a judgment on
their bodies. The pre-condition remains unchanged, but the
post-condition is modified by replacing the variables \verb+res{1}+
and \verb+res{2}+ by the return expressions of \verb!f1! and \verb!f2!
respectively.

For example, in the file \verb+examples/elgamal.ec+ after the
definition of the game \verb+DDH0+ we can start a new judgment,
stating that the two procedures \verb!INDCPA.Main! and
\verb! DDH0.Main! are equivalent if we observe their results
(\verb+={res}+ stands for \verb+res{1} = res{2}+):
\begin{verbatim}
equiv CPA_DDH0 : INDCPA.Main ~ DDH0.Main : true ==> ={res}.
\end{verbatim}
The judgment is automatically transformed into the following goal:
\begin{verbatim}
pre   = true
stmt1 =   1 : (sk, pk) = KG ();
          2 : (m0, m1) = A1 (pk);
          3 : b = {0,1};
          4 : mb = if b then m0 else m1;
          5 : c = Enc (pk, mb);
          6 : b' = A2 (pk, c);
stmt2 =   1 : x = [0..q - 1];
          2 : y = [0..q - 1];
          3 : d = B (g ^ x, g ^ y, g ^ (x * y));
post  = (b{1} = b'{1}) = d{2}
\end{verbatim}
At this point, the \EasyCrypt interpreter expects the user to provide
tactics to guide the verification of the judgment. Each tactic may
generate both logical verification goals (first-order formulae) that
are sent to SMT solvers and new verification subgoals that are stacked
for later verification by the user. The interactive verification task
concludes when there are no more goals in the stack and the result is
\emph{saved} (by typing \verb|save|) or when the verification goal is
\emph{aborted}.
%


Note that we have not implemented support to reason about the case
where one procedure is abstract, and another concrete. One possible
workaround is to wrap the abstract procedure, say \verb!f1!, into
a concrete procedure \verb!f1c! that simply calls \verb!f1!.  



 




\section{Tactics}
% !TeX root = easycrypt.tex

%% TODO (Francois): For index, rather than \texttt, use \rawec and make a class of keywords for tactics and tacticals

\chapter{Writing Proofs}

\EC comes with a proof engine that allows to state, in the \EC
underneath formalism, properties about the user defined programs
and to prove them.
%
Proofs are built interactively, starting from final goal, by applying
\emph{tactics} that transform a goal (the property we want to prove)
to a set of subsequent goals (the subgoals) s.t. the latter logical
implies the former.
%
This process is repeated iteratively up to the point where all the
subgoals are trivial and can be solved by the system.

This chapter is about the description of this proof engine, and is
structured as follow. We first define the notion of goals and show
how it relates to the \EC formalism. We then introduce the notion
of tactics as logically valid goal transformers. Finally, a listing
of all the existing tactics, along with their detailed descriptions,
is given.

\section{The proof engine}

The proof engine deals with \emph{judgments} or \emph{goals} of the form
$\Env; \Gamma \vdash \phi$ where $\Env$ is the (global) environments,
$\Gamma$ is a set of local facts and $\phi$ is the formula we want
to prove. Here is an example of such a judgment:

\begin{center}
$\Int; x, y , z: \tint, x \le y \vdash x + z \le y + z$.
\end{center}

It states that in the \emph{environment} ($\Env$) solely composed of the
theory $\Int$, having three local variables $x, y, z$ of type $\tint$ along
with the fact $x \le y$ (the \emph{context} $\Gamma$), we are interested
in proving $x + z \le y + z$.

\medskip

On top on this, a set of \emph{deduction rules} is given. They
describe how one can derive a judgment $\Env; \Gamma \vdash \phi$ given
that a set of prerequisites (or \emph{premises}) are fulfilled. The general
form of such a rule is given as follow:

\begin{displaymath}
 \infrule{A_1 \cdots A_n}{\Env; \Gamma \vdash \phi}
\end{displaymath}

It has to be read as: \emph{given that $A_1 \cdots A_n$ are derivable, then
so is $\Env, \Gamma \vdash \phi$}. We give three examples of such deduction
rules:

\begin{displaymath}
 \infrule
         {\Env; \Gamma \vdash \phi_1 \quad
          \Env; \Gamma \vdash \phi_1 \Rightarrow \phi_2}
         {\Env; \Gamma \vdash \phi_2}
         {\rname{MP}}
 \quad\quad
 \infrule
         {\Env; \Gamma, \phi_1 \vdash \phi_2}
         {\Env; \Gamma \vdash \phi_1 \Rightarrow \phi_2}
         {\rname{$\Rightarrow$-I}}
 \quad\quad
 \infrule{ }{\Env; \Gamma, \phi, \Delta \vdash \phi}{\rname{Ax}}
\end{displaymath}

The first, the \emph{modus ponens}, states that one can derive
$\Env; \Gamma \vdash \phi_2$ given that $\Env; \Gamma \vdash \phi_1
\Rightarrow \phi_2$ and $\Env; \Gamma \vdash \phi_1$ are derivable.
%
The next provides a way for deriving $\phi_1 \Rightarrow \phi_2$ from
a derivation of $\phi_2$, but with a context augmented by $\phi_1$.
%
The last states that $\Env; \Gamma, \phi, \Delta \vdash \phi$ is derivable as-is.

\medskip

Combining these deduction rules, it is possible to build a tree rooted by
a judgment $\Env; \Gamma \vdash \phi$ and with leaves composed of deduction
rules with no premises (as the third one in the previous example). Such a
tree forms a \emph{proof} of $\Env; \Gamma \vdash \phi$.
%
For instance, Figure~\ref{fig:LJproof} gives a proof of
%
\begin{center}
 $\Env; b_1, b_2 : \tbool \vdash (b_1 \Rightarrow b_2) \Rightarrow b_1 \Rightarrow b_2$
\end{center}

\begin{figure}
  \begin{displaymath}
    \infrule
      {\infrule{ }{\Env; b_1, b_2 : \tbool, b_1 \Rightarrow b_2, b_1 \vdash b_1 \Rightarrow b_2} \quad
       \infrule{ }{\Env; b_1, b_2 : \tbool, b_1 \Rightarrow b_2, b_1 \vdash b_1}}
      {\infrule
        {\Env; b_1, b_2 : \tbool, b_1 \Rightarrow b_2, b_1 \vdash b_2}
        {\infrule
           {\Env; b_1, b_2 : \tbool, b_1 \Rightarrow b_2 \vdash b_1 \Rightarrow b_2}
           {\Env; b_1, b_2 : \tbool \vdash (b_1 \Rightarrow b_2) \Rightarrow b_1 \Rightarrow b_2}}}
  \end{displaymath}

  \caption{\label{fig:LJproof} Proof tree of
    $\Env; b_1, b_2 : \tbool \vdash
        (b_1 \Rightarrow b_2) \Rightarrow b_1 \Rightarrow b_2$}
\end{figure}

\bigskip

The \EC proof engine helps the user building such proof. At each step
of the proof building, the system presents to the user the set of goals
that has to be proved. The user can then \emph{apply} a tactic to one of
them, each tactic corresponding to a deduction rule. If the conclusion
of the rule corresponding to the applied tactic matches the goal to witch
it is applied, the proof engine replaces it with the set of the
premises of the applied rule - the subgoals. This application may generate
no, one or several subgoals depending on the rule. This process is repeated
iteratively, up to the point where no goals remain.

\section{Ambient Logic (Guillaume)}

\begin{center}
\begin{tabular}{l@{$\quad$}l@{$\quad$}ll}
{\rawec{(lambda (x : t), phi1)\ phi2}} & $\rightarrow_\beta$ &
  \multicolumn{2}{@{}l}{{\rawec{phi2} \{\rawec{x} $\leftarrow$ \rawec{phi1}\}}}\\
{\rawec{if (true) \{ phi1 \} else \{ phi2 \}}} & $\rightarrow_\iota$ &
  {\rawec{phi1}}\\
{\rawec{if (false) \{ phi1 \} else \{ phi2 \}}} & $\rightarrow_\iota$ &
  \multicolumn{2}{@{}l}{{\rawec{phi2}}}\\
{\rawec{let (x1, ..., xn) = (phi1, ..., phin) in phi}} & $\rightarrow_\iota$ &
  \multicolumn{2}{@{}l}{{\rawec{phi} \{ \rawec{x1, ..., xn} $\leftarrow$ \rawec{phi1, ..., phin} \}}}\\
{\rawec{let x = phi1 in phi2}} & $\rightarrow_\zeta$ &
  \multicolumn{2}{@{}l}{{\rawec{phi2} \{ \rawec{x} $\leftarrow$ \rawec{phi1} \}}}\\
{\rawec{o}} & $\rightarrow_\delta^{\Env,\Gamma}$ &
  {\rawec{e}} & if {\rawec{op o := e}} $\in \Env$\\
{\rawec{x}} & $\rightarrow_\delta^{\Env,\Gamma}$ &
  {\rawec{phi}} & if {\rawec{x := phi}} $\in \Gamma$\\
\end{tabular}
\end{center}

\ambientDesc

\section{Program Transformation Tactics}

TODO: fun, inline, swap, unroll, splitwhile, fusion, fission, condt, condf, 

\section{Program Logics Tactics}

\subsection{: the \rawec{skip} tactic}

\Syntax \rawec{skip}

\Description Reduces logical program judgements with empty statements
to a first-order logical goal, as in the following rule for relational
Hoare Logic.
%
\begin{displaymath}
\infrule{
  \pre \Rightarrow \post
}{
  \equiv{}{}{\pre}{\post}
}
\end{displaymath}
%
Similar rules apply for Hoare judgements and probabilistic Hoare
judgements.

\subsection{Reasoning about random samplings: the \rawec{rnd} tactic}
%
\subsubsection{Hoare Logic}
\index{hoare}{Program Reasoning!rnd@\rawec{rnd}}

\Description

Assume $d:A\, \verb+distr+$...

\begin{displaymath}
\infrule{
  \Hoare{c}{\pre}{\forall z:A,in\_supp\,z\,d \Rightarrow \post\subst{x}{z}}
}{
  \Hoare{c;\Rand{x}{d}}{\pre}{\post}
}
\end{displaymath}

\subsubsection{Probabilistic Hoare Logic}
\index{phl}{Program Reasoning!rnd@\rawec{rnd}}
\Syntax 
\verb+rnd+ (\textit{formula} $|$ \_ ) (\textit{formula} $|$ \_ )

\Description
the first optional parameter $p$ is a computable predicate (i.e., \verb+'a cPred+)
(i.e., \verb+'a -> bool+ ). Assume $d$ of type \verb+A Distr.distr+. 
\begin{displaymath}
\begin{array}{c}
  \infrule{
    \Hoare{c}{\pre}{\mu\, d\, p \leq f \land 
      (\forall v\in \mathsf{support}(d).~ \post\subst{x}{v} \Rightarrow p\, v)}
  }{
    \HoareLe{c;\Rand{x}{d}}{\pre}{\post}{f}
  }\left[\verb+rnd+\ p\right]
\\[4ex]
\end{array}
\end{displaymath}
If $p$ is not given then the tool attempts to build it from $\post$
(not implemented yet).

For lower-bounded and exact probabilistic judgments the tactic
additionally accepts an optional parameter $g$ of type \verb+real+
representing a bound:
\begin{displaymath}
  \infrule{
    \HoareGe{c}{\pre}{\mu\, d\, p \geq g \land 
      (\forall v\in \mathsf{support}(d).~ p\, v \Rightarrow \post\subst{x}{v} )}{\frac{f}{g}} 
  }{
    \HoareGe{c;\Rand{x}{d}}{\pre}{\post}{f}
  }\left[\verb+rnd+\ p\ g\right]
\end{displaymath}
%
\begin{displaymath}
  \infrule{
    \HoareEq{c}{\pre}{\mu\, d\, p = g \land 
      (\forall v\in \mathsf{support}(d).~ p\, v \Leftrightarrow \post\subst{x}{v} )}{\frac{f}{g}} 
  }{
    \HoareEq{c;\Rand{x}{d}}{\pre}{\post}{f}
  }\left[\verb+rnd+\ p\ g\right]
\end{displaymath}
%
If $g$ is not given then $g=f$ in the rule.

\subsubsection{Relational Hoare Logic}
\index{prhl}{Program Reasoning!rnd@\rawec{rnd}}

\Syntax \verb+rnd+[\textit{side}] [\textit{bij\_info}]
where
\textit{bij\_info} is either
\begin{itemize}
  \item \textit{form} \textit{form}
  \item \textit{form} \_
\end{itemize}


\Description

The logical rule implemented by the \verb+rnd+ tactic depends on the
the optional parameter \textit{side}. If a left/right side flag is
provided then the one-sided logical rule for random sampling is
applied. If missing, then the two-sided rule for random assignment is
considered.
%

\paragraph*{Two-sided application.} 
In this case case, the \verb+rnd+ tactic takes as parameter a
representation of a bijective function. 

When two formulae are provided as the \textit{bij\_info} parameter,
they are verified to be a bijective function and its inverse. If only
one function is given then it is required to be an involution, and
lastly if no argument is given then the identity function is assumed.

The description of the rule below assumes that a bijective function
$f$ and its inverse is provided and generates according verification
conditions. Furthermore, it requires the following type constraints
for some types \verb+'a+ and \verb+'b+: 
\begin{itemize}
\item $d_1:\verb+'a distr+$,
\item $d_2:\verb+'b distr+$, $f:\verb+'a+\to\verb+'b+$,
\item $f^{-1}:\verb+'b+\to\verb+'a+$, 
\item ...
\end{itemize}

\begin{displaymath}
\infrule{
  \Equiv{c_1}{c_2}{\pre} 
  { \forall z,z',in\_supp \,z\,d_1\Rightarrow in\_supp \,z'\,d_2\Rightarrow
    \begin{array}{l}
      (\mu\,d_1\,\charfun_{\{z\}}=\mu\,d_2\,\charfun_{\{f\,z\}} ) 
      \land \\
      (in\_supp\,d_1\,(f^{-1}\,z'))
      \land \\
      (f^{-1}\,(f\,z)=z)
      \land \\
      (f\,(f^{-1}\,z')=z')
      \land \\
      (\post\subst{x_1}{z}\subst{x_2}{f\,z})
    \end{array}
  }
}{
  \Equiv{c_1;\Rand{x_1}{d_1}}{c_2;\Rand{x_2}{d_2}}{\pre}{\post}
}
\end{displaymath}

\paragraph*{Two-sided application.} 
The logical rule implemented when the optional parameter \textit{side}
is used is similar to the random sampling rule for Hoare judgements:


\begin{displaymath}
\infrule{
  \Equiv{c}{c'}{\pre}{\forall z:A,in\_supp\,z\,d \Rightarrow \post\subst{x}{z}}
}{
  \Equiv{c;\Rand{x}{d}}{c'}{\pre}{\post}
}
\end{displaymath}


\subsection{Reasoning about sequential composition: the \rawec{seq} tactic}
%
\subsubsection{Hoare Logic}
\index{hoare}{Program Reasoning!seq@\rawec{seq}}

\Syntax 
\verb+app+ \textit{codepos} \textit{formula} 

\Description

\Description
Applies the Hoare Logic rule for sequential composition:
$$
\infrule{\Hoare{c}{\post}{\post'} \quad
         \Hoare{c'}{\post'}{\post''}}
        {\Hoare{c;c'}{\post}{\post''}}
$$
The application of tactic \verb+app k p+ defines $c$ as the first
\verb+k+ instructions of the statement $c;c'$ and $\post'$ as
\verb+p+.


\subsubsection{Probabilistic Hoare Logic}
\index{phl}{Program Reasoning!seq@\rawec{seq}}
\Syntax 
\verb+app+ \verb+[>>|<<]+ \textit{codepos} \textit{formula} (
[\textit{formula} \verb+|+ \textit{formula} \textit{formula}
\textit{formula} \textit{formula}]

\Description
The application of the \verb+seq+ tactic is more complicated when
dealing with Probabilistic Hoare Logic judgements. 

The direction parameter is accepted for \emph{lower-bounded} and \emph{exact}
judgments. The direction \verb+<<+ is assumed by default (as it is globally).
The first formula represents the intermediate predicate that must hold
at the splitting program point.

In the following, the rule descriptions assume that $n$ indicates the
program position of statement $s_2$.

\paragraph*{Upper bounded judgements.}
For upper bounded judgments, the most general variant of the
\verb+app+ rule (i.e., when four bounds are given as parameters) implements the following rule:
\begin{displaymath}
  \infrule{
    \begin{array}{c}
      \HoareLe{s1}{P}{R}{f_1} \qquad \HoareLe{s2}{R}{Q}{f_2}
      \\
      \HoareLe{s1}{P}{R}{g_1} \qquad \HoareLe{s2}{R}{Q}{g_2}
      \\
      f_1 f_2 + g_1 g_2 \leq f 
    \end{array}
  }{
    \HoareLe{s1;s2}{P}{Q}{f}
  }
\end{displaymath}
%
If no argument is given then the following rule is applied:
\begin{displaymath}
  \infrule{
    \Hoare{s1}{P}{R} \qquad \HoareLe{s2}{R}{Q}{f}
  }{
    \HoareLe{s1;s2}{P}{Q}{f}
  }\left[\verb+app+\ n\ R\right]
\end{displaymath}
%
% \warningbox{Which, if preferred, can be rewritten to:}
% \begin{displaymath}
%   \infrule{
%     \Hoare{s1}{P}{\lambda m. \Prm{s_2}{m}{Q}\leq f} \qquad 
%   }{
%     \HoareLe{s1;s2}{P}{Q}{f}
%   }\left[\verb+app+\ n\ R\right]
% \end{displaymath}

Single bound parameters are not accepted for upper-bounded judgements.

\paragraph*{Lower-bounded and exact  judgements.}

The application of the \verb+seq+ tactic have similar regardless of
lower or exact bounds. 

The second optional parameter of type $\verb+real+$ represents a
probability bound (only supported for $=$ and $\geq$), and the
optional direction parameter indicates whether this bound is to be
applied to the first or second half of the split statement.

\begin{displaymath}
\begin{array}{c}
  \infrule{
    \HoareGe{s1}{P}{R}{f/g} \qquad \HoareGe{s2}{R}{Q}{g}
  }{
    \HoareGe{s1;s2}{P}{Q}{f}
  }\left[\verb+app+\ n\ R\ g\right]
\\[4ex]
  \infrule{
    \HoareGe{s1}{P}{R}{g} \qquad \HoareGe{s2}{R}{Q}{f/g}
  }{
    \HoareGe{s1;s2}{P}{Q}{f}
  }\left[\verb+app>>+\ n\ R\ g\right]
\end{array}
\end{displaymath}
%
%
Similar rules hold for $=$. If the bound parameter $g$ is not given then
$g$ is defined as $f$ in the above rule description.

\subsubsection{Relational Hoare Logic}
\index{prhl}{Program Reasoning!seq@\rawec{seq}}

\Syntax
\verb+app+ \textit{codepos} \textit{form}

\Description
Applies the RHL rule for sequential composition:
$$
\infrule{\Equiv{c_1}{c_2}{\post}{\post'} \quad
         \Equiv{c_1'}{c_2'}{\post'}{\post''}}
        {\Equiv{c_1;c_1'}{c_2;c_2'}{\post}{\post''}}[\textrm{R-Seq}]
$$
The application of tactic \verb+app m n p+ defines $c_1$ as the first
\verb+m+ instructions of the program on the left-hand side and $c_2$ as
the first \verb+n+ instructions of the program on the right-hand side
and $\post'$ as \verb+p+.



\subsection{Reasoning about conditionals: the \rawec{if} tactic}
%

\subsubsection{Hoare Logic}
\index{hoare}{Program Reasoning!if@\rawec{if}}
\index{phl}{Program Reasoning!if@\rawec{if}}

Applies the following rule for conditional statements. It expects a
conditional statement at the first program position.
\begin{displaymath}
\begin{array}{c}
  \infrule{
    \Hoare{c_1}{\pre \land b}{\post}\qquad
    \Hoare{c_2}{\pre \land \neg b}{\post}
  }{
    \Hoare{\Cond{b}{c_1}{c_2}}{\pre}{\post}
  }\left[\verb+if+ \right] 
\\[4ex]
\end{array}
\end{displaymath}


\subsubsection{Hoare and Probabilistic Hoare Logic}
\index{hoare}{Program Reasoning!if@\rawec{if}}
\index{phl}{Program Reasoning!if@\rawec{if}}

Applies the following rule for conditional statements. It expects a
conditional statement at the first program position.
\begin{displaymath}
\begin{array}{c}
  \infrule{
    \HoareLe{c_1}{\pre \land b}{\post}{f}\qquad
    \HoareLe{c_2}{\pre \land \neg b}{\post}{f}
  }{
    \HoareLe{\Cond{b}{c_1}{c_2}}{\pre}{\post}{f}
  }\left[\verb+if+ \right] 
\\[4ex]
\end{array}
\end{displaymath}
Similar rules hold for $=,\geq$.

\subsubsection{Relational Hoare Logic}
\index{prhl}{Program Reasoning!if@\rawec{if}}

\Syntax \verb+if+ [\textit{side}]

\Description Applies the pRHL rule for conditional.
If the \textit{side} argument is given then the corresponding
one side rule is used, else the two side rule is used.
The \verb+if+ tactic expects a conditional as first instruction. 
\begin{center}
\begin{tabular}{c|c}
Syntax & Rule \\
\hline\\
\verb+if{1}+ &
$
\infrule{\Equiv{c_1;c}{c'}{\pre \land e\sidel}{\post}
        \quad \Equiv{c_2;c}{c'}{\pre \land \neg e\sidel}{\post}}
        {\Equiv{\Cond{e}{c_1}{c_2};c}{c'}{\pre}{\post}}
$\\
\\\hline\\
\verb+if{2}+ &
$
\infrule{\Equiv{c'}{c_1;c}{\pre \land e\sider}{\post}
        \quad \Equiv{c'}{c_2;c}{\pre \land \neg e\sider}{\post}}
        {\Equiv{c'}{\Cond{e}{c_1}{c_2};c}{\pre}{\post}}
$\\
\\\hline\\
\verb+if+ &
$
\infrule{
 \begin{array}{c}
   \vdash \pre \Rightarrow e\sidel = e'\sider \\
   \Equiv{c_1;c}{c'_1;c'}{\pre \land e\sidel \land e'\sider}{\post}\\
   \Equiv{c_2;c}{c'_2;c'}{\pre \land \neg e\sidel \land \neg e'\sider}{\post}
 \end{array}
}{\Equiv{\Cond{e}{c_1}{c_2};c}
        {\Cond{e'}{c'_1}{c'_2};c'}
        {\pre}{\post}}
$\\
\end{tabular}
\end{center}


\subsection{Computing weakest preconditions: the \rawec{wp} tactic}
%

\Syntax \verb+wp+ [\textit{codepos}]

\Description The \verb+wp+ tactic computes the weakest-precondition of
deterministic, loop and procedure-call free program fragments
(i.e. deterministic assignments and conditionals).  If the op code
position parameter is not provided, The tactic processes instructions
bottom-up until a random sampling, a loop or a function call is
reached. The computation of the weakest precondition over a
conditional instruction is only possible if its branches do not
contain random samplings, while loops nor function calls.

The optional code position parameter \textit{pos} restricts the range
of instructions that may be affected by the tactic invocation. 
See \ref{???} for a description of its syntax.


\Example


\subsubsection{Hoare Logic}
\index{hoare}{Program Reasoning!wp@\rawec{wp}}

\subsubsection{Probabilistic Hoare Logic}
\index{phl}{Program Reasoning!wp@\rawec{wp}}

\begin{displaymath}
  \infrule{
    \HoareLe{c_1}{\pre }{\mathsf{wp}(c_2,\post)}{f}
  }{
    \HoareLe{c_1;c_2}{\pre}{\post}{f}
  }\left[\verb+wp+ \right] 
\end{displaymath}
Similar rules hold for $=,\geq$.

\subsubsection{Relational Hoare Logic}
\index{prhl}{Program Reasoning!wp@\rawec{wp}}

\subsection{Concluding proofs of programs: the \rawec{skip} tactic}
\index{hoare}{Program Reasoning!skip@\rawec{skip}}
\index{phl}{Program Reasoning!skip@\rawec{skip}}
\index{prhl}{Program Reasoning!skip@\rawec{skip}}
%

\subsection{Simplifying conditionals: the \rawec{condt,condf} tactic}
%
\subsubsection{Probabilistic Hoare Logic}
\index{tactics}{probabilistic Hoare logic!condt@\rawec{condt}}
\index{tactics}{probabilistic Hoare logic!condf@\rawec{condf}}

\subsection{Reasoning about abstract adversaries: the \rawec{fun} tactic}

\subsubsection{Relational Hoare Logic}

\Syntax \verb+fun+ formula

\Description
The formula given as parameter represents the general oracle
invariant. 

The tactic implements the following rule:
\begin{displaymath}
\infrule{
  \begin{array}{c}
    \pre \Rightarrow \chi \land \glob_A = \glob_B \land \vec{p}_A=\vec{p}_B
    \\[.5ex]
    \chi\land\glob_A=\glob_B\land\result_A=\result_B\Rightarrow\post
    \\ 
    \Equiv{O_i}{O_i'}{\chi\land
      \vec{p}_{O_i}=\vec{p}_{O'_i}}{\chi\land \result_{o_i}=\result_{o'_i}}
  \end{array}
}{
  \Equiv{A}{B}{\pre}{\post}
} [\verb+fun+~\chi]
\end{displaymath}
%
where $\vec{p}_f$ represent the formal parameters of a function
(abstract adversary or oracle) $f$, $\result_f$ represents the result of
a function (abstract adversary or oracle) $f$, $\left\{O_i\right\}_{i=0}^k$ and
$\left\{O'_i\right\}_{i=0}^k$ are the oracles of the abstract adversaries $A$ and
$B$, $\glob_A$ and $\glob_B$ represent the global state of the abstract
adversaries $A$ and $B$, ...

\subsubsection{Probabilistic Hoare Logic}
\begin{displaymath}
\infrule{
  \begin{array}{c}
    \pre \Rightarrow \chi  \qquad 
    \chi \Leftrightarrow\post
    \\[.5ex]
    \HoareEq{O_i}{\chi}{\chi}{1}
  \end{array}
}{
  \HoareEq{A}{\pre}{\post}{1}
} [\verb+fun+~\chi]
\end{displaymath}

\subsubsection{Hoare Logic}
\begin{displaymath}
\infrule{
  \begin{array}{c}
    \pre \Rightarrow \chi  \qquad 
    \chi \Rightarrow\post
    \\[.5ex]
    \Hoare{O_i}{\chi}{\chi}
  \end{array}
}{
  \Hoare{A}{\pre}{\post}
} [\verb+fun+~\chi]
\end{displaymath}

\subsection{??????: The \rawec{exfalso} rule}

\subsection{Frame rules ?? : The \rawec{eqobsin} rule}

\subsection{Weakening judgements: The \rawec{conseq} rule}

\Syntax \verb+conseq+ \textit{formula} \textit{formula}
\subsubsection{Hoare Logic}

\begin{displaymath}
\infrule{
  \Hoare{c}{\pre'}{\post'} \qquad \pre\Rightarrow\pre' \qquad  \post'\Rightarrow\post
}{
  \Hoare{c}{\pre}{\post}
}\left[\verb+conseq+~ \pre'~ \post' \right]
\end{displaymath}

\subsubsection{Probabilistic Hoare Logic}
\begin{displaymath}
\infrule{
  \HoareLe{c}{\pre'}{\post'}{\delta} \qquad \pre\Rightarrow\pre' \qquad  \post\Rightarrow\post'
}{
  \HoareLe{c}{\pre}{\post}{\delta}
}\left[\verb+conseq+~ \pre'~ \post' \right]
\end{displaymath}

\begin{displaymath}
\infrule{
  \HoareEq{c}{\pre'}{\post'}{\delta} \qquad \pre\Rightarrow\pre' \qquad  \post\Leftrightarrow\post'
}{
  \HoareEq{c}{\pre}{\post}{\delta}
}\left[\verb+conseq+~ \pre'~ \post' \right]
\end{displaymath}

\begin{displaymath}
\infrule{
  \HoareGe{c}{\pre'}{\post'}{\delta} \qquad \pre\Rightarrow\pre' \qquad  \post'\Rightarrow\post
}{
  \HoareGe{c}{\pre}{\post}{\delta}
}\left[\verb+conseq+~ \pre'~ \post' \right]
\end{displaymath}

\warningbox{(changing the bound is not yet implemented)}

\subsubsection{Relational Hoare Logic}

\begin{displaymath}
\infrule{
  \Equiv{c_1}{c_2}{\pre'}{\post'} \qquad \pre\Rightarrow\pre' \qquad  \post'\Rightarrow\post
}{
  \Equiv{c_1}{c_2}{\pre}{\post}
}\left[\verb+conseq+~ \pre'~ \post' \right]
\end{displaymath}


\subsection{Reasoning about function calls: the \rawec{call} tactic}
%
\subsubsection{Hoare Logic}
\index{hoare}{Program Reasoning!call@\rawec{call}}
\Syntax \verb+call+ formula formula
\Description

Let $p$ stand for the formal parameters of function $f$, $\result_f$
the result variable of function $f$, and $\vec{m}$ the set of
variables modifiable by $f$.
\begin{displaymath}
  \infrule{
    \begin{array}{c}
      \Hoare{c}{\pre}{\pre_f\subst{\vec{p}}{\vec{y}} \land
        \forall v.~ \forall \vec{z}.~ 
        \post_f\subst{\result_f}{v}\subst{\vec{m}}{\vec{z}}
        \Rightarrow \post\subst{x}{v}\subst{\vec{m}}{\vec{z}}
      }
      \\[.5ex]
      \Hoare{f}{\pre_f}{\post_f}
    \end{array}
  }{
    \Hoare{c;\Call{x}{f}{\vec{y}}}{\pre}{\post}
  }\left[\verb+call+~ \pre_f~ \post_f \right]
\end{displaymath}



\subsubsection{Probabilistic Hoare Logic}
\index{phl}{probabilistic Hoare logic!call@\rawec{call}}

\Syntax \verb+call+ formula formula [formula]

\Description

Let $p$ stand for the formal parameters of function $f$, $\result_f$
the result variable of function $f$, and $\vec{m}$ the set of
variables modifiable by $f$.
\begin{displaymath}
  \infrule{
    \begin{array}{c}
      \Hoare{c}{\pre}{\pre_f\subst{\vec{p}}{\vec{y}} \land
        \forall v.~ \forall \vec{z}.~ 
        \post_f\subst{\result_f}{v}\subst{\vec{m}}{\vec{z}}
        \Rightarrow \post\subst{x}{v}\subst{\vec{m}}{\vec{z}}
      }
      \\[.5ex]
      \HoareLe{f}{\pre_f}{\post_f}{\delta}
    \end{array}
  }{
    \HoareLe{c;\Call{x}{f}{\vec{y}}}{\pre}{\post}{\delta}
  } \left[\verb+call+~ \pre_f~ \post_f \right]
\end{displaymath}

\begin{displaymath}
  \infrule{
    \begin{array}{c}
      \HoareEq{c}{\pre}{\pre_f\subst{\vec{p}}{\vec{y}} \land
        \forall v.~ \forall \vec{z}.~ 
        \post_f\subst{\result_f}{v}\subst{\vec{m}}{\vec{z}}
        \Rightarrow \post\subst{x}{v}\subst{\vec{m}}{\vec{z}}}{\frac{\delta}{\delta'}}
    \\[.5ex]
    \HoareEq{f}{\pre_f}{\post_f}{\delta'}
  \end{array}
  }{
    \HoareEq{c;\Call{x}{f}{\vec{y}}}{\pre}{\post}{\delta}
  } \left[\verb+call+~ \pre_f~ \post_f~ \delta' \right]
\end{displaymath}

\begin{displaymath}
  \infrule{
    \begin{array}{c}
      \HoareGe{c}{\pre}{\pre_f\subst{\vec{p}}{\vec{y}} \land
        \forall v.~ \forall \vec{z}.~ 
        \post_f\subst{\result_f}{v}\subst{\vec{m}}{\vec{z}}
        \Rightarrow \post\subst{x}{v}\subst{\vec{m}}{\vec{z}}}
      {\frac{\delta}{\delta'}}
    \\[.5ex]
    \HoareGe{f}{\pre_f}{\post_f}{\delta'}
  \end{array}
  }{
    \HoareGe{c;\Call{x}{f}{\vec{y}}}{\pre}{\post}{\delta}
  } \left[\verb+call+~ \pre_f~ \post_f ~\delta' \right]
\end{displaymath}

If no parameter is given for the lower-bounded and exact judgements
then $\delta'=1$.

\warningbox{New tactics, needs structuring.}

\subsection{: the \rawec{hoare,hoare\_bd,pr\_bounded,bd\_eq}}

\subsubsection{Possibilistic and probabilistic Hoare Logic}
\Syntax \verb+hoare+, \verb+hoare_bd+
allows to switch between possibilistic and probabilistic logics
according to these rules:
\begin{displaymath}
\begin{array}{cc}
\infrule{
  \Hoare{c}{\pre}{\neg \post} \quad f = 0
}{
  \HoareEq{c}{\pre}{\post}{f}
}
&
\infrule{
  \HoareEq{c}{\pre}{\neg\post}{0}
}{
  \Hoare{c}{\pre}{\post}
}
\end{array}
\end{displaymath}

\Syntax \verb+pr_bounded+
discharges goals by applying trivial probability properties:
\begin{displaymath}
\begin{array}{cc}
\infrule{
}{
  \HoareLe{c}{\pre}{\post}{1}
}
&
\infrule{
}{
  \HoareGe{c}{\pre}{\post}{0}
}
% \\[3ex]
% \infrule{
% }{
%   \Prm{c}{m}{\post} \leq 1
% }
% &
% \infrule{
% }{
%   \Prm{c}{m}{\post} \geq 0
% }
\end{array}
\end{displaymath}

\Syntax \verb+bd_eq+
\begin{displaymath}
\begin{array}{cc}
\infrule{
  \HoareEq{c}{\pre}{\post}{f}
}{
  \HoareLe{c}{\pre}{\post}{f}
}
&
\infrule{
  \HoareEq{c}{\pre}{\post}{f}
}{
  \HoareGe{c}{\pre}{\post}{f}
}
\end{array}
\end{displaymath}


\subsection{\rawec{Denot} tactics}
%
\subsubsection{Probabilistic Hoare Logic}

\begin{displaymath}
\infrule{
    \pre 
    \qquad 
    \chi\Rightarrow\post 
    \qquad 
    \HoareLe{f}{\pre}{\post}{\delta}
}{
  \Prm{c}{m}{\chi} \leq \delta
}\left[\verb+hoare_deno+\ \pre\ \post\right]
\end{displaymath}

\begin{displaymath}
\infrule{
    \pre 
    \qquad 
    \post\Leftrightarrow \chi 
    \qquad 
    \HoareEq{f}{\pre}{\post}{\delta}
}{
  \Prm{c}{m}{\chi} = \delta
}\left[\verb+hoare_deno+\ \pre\ \post\right]
\end{displaymath}

\begin{displaymath}
\infrule{
    \pre 
    \qquad 
    \post\Rightarrow\chi
    \qquad 
    \HoareGe{f}{\pre}{\post}{\delta}
}{
  \delta \leq \Prm{c}{m}{\chi}
}\left[\verb+hoare_deno+\ \pre\ \post\right]
\end{displaymath}


\subsubsection{Relational Hoare Logic}

\begin{displaymath}
\infrule{
  \Equiv{c_1}{c_2}{\pre}{\post} 
  \qquad
  \pre
  \qquad
  \post \Rightarrow \chi_1 \Rightarrow \chi_2
}{
  \Prm{c_1}{m_1}{\chi_1} \leq \Prm{c_2}{m_2}{\chi_2}
}\left[\verb+deno+\ \pre\ \post\right]
\end{displaymath}

\begin{displaymath}
\infrule{
  \Equiv{c_1}{c_2}{\pre}{\post} 
  \qquad
  \pre
  \qquad
  \post \Rightarrow (\chi_1 \Leftrightarrow \chi_2)
}{
  \Prm{c_1}{m_1}{\chi_1} = \Prm{c_2}{m_2}{\chi_2}
}\left[\verb+deno+\ \pre\ \post\right]
\end{displaymath}


\subsection{Some \textsf{Pr} tactics: \rawec{pr\_false},
  \rawec{pr\_or}}

\begin{displaymath}
\infrule{
  \false \Rightarrow \post
}{
  \Prm{c}{m}{\post} = 0
}
\end{displaymath}

\begin{displaymath}
\infrule{
\Prm{c}{m}{\pre} \land
  \Prm{c}{m}{\post} \land \Prm{c}{m}{\pre \wedge \post} = \delta
}{
  \Prm{c}{m}{\pre \vee \post} = \delta
}
\end{displaymath}


\subsection{The \rawec{inline} tactic}
%

\subsection{The \rawec{swap} tactic}
%
\Syntax \verb+swap+ [\textit{side}] \textit{swap\_pos}

\textbf{where:} 
\begin{tabular}[t]{l}
  \textit{swap\_pos} ::= 
  \textit{n} \textit{n} \textit{n} $\mid$ \textit{n} \textit{z} $\mid$ [\textit{n}:\textit{n}] \textit{z}
  \\
  $n$ a natural number
  \\
  $z$ an integer number
\end{tabular}
  

The tactic [\verb+swap+ $p_1$ $p_2$ $p_3$] swaps the code between
positions $p_1$ and $p_2$ with the code between positions $p_2$ and
$p_3$. That is, assuming that $c_1$ and $c_2$ are syntactically
independent, that $c_1$ is between positions $p_1$ and $p_2$ and that
$c_2$ is between positions $p_2$ and $p_3$, the tactic implements the
following rule:
\begin{displaymath}
\infrule{
  \Hoare{c;c_2;c_1;c_3}{\pre}{\post}
}{
  \Hoare{c;c_1;c_2;c_3}{\pre}{\post}
} [\verb+swap+\ p_1\ p_2\ p_3]
\end{displaymath}

If $k$ is positive (negative) then [\verb+swap+ $k$] moves the first
(last) instruction $k$ positions forwards (backwards). Similarly,
[\verb+swap+ $i$ $k$] moves the $i^{th}$ instruction forwards or
backwards, and [\verb+swap+ $[i_1:i_2]$ $k$] moves the instructions
between positions $i_1$ and $i_2$.


\subsection{Reasoning about loops: the \rawec{while} tactic}
%
\subsubsection{Hoare Logic}

\Syntax

\Description


\subsubsection{Probabilistic Hoare Logic}
\index{phl}{Program Reasoning!while@\rawec{while}}

\Syntax \verb+while+ \textit{formula} \textit{formula} 
[\textit{formula} \textit{formula}]
%

\Description
%
The first formula is the loop invariant.
%
The second one is a variant expression. 
%
The third one is a real expression bound $g$ and the fourth one an
integer expression $n$.
%
If $g$ is not given then it is interpreted as $g=1$, and the fourth
formula is ignored, otherwise required. $M$ stands for the variables
that may be modified by $c$.

\begin{displaymath}
  \infrule{
    \begin{array}{c}
    \HoareGe{c'}{\pre }{\chi \land 
      \forall M.~ (\chi \land 0 \leq e \Rightarrow \neg b)  \land
      \chi \land \neg b \Rightarrow \post}{f} 
    \\[.5ex]
    \forall k.~ \HoareEq{c}{\chi \land b \land e = k}{\chi \land e
      < k}{1}
  \end{array}
}{
    \HoareGe{c';\While{b}{c}}{\pre}{\post}{f}
  }\left[\verb+while+\ \chi\ e \right] 
\end{displaymath}
Similarly for (=).

\warningbox{The following variants are not implemented}

For an arbitrary bound $g$ the following rule generalizes the one
above for lower bounded judgments:
\begin{displaymath}
  \infrule{
    \begin{array}{c}
    \HoareGe{c'}{\pre }{\chi \land e \leq n \land 
      \forall M.~ (\chi \land 0 \leq e \Rightarrow \neg b) 
      \land (\chi \land \neg b \Rightarrow \post)}{\frac{f}{g^n}} 
    \\[.5ex]
    \HoareGe{c}{\chi \land b}{\chi}{g}
    \\[.5ex]
    \forall k.~ \HoareEq{c}{\chi \land b \land e = k}{e<k}{1}
  \end{array}
}{
    \HoareGe{c';\While{b}{c}}{\pre}{\post}{f}
  }\left[\verb+while+\ \chi\ e\ g\ n \right] 
\end{displaymath}

and the folowing one for exact judgments (=):
\begin{displaymath}
  \infrule{
    \begin{array}{c}
    \HoareGe{c'}{\pre }{\chi \land e = n \land 
      \forall M.~ (\chi\Rightarrow (0\leq e \Leftrightarrow \neg b)) 
        \land (\chi \land \neg b \Rightarrow \post)}
      {\frac{f}{g^n}}
    \\[.5ex]
    \HoareGe{c}{\chi \land b}{\chi}{g}
    \\[.5ex]
    \forall k.~ \HoareEq{c}{\chi \land b \land e = k}{e=k-1}{1}
  \end{array}
}{
    \HoareGe{c';\While{b}{c}}{\pre}{\post}{f}
  }\left[\verb+while+\ \chi\ e\ g\ n \right] 
\end{displaymath}

There is no appropriate rule for $(\leq)$.


\subsubsection{Relational Hoare Logic}

\Syntax  \verb+while+ [\textit{side}] \textit{form} [\textit{form}]

\Description This tactic applies the pRHL verification rules for
loops:
\begin{itemize}
\item the optional argument \textit{side} can be either \verb+{1}+ or
  \verb+{2}+ to indicate the application of one-sided versions of the
  rule. If missing, the two-sided rule for loops is considered.
\item the first \textit{form} argument is mandatory and is used as
  loop invariant. It can refer to variables in both the left and right
  programs.
\item the optional parameter \textit{form} is required (and accepted
  only) in the one-sided application of the rule. This parameter
  corresponds to the decreasing variant expression used to prove loop
  termination.
\end{itemize}



\paragraph{Two-sided version.}
%
\Syntax \verb+while+ \textit{form} 
%
\Description Applies the two-sided RHL rule for while loops, using the
\textit{form} parameter as loop invariant. This tactic requires that
the last instruction of both left and right statements are while loops.
In the rule, $M$ refers to the variables that may be modified by the
loop bodies.

\begin{displaymath}
\infrule{ 
  \begin{array}{c}
    \Equiv{c_2}{c'_2}{I \land e\sidel \land e'\sider}{I \land  e\sidel = e'\sider}\\
    \Equiv{c_1}{c'_1}{\pre}{ I \land e\sidel = e'\sider \land 
      \forall M, (I \land \neg e\sidel \land \neg e'\sider \Rightarrow \post)}
  \end{array}
}{
  \Equiv{c_1;\While{e}{c_2}}{c'_1;\While{e'}{c'_1}}{\pre}{\post}
}
\end{displaymath}

\paragraph{One-sided version.}

\Syntax \verb+while+ \textit{side} \textit{form} \textit{form} 

\Description Applies the one-sided pRHL rule for while loops, using
the first parameter \textit{form} as loop invariant and the second
parameter \textit{form} as a decreasing \textit{variant}
expression. The variant is used to verify the loop termination. The
one-sided rule are described below. Only the left (\verb+{1}+) variant
is shown; the right (\verb+{2}+) variant is symmetric. The expressions
$\forall X,~\varphi$ and $\exists X,~\varphi$ denote, respectively,
universal and existential quantification over the set of variables $X$
modified in the loop body $c$.

\begin{displaymath}
\infrule{
  \begin{array}{c}
    \vdash I \land v \leq b \Rightarrow \neg e  \\
    \Equiv{c}{\Skip}{b=B \land v=C \land e \land I }{b=B \land v<C \land I} \\
    \Equiv{c_1}{c_2}{\pre}{I \land \forall X, (I \land \neg e
      \Rightarrow \post)}
  \end{array}
}{
  \Equiv{c_1;\While{e}{c}}{c_2}{\pre}{\post}
}
\end{displaymath}

\subsection{Reasoning on function invocation: the \rawec{call}
  tactic}

\subsubsection{Hoare Logic}

\subsubsection{Probabilistic Hoare Logic}

\subsubsection{Relational Hoare Logic}


\subsection{Reasoning with \emph{failure events}: the \rawec{fel} tactic}
%
The following rule describes the application of the tactic
$\verb+fel+\ k\ q\ c\ \delta\ F\ P$.  Assume $f$ is defined and
$c_1,c_2$ stands for the splitting of its body at position $n$. Let
$\left\{O_i\right\}_{i=0}^k$ stand for all oracles accessed by any
adversary called at $c_2$. Assume that variables in $F$ can at most be
modified by $\left\{O_i\right\}_{i=0}^k$.
 
\begin{displaymath}
\infrule{
  \begin{array}{c}
    \left\{
    \begin{array}{l}
      \HoareLe{O_i}{\neg F}{F}{c \delta} \\
      \forall c_0,\ \Hoare{O_i}{P\land c=c_0}{c_0 < c} \\
      \forall c_0,\ \forall f_0,\ \Hoare{O_i}{\neg P\land F=f_0 \land c=c_0}{F=f_0 \land c=c_0} \\
    \end{array}\right\}_{i=0}^k\\[5ex]
    \forall m', (\varphi \Rightarrow F \land c\leq q) 
    \qquad 
    q (q-1) \delta \leq \epsilon 
    \qquad
    \Hoare{c_1}{\true}{\neg F \land c=0}
  \end{array}
}{
  \Prm{f}{m}{\varphi} \leq \epsilon  
} \left[\verb+fel+\ n\ q\ c\ \delta\ F\ P\right]
\end{displaymath}

\subsection{Proving equivalences by probability computation: 
  the \rawec{bypr} tactic}
%
\subsubsection{Relational Hoare Logic}

\begin{displaymath}
\infrule{
  \forall m_1,\ \forall m_2,\ 
  \Prm{f_1}{m_1}{\varphi_1} = \Prm{f_2}{m_2}{\varphi_2}
}{
  \Equiv{f_1}{f_2}{\pre}{\post}
}
\end{displaymath}


\subsection{Loop reordering}

\Syntax 

\Description 
An invocation of 
$$\left[\verb+reordering+\ i\ (p_{w_1},p_{\mathsf{incr}_1})\
  (p_{w_2},p_{\mathsf{incr}_2})\ \mathcal{I}\ \varolessthan\ (f,f^{-1})\right]$$
%
where $i$ is the iteration expression, $p_{w_1},p_{w_2}$ indicates the
position of the while loops and
$p_{\mathsf{incr}_1},p_{\mathsf{incr}_2}$ the position of the $d$
statement increasing $i$, $b$ only depends on $i$, $c$ does not modify
$i$, $\mathcal{I}:\verb+'a+\to \verb+Bool+$, $\varolessthan:
\verb+'a+\to\verb+'a+\to\verb+Bool+$, $f:\verb+'a+\to\verb+'a+$

 implements the following rule
%
\begin{displaymath}
\infrule{
\begin{array}{c}
  \Equiv{c_1}{c_2}{\pre}{\varphi \land \left| 
      \begin{array}{l}
        (\mathcal{I}\,i \Rightarrow\forall j,~\mathcal{I}\,j
        \Rightarrow i<j) \land  
      \\
      \varphi \land
      \\
      \forall j.~\neg\mathcal{I}\,j\Rightarrow \neg\,j \land
      \\
      \forall M.~ (\varphi\Rightarrow Q) \land
    \end{array}
    \right.
  }
  \\
  \mbox{$f$ bijection on $\mathcal{I}$}
  \\
  \forall z_1\,z_2\Equiv
  {c\subst{i}{f\,z_1};c\subst{i}{f\,z_2}}
  {c\subst{i}{f\,z_2};c\subst{i}{f\,z_1}}
  {
    \begin{array}{l}
      \mathcal{I}\,z_1 \land \mathcal{I}\,z_2 \land z_1\varolessthan
      z_2 
      \\ 
      \land f\, z_2\varolessthan f\, z_1\land \varphi
    \end{array}
  }
  {\varphi}
  \\
  \Equiv{c}{c}
  {\mathcal{I}\,i \land i\sidel=i\sider \land \varphi}
  {\varphi}
\end{array}
}{
\Equiv{c_1;\While{b}{(c(i);d)}}{c_2;\While{b}{(c(f\,i);d)}}{\pre}{\post}
}
\end{displaymath}


\section{Tacticals}


\section{Automated Tactics}


%%% Local Variables: 
%%% mode: latex
%%% TeX-master: "easycrypt"
%%% End: 


\section{Miscellaneous tool directives}
\begin{itemize}
\item {\verb+include+~\textit{filename}}: Loads and processes the
  contents of the \EasyCrypt file \textit{filename}.

\item {\verb+timeout+ \textit{secs}:} Sets the current timeout given
  to SMT solvers to the value \textit{secs}. Used to increase the
  default timeout value when no SMT solver manage to prove the
  required logical goals.

\item %
  {\verb+prover+~\textit{prover${}_1$}\verb+,+..\verb+,+\textit{prover${}_k$}:}
  Sets the list of provers (separated by '\verb+,+') that are
  available to discharge the logical verification conditions. By
  default, \EasyCrypt tries with all provers recognized when invoking
  \verb|why3config --detect|. A prover name can be given either as an
  identifier or a string.

\item {\verb+check+~\textit{name}/ \verb+print+~\textit{name}} Show
  information about the object associated to the name \textit{name}. 
  

\item {\verb+checkproof+:} Enables and disables the verification of
  logical verification conditions. 

\item {\verb+set+~\textit{name}/\verb+unset+~\textit{name}}: Make the
  axiom or lemma with name \textit{name} available/unavailable as
  hypothesis for the verification of logical formulae.

\item {\verb+transparent+~\textit{name}/ \verb+opaque+~\textit{name}}:
  Set the definition of the predicate with name \textit{name} as
  transparent or opaque. If a predicate is opaque then its definition
  is not unfolded during the verification of logical formulae.


\end{itemize}

\chapter{Probability Claims and Computation}
Security properties are expressed in terms of probability of events,
rather than as pRHL judgments. Pleasingly, one can derive inequalities
(resp. equality) about probability quantities from valid judgments. In
particular, assume that the postcondition $\post$ implies $A\sidel
\Rightarrow B\sider$. Then for any programs $c_1$, $c_2$ and
precondition $\pre$ such that $\Equiv{c_1}{c_2}{\pre}{\post}$ is valid
and for any initial memories $m_1$, $m_2$ satisfying the precondition
$\pre$, we have
$$\Prm{c_1}{A}{m_1} \leq \Prm{c_2}{B}{m_2}$$
Up to now, \easycrypt assume that the two games start in the same initial
memory (i.e. $m_1 = m_2$), thus the equality of initial memories should
imply the validity of the precondition.

\section{Claims using equiv}

The natural way to obtain new claims is to deduce it from a pRHL judgment.
Assume we have proved a pRHL judgment of the form: 
\begin{verbatim}
equiv Fact1 : Game1.Main ~ Game2.Main : true ==> ={res}.
\end{verbatim} 
Then we can deduce:
\begin{verbatim}
claim c1 : Game1.Main[res] = Game2.Main[res] using Fact1.
\end{verbatim}
\easycrypt will check that the equality of the initial memories implies
the validity of the precondition (here \verb+true+) and that the
postcondition implies the logical equivalence of the two events 
(here \verb+ ={res} => (res{1} <=> res{2})+).

pRHL judgments also allow proving inequality relations between probability
expressions.  Assume we have proved a pRHL judgment of the form:
\begin{verbatim}
equiv Fact2 : Game1.Main ~ Game2.Main : 
    true ==> ={res} && (bad{1} => bad{2}).
\end{verbatim} 
Then we can deduce:
\begin{verbatim}
claim c2 : Game1.Main[res] = Game2.Main[res] using Fact2.
\end{verbatim}
but also:
\begin{verbatim}
claim c3 : Game1.Main[res && bad] <= Game2.Main[bad] using Fact2.
\end{verbatim}
For the last claim, \EasyCrypt checks that the postcondition of 
the pRHL judgment (\verb+={res} && (bad{1} => bad{2})+)
and the event associated to the first game (\verb+res{1} && bad{1}+)
imply the event associated to the second game (\verb+bad{2}+).

There is a third kind of claim which can be deduced from a pRHL judgment.
This kind of judgment is closely related to the fundamental lemma 
(also named difference lemma).
\paragraph{Fundamental lemma}{\it Let $F_1$ and $F_2$ be to distribution,
 and $A_1, A_2, B_1, B_2$ some events. Assume that
 \begin{itemize}
    \item $\Pr{F_1}{B_1} = \Pr{F_2}{B_2} $
    \item $\Pr{F_1}{A_1 \land \neg B_1} = \Pr{F_2}{A_2 \land \neg B_2}$
 \end{itemize}
then we have 
  $$ | \Pr{F_1}{B_1} - \Pr{F_2}{B_2} | \leq \Pr{F_i}{B_i}$$}

Now assume we have proved a specification of the form:
\begin{verbatim}
equiv Fact3 : Game1.Main ~ Game2.Main : 
    true ==> B1{1} <=> B2{2} && (!B1{1} => A1{1} <=> A2{2}).
\end{verbatim}
Then we can derive the following claims:
\begin{verbatim}
claim c4_1 : Game1.Main[B1] = Game2.Main[B2]
using Fact3.
claim c4_2 : Game1.Main[!B1 && A1] = Game2.Main[!B2 && A2]
using Fact3.
\end{verbatim}
So the two hypotheses of the fundamental lemma are satisfied. 
\EasyCrypt allows deriving directly the conclusion of the fundamental
lemma from \verb+Fact3+:
\begin{verbatim}
claim c4 : |Game1.Main[A1] - Game2.Main[A2] | <= Game2.Main[B2] 
using Fact3.
\end{verbatim} 
For this kind of claim, \EasyCrypt checks that the postcondition of
the pRHL judgment implies the equivalence of the bad events 
(here \verb+B1+ and \verb+B2+) in the two games. 
Furthermore if the postcondition is valid and the bad event (here \verb+B2+) 
is not set then the two events 
(here \verb+A1{1}+ and \verb+A2{2}+) should be equivalent.



\section{Claim using same and split}

There is some particular case of claim which can be deduced 
automatically without using pRHL judgments.
More precisely, the judgment $\Equiv{c}{c}{=}{=}$ is always valid 
(where $=$ means the equality of the memories).
Thus, we can derive some simple properties from it.
\begin{verbatim}
claim c_1 : G1.Main[res && (b || !b)] = G1.Main[res] 
same.
claim c_2 : G1.Main[res && b ] <= G1.Main[res]
same.
\end{verbatim}
Claim defined using \verb+same+ argument should relates the probability
of two events $A_1$ and $A_2$ in the same game.
If the comparison operator is the equality then we should have 
$A_1 \Leftrightarrow A_2$ (as in the claim \verb+c_1+).
If the comparison operator is the less or equal operator 
then we should have $A_1 \Rightarrow A_2$ (as in the claim \verb+c_2+).

Another way to simply derive claim is to use the \verb+split+ argument.
\begin{verbatim}
claim c_3 : G1.Main[res] = G1.Main[res && bad] + G1.Main[res && !bad]
split.
\end{verbatim}
If the comparison operator is the equality the claim should match the
generic form \verb?G.F[A] <= G.F[A&&B] + G.F[A&&!B]?.
If the comparison operator is the less or equal operator then
the claim should have the generic form \verb?G.F[A] <= G.F[B] + G.F[C]?.
Furthermore \easycrypt check that $A \Rightarrow (B \lor C)$.

An exemple of use of the \verb+split+ and \verb+same+ is the proof of the
fundamental lemma, assume we have proved the specification:
\begin{verbatim}
equiv Fact3 : Game1.Main ~ Game2.Main : 
    true ==> B1{1} <=> B2{2} && (!B1{1} => A1{1} <=> A2{2}).
\end{verbatim}
Then we can derive the following claims:
\begin{verbatim}
claim c4_1 : Game1.Main[B1] = Game2.Main[B2]
using Fact3.
claim c4_2 : Game1.Main[!B1 && A1] = Game2.Main[!B2 && A2]
using Fact3.
\end{verbatim}
but also:
\begin{verbatim}
claim c4_split1 : Game1.Main[A1] = Game1.Main[B1 && A1] + Game1.Main[!B1 && A1]
split.
claim c4_split2 : Game2.Main[A2] = Game2.Main[B2 && A2] + Game2.Main[!B2 && A2]
split.
claim c4_same1 : Game1.Main[B1 && A1] <= Game1.Main[A1]
same.
claim c4_same2 : Game2.Main[B2 && A2] <= Game1.Main[A2]
same. 
\end{verbatim}
Using the claims \verb+c4_1+, \verb+c4_2+, \verb+c4_split1+, \verb+c4_split2+,
\verb+c4_same1+, \verb+c4_same2+ the automatic provers (like \verb+alt-ergo+)
are able to derive the following claim:
\begin{verbatim}
claim c4 : |Game1.Main[A1] - Game2.Main[A2] | <= Game2.Main[B2].
\end{verbatim}





\section{Deducing claim from other claims}
Claim can be derived as a consequence of other claims.
When no argument is given after the statement of the claim \easycrypt
try to prove it using the previously proved claims.

Assume we have already proved the following claims:
\begin{verbatim}
claim c_1 : G1.Main[res] = G2.Main[res].
claim c_2 : | G2.Main[res] - G3.Main[res] | <= G3.Main[bad].
claim c_3 : G3.Main[res] = 1%r/2%r.
claim c_4 : G3.Main[bad] <= 1%r/(2^n)%r.
\end{verbatim}
Then the following claim is automatically deduced from the previous one:
\begin{verbatim}
claim c_5 : | G1.Main[res] - 1%r/2%r | <= 1%r/(2^n)%r.
\end{verbatim}

\section{Claims by compute}

During a reduction proof, we sometime need to compute or to bound
the probability of an event in a given game. This can be done using
the \verb+compute+ argument. Assume we have the following game:
\begin{verbatim}
game G = { 
   ...
   fun Main() : bool = {
     (pk,sk) = KG();
     (m0,m1) = A_1(pk);
     c       = {0,1}^k;
     b'      = A_2(c);
     b       = {0,1};
     return b = b';
  }
} 
\end{verbatim}
Then \easycrypt is able to compute the probability of \verb+res=true+
in the function \verb+G.Main+:
\begin{verbatim}
claim c : G.Main[res] = 1%r/2%r 
compute.
\end{verbatim} 

The \verb+compute+ argument is also able to prove the claim that can be
derive using \verb+split+ and \verb+same+, but it is less efficient.
On the other side it is also more powerful, for example we can prove:
\begin{verbatim}
claim c : G.Main[A || B || C] <= G.Main[A] + G.Main[B] + G.Main[C]
compute.  
\end{verbatim}
This claim can also be obtained using the \verb+split+ argument, using
the following sequence:
\begin{verbatim}
claim c_1 :  G.Main[A || B || C] <= G.Main[A || B] + G.Main[C]
split.
claim c_2 :  G.Main[A || B] <= G.Main[A] + G.Main[B]
split.
claim c : G.Main[A || B || C] <= G.Main[A] + G.Main[B] + G.Main[C]. 
\end{verbatim}
The claim \verb+c+ is a direct consequence of the claims \verb+c_1+ and 
\verb+c_2+.

A last example of use for \verb+compute+ is the following, assume
we have a game of the form:
\begin{verbatim}
game G = { 
   ...
   fun Main () : bool = {
     x = init();
     d = A(x);
     z = {0,1}^k;
     return d;
   }
}     
\end{verbatim}
Then \verb+compute+ is able to prove the following claim:
\begin{verbatim}
claim c :
  G.Main[res && mem(z,L) && length(L) <= q] <= q%r/(2^k)%r * G.Main[res]
compute.
\end{verbatim}

Sometime the event we want to bound is not set in the main function
but in an oracle, furthermore we known that the oracle can be call at
most $q$ time. Assume that the probability that the event is set during
one call to the oracle is bounded by $u$, we would like to conclude 
that the probability event is set in the main function is bounded by
$u*q$. This is possible using the failure event lemma.
\begin{verbatim}
game G = {
  var C: int
  var bad : bool
  fun O(x:int) : bitstring{k} = {
    var r = {0,1}^k;
    C = C + 1;
    if (r = 0) bad = true;
    return r;
  }
  abs A = A {O}
  fun Main() : bool = {
    var d : bool;
    C = 0;
    bad = false;
    d = A();
    return d;
  }    
\end{verbatim}
In the example the probability that \verb+bad+ is set in during a
call to the oracle \verb+O+ is $1/2^k$, furthermore the counter \verb+C+
count the number of call to \verb+O+.
We can use the following to bound the probability of \verb+bad+ in the main:
\begin{verbatim} 
claim pr_bad : G.Main[bad && C <= q] <= q%r * (1%r/(2^k)%r)
compute 2 (bad), (C).
\end{verbatim}

The second argument indicate the bad event (of the oracle) 
we consider and the third should be a expression representing the counter. 
The first argument is an integer indicating the number of instructions
in the main needed to initialize the failure event lemma. 
After those instructions the value associated to the counter should be 0
and the bad event should evaluate to false.
Then \easycrypt should be able to prove that the probability that the bad
event is set during an oracle call is bounded by $1/2^k$. Furthermore,
if the bad event is set during a call to the oracle then the counter
increase, and do not decrease in the other case, and that the bad event
is never reset.

\section{Claims by admit}
The last possibility to define a claim is to use the admit argument.
\begin{verbatim}
claim c : G.Main[res] = G'.Main[res] 
admit.
\end{verbatim}
It that case the validity of the claim is admitted without any check.

\section{Claims by auto}

It is also possible to directly define claim which normally should be 
defined using an \textit{equiv} specification directly:
\begin{verbatim}
claim c12 : G1.Main[res] = G2.Main[res]
auto.
\end{verbatim}
This is a shortcut for: 
\begin{verbatim}
equiv c12_aux : G1.Main ~ G2.Main : true ==> ={res}
by auto.
claim c12 : G1.Main[res] = G2.Main[res]
using c12_aux.
\end{verbatim}

%%% Local Variables: 
%%% mode: latex
%%% TeX-master: "easycrypt"
%%% End: 


\chapter{Example: elgamal}
\begin{flushright}
\it (The syntax used in this section may be outdated.)
\end{flushright}

We illustrate the key ingredients presented in prevous chapters with a
simple example: a game-based proof of the \INDCPA-security of the
ElGamal public-key encryption scheme.

The ElGamal encryption scheme is based on any cyclic group $G$ of
order $q$ with generator $g$ and is defined by the following triple of
algorithms

\begin{itemize} 
\item The key generation algorithm $\KG()$ selects uniformly a random
      number $x$ from $\{0,\ldots,q-1\}$; the secret (private) key is
      $x$, the public key is $g^x$.

\item Given a public key $pk$ and a plaintext $m$ (an element of the
      group $G$), the encryption algorithm $\Enc(pk, m)$ chooses
      uniformly a random element $y$ from $\{0,\ldots,q-1\}$ and
      returns the ciphertext $(g^y, pk^y * m)$.

\item Given a secret key $sk$ and a ciphertext $c$, 
      the decryption algorithm $\Dec(sk, c)$, parses $c$ as
      $(\beta,\zeta)$ and returns a plaintext computed as
      $\zeta * \beta^{-x}$.
\end{itemize}

We start by declaring a type for elements of the group $G$, and
defining type synonyms for the type of public and secret keys,
plaintexts and ciphertexts:
%
\begin{verbatim} 
type group 
type skey = int 
type pkey = group 
type plaintext = group 
type ciphertext = group * group 
\end{verbatim} 
%
The order of the group $q$ and its generator $g$ are declared as
constants:
%

\begin{verbatim}
cnst q : int
cnst g : group
\end{verbatim}

%
We then declare operators that will denote the group law in $G$,
exponentiation and discrete logarithm (in base $g$).
%
\begin{verbatim}
op (*) : group, group -> group = group_mult
op (^) : group, int -> group   = group_pow
op log : group-> int           = group_log
\end{verbatim}
%
% The first two operators are declared as infix, and denoted by the
% symbols \verb|*| and \verb|^|, respectively. The operator
% corresponding to the discrete logarithm is a normal prefix
% operator. The names appearing on the right of the declaration are
% identifiers that will be used as internal names for the operators when
% generating proof obligations that are sent to SMT solvers (this is
% needed because fancy identifiers like \verb|*| are not valid
% identifiers, and useful to avoid name clashes with predefined
% operators).

At this point the operators and constants that we declared above are
completely abstract, nothing is known about them besides their
type. To specify

At that point nothing say that the type \verb+group+ is a cyclic
group, we only known that the type come with three
operators \verb+*+, \verb+^+ and \verb+log+. We should specify the
behavior of the operators this is done using axioms:

\begin{verbatim}
axiom q_pos : {0 < q}

axiom group_pow_add : 
 forall (x:int, y:int). { g ^ (x + y) == g ^ x * g ^ y }

axiom group_pow_mult :
 forall (x:int, y:int). { (g ^ x) ^ y == g ^ (x * y) } 

axiom log_pow : 
 forall (g':group). { g ^ log(g') == g' }

axiom pow_mod : 
 forall (z:int). { g ^ (z%q) == g ^ z }
\end{verbatim}
      The first axiom \verb+q_pos+ expresses that the integer \verb+q+ 
      representing the order of the group is positive. 
      The next \verb+group_pow_add+ and \verb+group_pow_mult+ specify the 
      behavior of the multiplication and the exponentiation,  
      \verb+log_pow+ partially specify the behavior of the logarithm operator.
      The \verb|+| operator used in \verb+group_pow_add+ is the predefined 
      additive operator over integer. Note that the \verb+*+ operator in 
      the axiom \verb+group_pow_mult+ represent the multiplication over 
      integer and not the multiplication law of the group 
      (\easycrypt{} allows to overloading of operator). 
      The last axiom expresses the fact that the group is a cyclic group of 
      order \verb+q+, \verb+%+ stand for the modulus operator over integer.

      To be able to perform the proof we also add axioms on the modulus operator:
\begin{verbatim}
axiom mod_add : 
 forall (x:int, y:int). { (x%q + y)%q == (x + y)%q }

axiom mod_small : 
 forall (x:int). { 0 <= x } => { x < q } => { x%q == x}

axiom mod_sub : 
 forall (x:int, y:int). { (x%q - y)%q == (x - y)%q } 
\end{verbatim}

The IND-CPA semantic security is expressed as a game parameterized by an pair 
of adversaries, let us declare this two adversaries:
\begin{verbatim}
adversary A1(pk:pkey)               : plaintext * plaintext {}
adversary A2(pk:pkey, c:ciphertext) : bool {}
\end{verbatim}
The first one \verb+A1+ expect a public key \verb+pk+ and return a pair 
of plaintext, the second one expect a public key and a cyphertext and return
a boolean. The semi-bracket contains the declaration of the oracles that can
be used by the adversaries, here there is no oracles.

We can now define the game representing the IND-CPA semantic security of ElGamal:
\begin{verbatim}
game INDCPA = {
  fun KG() : keys = {
    var x : int = [0..q-1];
    return (x, g^x);
  }

  fun Enc(pk:pkey, m:plaintext): ciphertext = {
    var y : int = [0..q-1];
    return (g^y, (pk^y) * m);
  }

  abs A1 = A1 {}
  abs A2 = A2 {}
  
  fun Main() : bool = {
    var sk : skey;
    var pk : pkey;
    var m0, m1, mb : plaintext;
    var c: ciphertext;
    var b, b' : bool;

    (sk,pk) = KG();
    (m0,m1) = A1(pk);
    b = {0,1};
    mb = b ? m0 : m1;
    c = Enc(pk, mb);
    b' = A2(pk, c);
    return (b == b');
  } 
}      
\end{verbatim}
The game start by the declaration of two functions the key generation
algorithm \verb+KG+ and the encryption algorithm \verb+Enc+. Then come
the definition of the two adversary \verb+A1+ and \verb+A2+, they are
defined to be equal to the abstract functions previously defined.
The main function, at the end of the game, represent the IND-CPA experiment.
First the key generation algorithm is used to generate the secret and public
keys, then the public key is given to \verb+A1+ which generate two plaintext
\verb+m0+ and \verb+m1+. The instruction \verb+b = {0,1}+ uniformly sample a
boolean which is stored in \verb+b+. Depending on this bit \verb+b+
either the plaintext \verb+m0+ or \verb+m1+ is encrypted with the
public key \verb+pk+, generating the ciphertext \verb+c+. The public key and 
ciphertext are then give back to the adversary \verb+A2+. The goal of the
adversary is to discover which plaintext as been encrypted. It win if
\verb+b+ is equal to \verb+b'+.

The IND-CPA semantic security of ElGamal express that there exists a 
adversary \verb+B+ build on top of \verb+A1+ and \verb+A2+ which as a higher 
probability of breaking the Decisional Diffie Hellman problem (DDH) than
\verb+A1+ and \verb+A2+ of winning the IND-CPA game. 
% The DDH hypothesis 
% say that it is hard to distinguish ....\todo{finish this}.
The first thing to do is to define the two games and the 
adversary \verb+B+ involved in DDH problem:
\begin{verbatim}

game DDH0 = {
  abs A1 = A1 {}
  abs A2 = A2 {}
  
  fun B(gx:group, gy:group, gz:group) : bool = {
    var m0, m1, mb : plaintext;
    var c : ciphertext;
    var b, b' : bool;
 
    (m0, m1) = A1(gx);
    b = {0,1};
    mb = b ? m0 : m1;
    c = (gy, gz * mb);
    b' = A2(gx,c);
    return (b == b');
  }

  fun Main() : bool = {
    var x, y : int;
    var d : bool;

    x = [0..q-1];
    y = [0..q-1];
    d = B(g^x, g^y, g^(x*y));
    return d;
  }     
}

game DDH1 = DDH0 where 
  Main = {
    var x, y, z : int;
    var d : bool;

    x = [0..q-1];
    y = [0..q-1];
    z = [0..q-1];
    d = B(g^x, g^y, g^z);
    return d;
  } 

\end{verbatim}
The main experiment in the game \verb+DDH0+ start by uniformly sample
two values \verb+x+ and \verb+y+ between 0 and $q-1$ and then 
send $g^x, g^y, g^{xy}$ to the adversary \verb+B+. The game \verb+DDH1+
is defined to be equal to the game \verb+DDH0+ where only the main function
changes: a new variable \verb+z+ is uniformly sample and $g^z$ is send
to the adversary instead of $g^{xy}$. The goal of the adversary is to discover
if its last argument correspond to $g^{xy}$ or $g^z$, i.e. if it play
between \verb+DDH0+ or \verb+DDH1+.

We can know start our proof:
\begin{verbatim}
prover alt-ergo

equiv auto Fact1 : INDCPA.Main ~ DDH0.Main : {true} ==> ={res};;

claim Pr1 : INDCPA.Main[res] == DDH0.Main[res] 
using Fact1;;
\end{verbatim}
The first line select the prover to be used, here \verb+alt-ergo+ (the
default one is \verb+simplify+). The second line is the main component of
\easycrypt. We demonstrate using the probabilistic Relational Hoare Logic (pRHL)
that the two functions \verb+INDCPA.Main+ and \verb+DDH0.Main+ are 
indistinguishable if we observe only their results.
This allows to proving the claim \verb+Pr1+ which state that the probability
that \verb+res+ is true after running the two programs is equal.
% \todo{rewrite this ...}

\begin{verbatim}
game G1 = INDCPA where 
  Main = {
    var x, y, z : int;
    var gx, gy, gz : group;
    var d, b, b' : bool;
    var m0, m1, mb : plaintext;
    var c : ciphertext;
 
    x = [0..(q - 1)];
    y = [0..(q - 1)];
    gx = g^x;
    gy = g^y;
    (m0, m1) = A1 (gx);
    b = {0,1};
    mb = b ? m0 : m1; 
    z = [0..(q - 1)];
    gz = g^z;
    c = (gy, gz * mb);
    b' = A2 (gx, c);
    d = (b == b');
    return d;
  }

equiv auto Fact2 : G1.Main ~ DDH1.Main : {true} ==> ={res};;
 
claim Pr2 : G1.Main[res] == DDH1.Main[res] 
using Fact2;;
\end{verbatim}

\begin{verbatim}
game G2 = G1 where 
  Main = {
    var x, y, z : int;
    var gx, gy, gz : group;
    var d, b, b' : bool;
    var m0, m1, mb : plaintext;
    var c : ciphertext;
 
    x = [0..(q - 1)];
    y = [0..(q - 1)];
    gx = g^x;
    gy = g^y;
    (m0, m1) = A1(gx);
    z = [0..(q - 1)];
    gz = g^z;
    c = (gy, gz); 
    b' = A2 (gx, c);
    b = {0,1};
    d = (b == b');
    return d;
  }

equiv Fact3 : G1.Main ~ G2.Main : {true} ==> ={res} 
 swap{2} [10-10] -4; auto;
 rnd (z + log(b?m0:m1)) % q, (z - log(b?m0:m1)) % q; wp; rnd; 
 auto; repeat rnd;
 trivial;;
save;;

claim Pr3 : G1.Main[res] == G2.Main[res]
using Fact3;;
\end{verbatim}

\begin{verbatim}
claim Pr4 : G2.Main[res] == 1%r / 2%r
compute;;

claim Conclusion : 
 | INDCPA.Main[res] - 1%r / 2%r | <= | DDH0.Main[res] - DDH1.Main[res] | 
\end{verbatim}

%%% Local Variables: 
%%% mode: latex
%%% TeX-master: "easycrypt"
%%% End: 


%%% Local Variables: 
%%% mode: latex
%%% TeX-master: "easycrypt"
%%% End: 
