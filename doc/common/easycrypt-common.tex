% --------------------------------------------------------------------
\usepackage{amsmath}

\newcommand{\rel}[1]{\mathrel{#1}}

% --------------------------------------------------------------------
\newenvironment{tightcenter}{%
  \setlength\topsep{0pt}
  \setlength\parskip{0pt}
  \begin{center}}
{\end{center}}

% --------------------------------------------------------------------
% Acronyms, names, ...

\usepackage{xspace}

\def\EasyCrypt{\textsc{EasyCrypt}\xspace}
\def\prhl{\textsc{prhl}\xspace}
\def\phl{\textsc{phl}\xspace}
\def\hl{\textsc{hl}\xspace}

% --------------------------------------------------------------------
% Inference rules
\usepackage{mathpartir}

\newenvironment{cmathpar}
{\begin{tightcenter}\begin{mathpar}}
{\end{mathpar}\end{tightcenter}}

% --------------------------------------------------------------------
% EasyCrypt listings
\usepackage[final]{listings}

\newcommand{\ensuretext}[1]{\ensuremath{\text{#1}}}

\documentclass[a4paper,notitlepage]{book}

\makeindex

\usepackage{xspace}
\usepackage{makeidx}
\usepackage{mdframed}
\usepackage[procnames]{listings}
\usepackage{infer}
\usepackage[pagebackref,colorlinks=true,linkcolor=black,linktoc=all,citecolor=blue]{hyperref}
\usepackage[usenames,dvipsnames]{xcolor}

% !TeX root = easycrypt.tex
%% Names
% Tools
 \newcommand{\EasyCrypt}{\textsf{EasyCrypt}\xspace}
 \newcommand{\CertiCrypt}{\textsf{CertiCrypt}\xspace}
 \newcommand{\CertiPriv}{\textsf{CertiPriv}\xspace}
 \newcommand{\SsReflect}{\textsf{SsReflect}\xspace}
 \newcommand{\Coq}{\textsf{Coq}\xspace}
 \newcommand{\WhyThree}{\textsf{Why3}\xspace}

% Languages and Logics
 \newcommand{\pWHILE}{\textsf{p}\textsc{While}\xspace}
 \newcommand{\pRHL}{\textsf{pRHL}\xspace}

% Version numbers
 \newcommand{\ECversion}{0.$\beta$\xspace}

%% Misc
\newcommand{\DONE}{}% {{\color{red}DONE}}
\newcommand{\Example}{\paragraph*{Example}}
\newcommand{\Syntax}{\paragraph*{Syntax}}
\newcommand{\Description}{\paragraph*{Description}}
\setcounter{secnumdepth}{3}
\renewcommand{\thesubsubsection}{\arabic{chapter}.\arabic{section}.\arabic{subsection}.\arabic{subsubsection}}
\newbox\minicodebox
\newenvironment{minicode}[1]{%
\minipage[t]{#1\linewidth} %
\centering %
\verbatim %
}{%
\endverbatim %
\endminipage% 
}


\newcounter{alarmcounter}
\setcounter{alarmcounter}{1}
\newcommand{\alarm}[1]
            {\begingroup
              \def\thefootnote{{\normalsize\color{red}(\arabic{footnote})}}
              \footnote{\textsf{\textbf{{\color{red}\sc \bf ALARM:}
                  #1}}}\endgroup}


\newcommand{\infr}[2]{
{\renewcommand{\arraystretch}{1.1}
\begin{array}{c}
{#1}\\
\hline
{#2}
\end{array}}}

% \providecommand{\eqref}[1]{\textup{(\ref{#1})}}
% \providecommand{\eqdef}{\raisebox{-.2ex}[.2ex]{$\stackrel{\textrm{\tiny def}}{~=~}$}}
\newcommand{\rname}[1]{[\textsc{#1}]}
\newcommand{\result}{\mathsf{res}}

%% Security properties and schemes

\newcommand{\INDCPA}{\textsf{IND-CPA}\xspace}
\newcommand{\INDCCAone}{\textsf{IND-CCA1}\xspace}
\newcommand{\INDCCA}{\textsf{IND-CCA}\xspace}
\newcommand{\EFCMA}{\textsf{EF-CMA}\xspace}
\newcommand{\LCDH}{\ensuremath{\mathsf{LCDH}}\xspace}
\newcommand{\CDH}{\textsf{CDH}\xspace}
\newcommand{\DDH}{\textsf{DDH}\xspace}
\newcommand{\ElGamal}{\textsf{ElGamal}\xspace}
\newcommand{\HElGamal}{\textsf{HElGamal}\xspace}
\newcommand{\RSA}{\textsf{RSA}\xspace}
\newcommand{\OAEP}{\textsf{OAEP}\xspace}
\newcommand{\FDH}{\textsf{FDH}\xspace}
\newcommand{\HMAC}{\textsf{HMAC}\xspace}
\newcommand{\CS}{\textsf{CS}\xspace}
\newcommand{\Skein}{\textsf{Skein}\xspace}
\newcommand{\TCR}{\textsf{TCR}\xspace}

\newcommand{\KG}{\mathcal{KG}}
\newcommand{\Enc}{\mathcal{E}}
\newcommand{\Dec}{\mathcal{D}}

%% Complexity and termination

\newcommand{\lossless}{\mathsf{lossless}}

%% Sets

\newcommand{\zeroone}{[0,1]}
\newcommand{\bit}{\{0,1\}}
\newcommand{\bitstring}[1]{\ensuremath{\bit^{#1}}}
\newcommand{\bool}{\mathbb{B}}
\newcommand{\nat}{\mathbb{N}}
\newcommand{\real}{\mathbb{R}}
\newcommand{\option}[1]{#1_\bot}

%% Mathematics

\renewcommand{\Pr}[2]{\mathrm{Pr}\left[#1 : #2\right]}
\newcommand{\Prm}[3]{\mathrm{Pr}\left[#1,#3 : #2\right]}
\newcommand{\labs}{\left\lvert}
\newcommand{\rabs}{\right\rvert}
\newcommand{\charfun}{\mathds{1}}

%% Distribution monad

\newcommand{\supp}{\mathsf{support}}
\newcommand{\range}[2]{\mathsf{range}~{#1}~{#2}}

%% Semantics 

% \newcommand{\sem}[1]{\llbracket #1 \rrbracket}
\newcommand{\subst}[2]{\left[{}^{#2}/{}_{#1}\right]}
\newcommand{\fv}{\mathsf{fv}}
\newcommand{\modifies}{\mathsf{mod}}
\newcommand{\glob}{\mathsf{glob}}
\newcommand{\Env}{\mathcal{E}}

%% Well-formed adversaries
%% Program Judgements 
\newcommand{\Hoare}[3]{\left[{#1}:{#2}\Longrightarrow{#3}\right]}
\newcommand{\Equiv}[4]{\left[{#1}\sim{#2}:{#3}\Longrightarrow{#4}\right]}
\newcommand{\bdHoareS}[5]{{\Hoare{#1}{#2}{#3}}\,{#4}\,{#5}}
\newcommand{\HoareLe}[4]{\bdHoareS{#1}{#2}{#3}{\leq}{#4}}
\newcommand{\HoareEq}[4]{\bdHoareS{#1}{#2}{#3}{=}{#4}}
\newcommand{\HoareGe}[4]{\bdHoareS{#1}{#2}{#3}{\geq}{#4}}


% \newcommand{\Equiv}[4]{\models {#1} \sim {#2} : {#3} \Longrightarrow {#4}}
% \newcommand{\AEquiv}[6]{\models {#2} \sim_{#5,#6} {#3} : {#1} \Longrightarrow {#4}}
% \newcommand{\JAEquiv}[6]{{#2} \sim_{#5,#6} {#3} : {#1} \Longrightarrow {#4}}
% \newcommand{\EquivMem}[2]{\models {#1} \equiv {#2}}
% \newcommand{\EqObs}[4]{\models {#1} \simeq^{#3}_{#4} {#2}}
% \newcommand{\AEqObs}[5]{\models {#1} \simeq^{#3}_{{#4}} {#2} \preceq {#5}} 
% \newcommand{\ACEqObs}[7]
%         {\AEqObs{\left[ #1 \right]_{#6}}{\left[ #2 \right]_{#7}}{#3}{#4}{#5}}
% \newcommand{\Triple}[3]{\sem{#2} {#3} \preceq {#1}}
% \newcommand{\DTriple}[3]{\sem{#2} {#3} \succeq {#1}}
% \newcommand{\dequiv}[3]{{#1} \simeq_{#3} {#2}}
% \newcommand{\fequiv}[3]{{#1} =_{#3} {#2}}
% \newcommand{\Pre}{\Psi}
% \newcommand{\Post}{\Phi}
% \newcommand{\Inv}{\Phi}
\newcommand{\side}[1]{\langle #1 \rangle}
\newcommand{\sidel}{\side{1}}
\newcommand{\sider}{\side{2}}
% \newcommand{\eqobsin}{\mathsf{eqobs\_in}}
% \newcommand{\eqobsout}{\mathsf{eqobs\_out}}
\newcommand{\pre}{\Psi}
\newcommand{\post}{\Phi}

%% Variables

% \newcommand{\LH}{\gl{L}_H}
% \newcommand{\LD}{\gl{L}_\Dec}
% \newcommand{\cdef}{\gl{\gamma_\mathsf{def}}}
\newcommand{\var}[1]{\ensuremath{\mathit{#1}} \xspace}

%% Constants and operators
% TODO: Update!
\newcommand{\true}{\mathsf{true}}
\newcommand{\false}{\mathsf{false}}
\newcommand{\nil}{\mathsf{nil}}
\newcommand{\hd}{\mathsf{hd}}
\newcommand{\tl}{\mathsf{tl}}
\newcommand{\app}{\mathbin{+\mkern-7mu+}}
\newcommand{\concat}{\parallel}
\newcommand{\xor}{\oplus}
\newcommand{\msb}[2]{[#1]^{#2}}
\newcommand{\lsb}[2]{[#1]_{#2}}
\newcommand{\dom}{\mathsf{dom}}
\newcommand{\ran}{\mathsf{ran}}
\newcommand{\fst}{\mathsf{fst}}
\newcommand{\snd}{\mathsf{snd}}
\newcommand{\some}[1]{#1}
\newcommand{\none}{\bot}

\newcommand{\Int}{\mathsf{Int}}
\newcommand{\tint}{\mathsf{int}}

\newcommand{\tbool}{\mathsf{bool}}

%% Language
% TODO: Update!
\newcommand{\Skip}{\mathsf{skip}}
\newcommand{\Seq}[2]{#1;\ #2}
\newcommand{\Ass}[2]{#1 \leftarrow #2}
\newcommand{\Rand}[2]{#1 \stackrel{\raisebox{-.25ex}[.25ex]%
{\tiny $\mathdollar$}}{\raisebox{-.2ex}[.2ex]{$\leftarrow$}} #2}
\newcommand{\Randi}[2]{\Rand{#1}{[0..#2]}}
\newcommand{\Randb}[1]{\Rand{#1}{\bit}}
\newcommand{\Randbs}[2]{\Rand{#1}{\bitstring{#2}}}
\newcommand{\Cond}[3]{\mathsf{if}\ #1\ \mathsf{then}\ #2\ \mathsf{else}\ #3}
\newcommand{\Condt}[2]{\mathsf{if}\ #1\ \mathsf{then}\ #2}
\newcommand{\Else}{\mathsf{else}\ }
\newcommand{\Elsif}{\mathsf{elsif}\ }
\newcommand{\nWhile}[3]{\mathsf{while}_{#1}\ #2\ \mathsf{do}\ #3}
\newcommand{\While}[2]{\mathsf{while}\ #1\ \mathsf{do}\ #2}
\newcommand{\Call}[3]{#1 \leftarrow #2\mathsf{(}#3\mathsf{)}}
\newcommand{\Return}{\mathsf{return}}
% \newcommand{\Assert}[1]{\mathsf{assert}~#1}

%% Language definition
\lstnewenvironment{easycrypt}[2][]%
  {\lstset{language=easycrypt,caption=#2,#1}}%
  {}

\newcommand{\rawec}[2][]{\lstinline[language=easycrypt,#1]{#2}}
\newcommand{\ec}[2][]{\lstinline[language=easycrypt,style=easycrypt-pretty,#1]{#2}}

\newcommand{\ecimport}[4][]{\lstinputlisting[language=easycrypt,linerange=#4,caption=#2,#1]{#3}}

\def\createEasycrypt#1\relax{
\lstdefinelanguage{easycrypt}{
  style=easycrypt-default,
%  procnamekeys={op,pred,fun},
%  procnamestyle={\sffamily\itshape},
  keywordsprefix={'},
  morekeywords=[1]{unit,bool,int,real,bitstring,array,list,matrix,word},
  morekeywords=[2]{type,op,axiom,lemma,module,pred,const,declare},
  morekeywords=[3]{var,fun},
  morekeywords=[4]{while,if},
  morekeywords=[5]{theory,end,clone,import,export,as,with,section},
  morekeywords=[6]{forall,exists,lambda},
  morekeywords=[7]{#1},
%  moredirectives={prover,print}, % Incomplete
  morecomment=[n][\itshape]{(*}{*)},
  morecomment=[n][\bfseries]{(**}{*)}
}
}

% !TeX root = easycrypt.tex

%%%%%%%%%%%%%%%%%%%%%%%%%%%%%%%%%
% DEFS
%%%%%%%%%%%%%%%%%%%%%%%%%%%%%%%%%

\newcommand{\ambientKeywords}{}

\newcommand{\tacname}{Error tacname}
\newcommand{\vtacname}{Error tacname}

\newcommand{\addTacticIdx}[1]{%
\expandafter\def\expandafter\ambientKeywords\expandafter{\ambientKeywords,#1}}

\newcommand{\addTacticNoIdx}[2]{
  \renewcommand{\tacname}{\rawec{#1}}
  \renewcommand{\vtacname}{#1}
  \index{ambient}{#1@\rawec{#1}}
  \subsubsection{#1}
  \Syntax \ec{#1} #2
  \Description}

\newcommand{\addTactic}[2]{
  \addTacticIdx{#1}
  \addTacticNoIdx{#1}{#2}}

\newcommand{\example}[6]%proof,context,goal
{
\vspace*{3ex}
\begin{tabular}{ccc}
\parbox{100pt}{#1} & {\expandafter\rawec\expandafter{#3 #4.}} & \parbox{100pt}{#5} \\
\cline{0-0} \cline{3-3} {\ec{#2}} & ~ & {\ec{#6}} \\
\end{tabular}\\
}

\newcommand{\env}[2]{\ec{#1 : #2}\\}

\newcommand{\vararg}[1]{\ec{#1}}
\newcommand{\cstarg}[1]{\ec{#1}}
\newcommand{\typarg}[1]{\textit{#1}}

\newcommand{\tacarg}[2]{(\vararg{#1}:\typarg{#2})}

\newcommand{\refdef}[1]{\emph{#1}(\ref{#1})}


%%%%%%%%%%%%%%%%%%%%%%%%%%%%%%%%%
% END DEFS
%%%%%%%%%%%%%%%%%%%%%%%%%%%%%%%%%

\subsection{Generalities}

\EasyCrypt ambient logic is based on non-dependent higher-order logic.

\subsection{Convertibility}\label{convertible}

\EasyCrypt ambient logic enjoys a mechanism that \emph{identifies} all formulas
that are equal up to a given amount of computations. The computational power
of \EasyCrypt if defined as the closure by equivalence of the following 
rewriting system:

\begin{center}
\begin{tabular}{l@{$\quad$}l@{$\quad$}ll}
{\rawec{(lambda (x : t), phi1)\ phi2}} & $\rightarrow_\beta$ &
  \multicolumn{2}{@{}l}{{\rawec{phi2} \{\rawec{x} $\leftarrow$ \rawec{phi1}\}}}\\
{\rawec{if (true) \{ phi1 \} else \{ phi2 \}}} & $\rightarrow_\iota$ &
  {\rawec{phi1}}\\
{\rawec{if (false) \{ phi1 \} else \{ phi2 \}}} & $\rightarrow_\iota$ &
  \multicolumn{2}{@{}l}{{\rawec{phi2}}}\\
{\rawec{let (x1, ..., xn) = (phi1, ..., phin) in phi}} & $\rightarrow_\iota$ &
  \multicolumn{2}{@{}l}{{\rawec{phi} \{ \rawec{x1, ..., xn} $\leftarrow$ \rawec{phi1, ..., phin} \}}}\\
{\rawec{let x = phi1 in phi2}} & $\rightarrow_\zeta$ &
  \multicolumn{2}{@{}l}{{\rawec{phi2} \{ \rawec{x} $\leftarrow$ \rawec{phi1} \}}}\\
{\rawec{o}} & $\rightarrow_\delta^{\Env,\Gamma}$ &
  {\rawec{e}} & if {\rawec{op o := e}} $\in \Env$\\
{\rawec{x}} & $\rightarrow_\delta^{\Env,\Gamma}$ &
  {\rawec{phi}} & if {\rawec{x := phi}} $\in \Gamma$\\
\end{tabular}
\end{center}

\noindent augmented with a set of logical simplification rules denoted by
$\rightarrow_\Lambda$. We write $\phi_1 \rightarrow_\delta \phi_2$ for
$\phi_1 \rightarrow_\delta^{\Env;\Gamma}$ if $\Env; \Gamma$ can be deduced
form the context. We write $\rightarrow_{\Env;\Gamma}$ for the union of all
the $\beta\iota\delta\zeta\Lambda$-rewrite rules. As usual,
$\leftrightarrow^*_{\Env;\Gamma}$ denotes the closure by equivalence of
$\rightarrow_{\Env;\Gamma}$.

%change
\addTactic{change}{\tacarg{f}{formula}}
Change the current goal to the $\leftrightarrow^*$-equivalent one \ec{f}
\begin{displaymath}
  \infrule{\phi_1 \leftrightarrow^*_{\Env;\Gamma} \phi_2 \quad
           \Env; \Gamma \vdash \phi_1}
          {\Env; \Gamma \vdash \phi_2}
\end{displaymath}

%simplify
\addTactic{simplify}{\tacarg{names}{ident*} | \ec{delta}?}
 \addTacticIdx{beta}
 \addTacticIdx{iota}
 \addTacticIdx{zeta}
 \addTacticIdx{logic}
 Change the goal with its $\beta\iota\zeta\Lambda$-head normal-form, followed
 by one step of parallel, strong $\delta$-reduction if \ec{delta} is given.
 The $\delta$-reduction can be restricted to a set of defined symbols by
 replacing \ec{delta} by the non-empty sequence of targeted symbols. You can
 selectively change the goal with its $\beta$-head normal form
 (resp. $\iota$, $\zeta$, $\Lambda$-head normal form) by using the tactic
 \ec{beta} (resp. \ec{iota}, \ec{zeta}, \ec{logic}).

%delta
\addTactic{delta}{\tacarg{names}{ident*}}
Do one step of parallel, strong $\delta$-reduction, restricted to
 the symbols designed by \ec{names}. If \ec{names} if empty, no restriction
 on the $\delta$-reduction is applied.

\subsection{Bookkeeping}

%generalize
\addTactic{generalize}{\tacarg{p}{pattern}}
Search for the first subterm of the goal matching \ec{p} and leading
to the full instantiation of the pattern. Then, do a logical
generalization of all the occurrences of \ec{p}, after instantiation,
in the goal.
\begin{displaymath}
  \infrule{\Env; \Gamma \vdash p \quad
           \Env; \Gamma \vdash \forall x, \phi(x)}
          {\Env; \Gamma \vdash \phi(p)}
\end{displaymath}

%pose
\addTactic{pose}{\tacarg{x}{ident} \rawec{:=} \tacarg{p}{pattern}}
Search for the first subterm of the goal matching \ec{p} and leading
to the full instantiation of the pattern. Then, introduce, after
instantiation, the local definition \rawec{x := p} and abstract
all the occurrences of \ec{p} in the goal by \ec{x}
\begin{displaymath}
  \infrule{\Env; \Gamma \vdash p \quad
           \Env; \Gamma, x := p \vdash \phi(x)}
          {\Env; \Gamma \vdash \phi(p)}
\end{displaymath}

%intros
\addTactic{intros}{\tacarg{x}{\_|ident}}
This tactics permits to remove of your goal : a forall, the left side af an application or a let assignement by pushing it into your \refdef{context}.
Easycrypt checks that \vararg{x} is not already present in the \refdef{environment}.
\begin{displaymath}
  \infrule{\Gamma,x = a \vdash G(x)}{\Gamma \vdash let x = a in G(x)}
  ~~~~~~
  \infrule{\Gamma,x \vdash G(x)}{\Gamma \vdash \forall x, G(x)}
  ~~~~~~
  \infrule{\Gamma,H \vdash G}{\Gamma \vdash H => G}
  ~~~~~~
\end{displaymath}

\example
{}{forall (x y:int), x = 3 => x = 3}
{\vtacname}{a b hyp1}
{
\env{a}{int}
\env{b}{int}
\env{h1}{a=3}
}
{b = 3}

%%%%%%%%%%%%%%%%%%%%%%%%%%%%%%%%%
% LOGIC
%%%%%%%%%%%%%%%%%%%%%%%%%%%%%%%%%

\subsection{Logic}

%assumption
\addTactic{assumption}{}
Search in the context an hypothesis \refdef{convertible} to the goal and close it.
 If no such hypothesis exists, the tactic fails
\begin{displaymath}
  \infrule{(h : \phi) \in \Gamma}{\Env; \Gamma \vdash \phi}
\end{displaymath}

%reflexivity
\addTactic{reflexivity}{}
Solve goals of the form \ec{b = b} for any term \ec{b}.
\begin{displaymath}
  \infrule{ }{\Env; \Gamma \vdash b = b}
\end{displaymath}

%split
\addTactic{split}{}
\tacname{} breaks a goal that is intrinsically conjunctive into multiple subgoals.
 For instance, it
 \begin{itemize}
  \item closes any goal that is \refdef{convertible} to \ec{true} or provable
        by \ec{reflexivity},

  \begin{displaymath}
  \infrule{\Env; \Gamma \vdash a \equiv true}{a}
  ~~~~~~
  \infrule{\Env; \Gamma \vdash a \equiv b}{a = b}
  \end{displaymath}
       
  \item replaces a logical equivalence by the direct and indirect implication,

  \begin{displaymath}
  \infrule{\Env; \Gamma \vdash \phi_1 \Rightarrow \phi_2 \quad
           \Env; \Gamma \vdash \phi_2 \Rightarrow \phi_1}
          {\Gamma \vdash \phi_1 \Leftrightarrow \phi_2}
  \end{displaymath}
  
  \item replaces a goal of the form \rawec{f1 /\\ f2} or \rawec{f1 \&\& f2} by the two
        subgoals for \ec{f1} and \ec{f2},

  \begin{displaymath}
  \infrule{\Env; \Gamma \vdash \phi_1 \quad
           \Env; \Gamma \vdash \phi_2}
          {\Env; \Gamma \vdash \phi_1 \land \phi_2}
  ~~~~~~
  \infrule{\Env; \Gamma \vdash \phi_1 \quad
           \Env; \Gamma \vdash \phi_2}
          {\Env; \Gamma \vdash \phi_1 \&\& \phi_2}
  \end{displaymath}
        
  \item replaces an equality between two $n$-tuples by the $n$ equalities of
        of the paired components.

  \begin{displaymath}
  \infrule{\Env; \Gamma \vdash a_1 = b_1  \quad \cdots \quad
           \Env; \Gamma \vdash a_n = b_n}
          {\Gamma \vdash (a_1, ..., a_n) = (b_1, ..., b_n)}
  \end{displaymath}
\end{itemize}

%left / right
\addTacticNoIdx{left / right}{}
\addTacticIdx{left}
\addTacticIdx{right}
Reduce a disjunctive goal to its left (resp. right) part
\begin{displaymath}
  \infrule{\Env; \Gamma \vdash \phi_1}{\Env; \Gamma \vdash \phi_1 \lor \phi_2}
  ~~~~~~
  \infrule{\Env; \Gamma \vdash \phi_2}{\Env; \Gamma \vdash \phi_1 \lor \phi_2}
\end{displaymath}

%case
\addTactic{case}{\tacarg{f}{formula}}
Do an excluded-middle case analysis on \ec{f}
\begin{displaymath}
  \infrule{\Env; \Gamma \vdash b \Rightarrow \phi(true) \quad
           \Env; \Gamma \vdash \neg b \Rightarrow \phi(false)}
          {\Env; \Gamma \vdash \phi(b)}
\end{displaymath}

%cut
\addTactic{cut}{\tacarg{ip}{intro-pattern} : \tacarg{C}{formula}}
Logical cut. Generates two subgoals: on for $C$ (the cut formula),
 and one for $C \Rightarrow G$ where $G$ is the initial goal. Moreover,
 the intro-pattern \ec{ip} is applied to the second subgoal.
\begin{displaymath}
  \infrule{\Env; \Gamma \vdash \phi_1 \quad
           \Env; \Gamma, \vdash \phi_2 \Rightarrow \phi_1}
          {\Env; \Gamma \vdash \phi_1}
\end{displaymath}

%elim
\addTactic{elim}{\tacarg{h}{ident}}
This tactics take as argument the name of a \refdef{judgment} from the \refdef{context} or the \refdef{scope}.
\begin{displaymath}
  \infrule{\Gamma, h:A \land B \vdash A \Rightarrow B \Rightarrow G}{\Gamma, h:A \land B \vdash G}
  ~~~~~~
  \infrule{\Gamma, h:\exists x, A(x) \vdash \forall x, A(x) \rightarrow G}{\Gamma, h:\exists x, A \vdash G}
\end{displaymath}\\
\begin{displaymath}
  \infrule{\Gamma, h:(a_1, ..., a_n) = (b_1, ..., b_n) \vdash a_1 = b_1 \Rightarrow ... \Rightarrow a_n = b_n \Rightarrow G}{\Gamma, h:(a_1, ..., a_n) = (b_1, ..., b_n) \vdash G}
\end{displaymath}

%elimT
\addTacticNoIdx{elim}{$\!\!$/\tacarg{h}{ident} \tacarg{f}{pattern}}
\addTacticIdx{elimT}
Apply the induction principle \vararg{h} on \vararg{x}

%apply
\addTactic{apply}{\tacarg{p}{proof-term}}
Modus Ponens. If \ec{p} is a proof-term for the pattern (formula) for
  \begin{center}
    \ec{forall (x1 : t1) ... (xn : tn), A1 -> ... -> An -> B}
  \end{center}
  \noindent then \tacname{} tries to match B with the current G. If the
  match succeeds and leads to the full instantiation of the pattern,
  then the goal is replaced, after instantiation, with the $n$ subgoals
  \ec{A1, ..., An}

%rewrite
\addTactic{rewrite}{rw1 ... rw${}_n$ where the rw${}_i$ are of the form \ec{//},
\ec{/=}, \ec{//=}, a proof-term or a pattern prefixed by \ec{/}
(slash). The two last forms can be prefixed by a direction indicator (the sign
\ec{-}), followed by an occurrence selector (\ec{\{i1 ... in\}}),
followed by repetition marker (\ec{!}, \ec{?}, \ec{i!} or \ec{i?}). All
these prefixes are optional.}
Depending on the form of \ec{rw}, \tacname{} \ec{rw} does the following:
  \begin{itemize}
   \item For \ec{//}, \ec{/=}, and \ec{//=}, see \ec{intros}.
   \item If \ec{rw} is a proof-term for the pattern (formula)
     \begin{center}
      \ec{forall (x1 : t1) ... (xn : tn), A1 -> ... -> An -> f1 = f2}
     \end{center}
     \noindent then \tacname{} searches for the first subterm of the goal
     matching \ec{f1} and resulting in the full instantiation of the pattern.
     It then replaces, after instantiation of the pattern, all the occurrences
     of \ec{f1} by \ec{f2} in the goal, and creates $n$ new subgoals for the
     \ec{Ai}'s. If no subterms of the goal match \ec{f1} or if the pattern
     cannot be fully instantiated by matching, the tactic fails.
     The tactic works the same if the pattern ends by \ec{f1 <-> f2}. If the
     direction indicator \ec{-} is given, \tacname{} works in the reverse
     direction, searching for a match of \ec{f2} and then replacing all
     occurrences of \ec{f2} by \ec{f1}.
   \item If \ec{rw} is a \ec{/}-prefixed pattern of the form \ec{(o p1 ... pn)},
     with \ec{o} a defined symbol, then \tacname{} searches for the first subterm
     of the goal matching \ec{(o p1 ... pn)} and resulting in the full instantiation
     of the pattern. It then replaces, after instantiation of the pattern, all
     the occurrences of \ec{(o p1 ... pn)} by the $\beta\delta$ head-normal form
     of \ec{(o p1 ... pn)}, where the $\delta$-reduction are restricted to the one
     headed by the symbol \ec{o}. If no subterms of the goal match \ec{(o p1 ... pn)} or
     if the pattern cannot be fully instantiated by matching, the tactic fails. If the
     direction indicator \ec{-} is given, \tacname{} works in the reverse
     direction, searching for a match of the $\beta\delta_{\rm o}$ head-normal
     of \ec{(o p1 ... pn)} and then replacing all occurrences of this head-normal
     form with \ec{(o p1 ... pn)}.
  \end{itemize}
  
  \smallskip
  
  The occurrence selector \ec{\{i1 ... in\}} allows to restrict which occurrences
  of the matching pattern are replaced in the goal. If given, only the
  \ec{i1}-th, ..., \ec{in}-th ones are replaced (considering that the goal is
  traversed in DFS mode). Note that this selection applies after the matching has
  been done.
  
  \medskip
  
  Repetition markers allow the repetition of the same rewriting. For instance,
  \tacname{} \ec{!rw} leads to \ec{do!} \tacname{} \ec{rw}. See \ec{do} for
  more information.
  
  \medskip

  Last, \tacname{} \ec{rw1 ... rwn} is equivalent to
  \tacname{} \ec{rw1}; ...; \tacname{} \ec{rwn}
  
%subst
\addTactic{subst}{\tacarg{x}{ident}?}
Search for the first equation of the form \ec{x = f} or \ec{f = x} in the context
 and replace all the occurrences of \ec{x} by \ec{f} everywhere in the context and the
 goal before clearing it. If no idents are given, repeatedly apply the tactic to
 all identifiers for which such an equation exists.

%congr
\addTactic{congr}{}
This tactic applies to a goal of the form \ec{f t1 ... tn = f u1 ... un}
 replacing it by  the subgoals \ec{ti = ui} for all \ec{i}. Note that subgoals
 solvable by \ec{reflexivity} are automatically closed.

%%%%%%%%%%%%%%%%%%%%%%%%%%%%%%%%%
% AUTO
%%%%%%%%%%%%%%%%%%%%%%%%%%%%%%%%%

\subsection{Automation}

%smt
\addTactic{smt}{[\ec{nolocal}]}
Try to solve the goal using SMT solvers. The goal is sent along with all the
 lemmas proved so far plus the local hypotheses, unless the \ec{nolocal} is
 given.
 
 \noindent\begin{center}
 \warningbox{Not all lemmas can be sent translated in such a way that they can
  be sent to the SMT provers. For instance, any formulas involving pRHL
  constructions are ignored.}
 \end{center}

%progress
\addTactic{progress}{\ec{tactic}?}
Break the goal into multiple \emph{simpler} ones by repeatedly apply
\ec{split}, \ec{subst} and \ec{intros}. If a tactic is given to \tacname{},
it is tentatively applied after each step.

%trivial
\addTactic{trivial}{}
Try to solve the goal by calling \ec{try assumption; progress; assumption}.
This is the tactic call by the intro-pattern \ec{//}.

\expandafter\createEasycrypt \ambientKeywords \relax

\lstdefinestyle{easycrypt-default}{
  columns=fullflexible,
  captionpos=b,
  frame=tb,
  xleftmargin=.1\textwidth,
  xrightmargin=.1\textwidth,
  rangebeginprefix={(**\ begin\ },
  rangeendprefix={(**\ end\ },
  rangesuffix={\ *)},
  includerangemarker=false,
  basicstyle=\small\sffamily,
  identifierstyle={},
  keywordstyle=[1]{\itshape\color{OliveGreen}},
  keywordstyle=[2]{\bfseries\color{Blue}},
  keywordstyle=[3]{\bfseries},
  keywordstyle=[4]{\bfseries},
  keywordstyle=[5]{\bfseries\color{OliveGreen}},
  keywordstyle=[6]{\itshape\color{Blue}},
  keywordstyle=[7]{\itshape\color{Red}},
  literate={phi}{{$\!\phi\,$}}1
           {phi1}{{$\!\phi_1$}}1
           {phi2}{{$\!\phi_2$}}1
           {phi3}{{$\!\phi_3$}}1
           {phin}{{$\!\phi_n$}}1
}

\lstdefinestyle{easycrypt-pretty}{
    basicstyle=\small\sffamily,
    literate={:=}{{$\mathrel{\gets}$}}1
              {<=}{{$\mathrel{\leq}$}}1
              {>=}{{$\mathrel{\geq}$}}1
              {<>}{{$\mathrel{\neq}$}}1
              {=\$}{{$\stackrel{\$}{\gets}$}}1
              {forall}{{$\forall$}}1
              {exists}{{$\exists$}}1
              {->}{{$\rightarrow\;$}}1
              {<-}{{$\leftarrow\;$}}1
              {=>}{{$\Rightarrow\;$}}1
              {==>}{{$\Rrightarrow\;$}}1
              {\/\\}{{$\wedge$}}1
              {\\\/}{{$\vee$}}1
              {.\[}{{[}}1
              {''ora}{{$\mathrel{||}$}}1 %needed for correct display in index
              {'a}{{\color{OliveGreen}$\alpha\,$}}1
              {'b}{{\color{OliveGreen}$\beta\,$}}1
              {'c}{{\color{OliveGreen}$\gamma\,$}}1
              {'t}{{\color{OliveGreen}$\tau\,$}}1
              {'x}{{\color{OliveGreen}$\chi\,$}}1
              {lambda}{{$\lambda\,$}}1
}

%% Typesetting
\newcommand{\titledbox}[4]{{\color{#1}\fbox{\begin{minipage}{#2}{\textbf{#3:} \color{black}#4}\end{minipage}}}}
\newcommand{\warningbox}[1]{\titledbox{red}{.9\textwidth}{Warning}{#1}}
\newcommand{\futurebox}[1]{\titledbox{blue}{.9\textwidth}{Future}{#1}}

%%% Local Variables: 
%%% mode: latex
%%% TeX-master: "easycrypt"
%%% End: 


\newcommand{\EC}{\textsc{EasyCrypt}\xspace}
\newcommand{\ECversion}{0.$\beta$\xspace}

\newcommand{\pRHL}{\mathsf{pRHL}\xspace}
\newcommand{\hoareS}[3]{\left[{#1}:{#2}\Longrightarrow{#3}\right]}
\newcommand{\bdHoareS}[5]{{\hoareS{#1}{#2}{#3}} {#4}{#5}}
\newcommand{\bdHoareSle}[4]{\bdHoareS{#1}{#2}{#3}{\leq}{#4}}
\newcommand{\bdHoareSeq}[4]{\bdHoareS{#1}{#2}{#3}{=}{#4}}
\newcommand{\bdHoareSge}[4]{\bdHoareS{#1}{#2}{#3}{\geq}{#4}}

\lstset{numberbychapter=false} %% Set to true if chapter numbering is desired


\title{\EC Manual}
\date{Version \ECversion --- \today}
\author{The \EC Team}

\begin{document}
\maketitle

\tableofcontents

\part{User Manual}
% Getting Started: Installation and Basic Usage (updated ElGamal?)
% !TeX root = easycrypt.tex

\chapter{Getting Started}
\section{Building and Running \EasyCrypt}
The easiest way to install \EasyCrypt on a Unix-based OS is to use the local
toolchain-based installation scripts available from our public \texttt{git}
repository. Other solutions exist, but the wide variety of machines and
configurations we expect \EasyCrypt users to own prevent us from documenting or
supporting them all. We strongly recommend using the local installation scripts
unless you are ready to deal with configuration issues yourself.

\subsection{Building \EasyCrypt}
To build \EasyCrypt from scratch, you will need the following standard tools:
\texttt{git}, \texttt{make}, \texttt{curl}, \texttt{autoconf}, \texttt{gcc} and \texttt{g++}.

The following bash script clones our public \texttt{git} repository, builds the
local toolchain (including the provers), activates it and builds \EasyCrypt.

\begin{verbatim}
  git clone http://ci.easycrypt.info/easycrypt.git 
  cd easycrypt
  make toolchain provers
  $(./scripts/activate-toolchain.sh)
  make && make check
\end{verbatim}

The toolchain and provers may take up to an hour to build, but this only has to
be done once. To keep your version of \EasyCrypt up to date, simply run the
following command.

\begin{verbatim}
  git pull && make
\end{verbatim}

\paragraph{On the test suite}
The final \texttt{make check} runs \EasyCrypt on our test suite and
the standard libraries. Please do report any failure in the test suite
(if possible, attach the .xml file produced by \texttt{make
  check-xunit}) to
\url{easycrypt-support@lists.gforge.inria.fr}. Unless a large number
of tests fail, a failure at this point does not necessarily mean that
the installation failed: SMT solvers can be fickle beasts and the
computing capacity of computers varies greatly. We are doing our best
to make calls to SMT in the standard library as stable as possible
over a wide variety of machines, but we need your help to test them in
as many scenarios as possible and are counting on you to report
failures.

\paragraph{Included provers}
The local toolchain contains a minimal set of provers to which you can easily
add your own. Please look at \WhyThree's and your favourite prover's
documentation for installation instructions. By default, all installed provers
are used when SMT is called. This behaviour can be changed using a command-line
option and language pragmas.

\subsection{\EasyCrypt's interactive mode}
The instructions above give you a command-line version of \EasyCrypt that can be
used to check proof scripts, but is difficult to use to develop proofs. We also
provide an interactive interface based on \texttt{emacs} and ProofGeneral.
Again, we recommend using our local installation scripts and will only provide
limited support if you choose not to.

To install \EasyCrypt's interactive mode, ensure that you have \texttt{emacs}
version 23.3 or higher, and that it is in your \texttt{\$PATH}. Under MacOSX,
\texttt{emacs} is detected when installed in the system Applications or using
MacPorts. The following command, run from the public repository's root, will
fetch and compile the latest version of Proof General, along with the \EasyCrypt
definitions.
\begin{verbatim}
  make -C proofgeneral local
\end{verbatim}
You can then run \texttt{emacs} using this fresh configuration of Proof General
by running the following command from the repository's root and opening files as
normal.
\begin{verbatim}
  make pg
\end{verbatim}

\subsection{Bug reporting}
For now, please use the GForge tracker
(\url{https://gforge.inria.fr/tracker/?group_id=2622}) to report bugs and
unsightly behaviours. This requires an account on the forge (which will ask you
for an INRIA project-team name %that you can probably fill in at random
to create an account). We are working on setting up an alternative bug tracking
system.

\subsection{This documentation}
The source for this document, along with the macros and language definitions
used, are available in the public repository.

%%% Local Variables: 
%%% mode: latex
%%% TeX-master: "easycrypt"
%%% End: 

% Theories: Types and Operators, Modules, and Cloning
% !TeX root = easycrypt.tex

\chapter{The Languages of \EasyCrypt\label{chap:theories}}
Definitions and lemmas can be grouped in theories, that can be imported to
provide new language functionalities as required for a particular proof.
Currently, \EasyCrypt supports user-defined types and operators
(Section~\ref{sec:types}) and user-defined distributions
(Section~\ref{sec:distributions}). In addition, theories can declare modules and
functors (Section~\ref{sec:modules}), used to represent schemes, oracles and
experiments, as well as abstract types for such modules, which can be used to
modularize proofs or represent abstract adversaries. Finally, theories can be
cloned and refined (Section~\ref{sec:cloning}), which allows the user to
consider small variants of a theory without having to create a new one from
scratch, or to consider concrete implementation details only when necessary for
the proof.


\section{The Expression Language\label{sec:types}}
At the core of any \EasyCrypt specification is a collection of types and functional
operators on those types. Formally, \EasyCrypt types are \emph{non-empty} sets of
values, and operators are \emph{mathematical} functions between them.

\warningbox{Currently, the easiest way to define types and operators is to
declare them abstractly and specify them using axioms. It is \emph{very}
important for the consistency of the logical context to remember that types are
\emph{always} assumed to be non-empty, and that operators are total mathematical
functions when writing such definitions.}

The \EasyCrypt type system supports polymorphic, higher-order types: the type of
lists can be declared independently of the type of their contents, and operators
can take and return other operators. We first describe the various built-in
types and relevant operators before moving on to explaining how new definitions
can be built from them.

\subsection{Built-Ins}
The language is equipped with a few built-in types\footnote{This is only to
bridge our language with \WhyThree's logic and avoid the introduction of
indirection layers in the proof obligations sent to the SMT solvers: all these
built-in types could in theory be defined in \EasyCrypt itself.}:
\begin{itemize}\itemsep-.5em
\item the \rawec{unit}\index{easycrypt}{types!unit@\ec{unit}} type, which
        contains a unique element
        \rawec{tt}\index{easycrypt}{constants!tt@\ec{tt : unit}},
\item the \rawec{bool} type of booleans,
\item the \rawec{int} type of arithmetic integers (in $\mathbb{Z}$),
\item the \rawec{real} type of real numbers (in $\mathbb{R}$).
\end{itemize}

Some symbols are defined, declared or imported from \WhyThree in the
corresponding libraries, and are discussed further in the library documentation.

\subsubsection*{The \rawec{bool} type}
\index{easycrypt}{types!bool@\ec{bool}}
Boolean expression terms can be built from the constant
\rawec{true}\index{easycrypt}{constants!true@\ec{true : bool}} and
\rawec{false}\index{easycrypt}{constants!true@\ec{true : bool}}, in combination
with operators for logical conjunction and disjunction
(\rawec{/\\}\index{easycrypt}{operators!and@\ec{(/\\) : bool -> bool -> bool}} and
\rawec{\\\/}\index{easycrypt}{operators!or@\ec{(\\/) : bool -> bool -> bool}}),
short-circuiting conjunction and disjunction
(\rawec{\&\&}\index{easycrypt}{operators!anda@\ec{(&&) : bool -> bool -> bool}}
and
\rawec{||}\index{easycrypt}{operators!ora@\ec{(''ora) : bool -> bool -> bool}}),
and xor
(\rawec{\^\^}\index{easycrypt}{operators!xor@\ec{(\^\^) : bool -> bool -> bool}}).
Additional operators and simple rewriting rules are made available in the
\rawec{Bool} theory.

\subsubsection*{The \rawec{int} type}
\index{easycrypt}{types!int@\ec{int}}
Integer expression terms can be built from numerical constants between $0$ and
$2^{62} - 1$ (non-negative OCaml integers), in combintation with operators for
unary negation (\rawec{-}), and all arithmetic and comparison (\rawec{+},
\rawec{-}, \rawec{*}, \rawec{/}, \rawec{\%}, \rawec{^}, \rawec{=}, \rawec{<},
\rawec{<=}, \rawec{>}, \rawec{>=} \ldots). It is necessary to import the integer
library (via \rawec{require import Int.}, see Section~\ref{sec:cloning}) to have
access to these operators. Since this type is identified with the \WhyThree type
of integers, all \WhyThree lemmas on it can be used by the SMT solvers, and are
accessible by name for use in interactive proofs. We will eventually be
providing uniformly named wrappers and algebraic structures.

\subsubsection*{The \rawec{real} type}
\index{easycrypt}{types!real@\ec{real}}
Real expression terms can be built from integer constants suffixed with a
\rawec{\%r} (for example, \rawec{0\%r} is 0 in the \rawec{real} type), in
combination with the same operators available on integers. It is necessary to
import the library (using \rawec{require import Real.}) to access these
operators in scope. This type is also identified with the \WhyThree type of
reals, which provides some lemmas and tighter integration with the underlying
theories, but SMT solvers often have very limited support for real numbers and
will often fail to prove seemingly trivial results, with Z3 often outperforming
other solvers. We will be working on enriching the library of lemmas available
in this theory to allow more complex interactive proofs when automation fails.

\warningbox{\rawec{(1/2)\%r} and \rawec{1\%r/2\%r} are very different
expressions. The former evaluates to $0$ (the division is in $\mathbb{Z}$, and
the result ($0$) is interpreted in $\mathbb{R}$) whereas the latter evaluates to
$0.5$ (the operands are interpreted and the operator is applied in
$\mathbb{R}$). Eventually, we will introduce notations for scoping that should
allow less cluttered real expressions.}

\subsection{Building Types}
From these base types, many more can be built. \EasyCrypt allows the user to declare
 (polymorphic) abstract types, and combine existing types into tuple types or
function types.



\subsubsection*{Tuple types}
Given two types \ec{'t} and \ec{'t'}, \ec{('t * 't')} denotes the type of pairs
of elements of \ec{'t} and \ec{'t'}. More generally, the product can be iterated
to produce tuple types of arbitrary arity. For example, Listing~\ref{lst:tuples}
declares two abstract types, \ec{pkey} and \ec{skey}, and then defines a type
\ec{keypairs} as the type of pairs of elements of \ec{pkey} and \ec{skey}.

\warningbox{Types \ec{('t * 't' * 't'')}, \ec{(('t * 't') * 't'')} and
\ec{('t * ('t' * 't''))} are distinct (but isomorphic).}

\begin{easycrypt}[label={lst:tuples}]{A simple tuple type.}
type pkey.
type skey.

type keypair = (pkey * skey).
\end{easycrypt}

\subsubsection*{Function types}
Given two types \ec{'t} and \ec{'t'}, \ec{'t -> 't'} is the type of total
functions from \ec{'t} to \ec{'t'}. Listing~\ref{lst:functions} illustrates such
a type definition, and the use of polymorphism, by defining a polymorphic type
of predicates. For any type \rawec{'a}, we construct a type \rawec{'a cpred} of
functions from \rawec{'a} to \rawec{bool}.

\begin{easycrypt}[label={lst:functions}]{A simple polymorphic function type.}
type 'a cpred = 'a -> bool.
\end{easycrypt}

\subsubsection*{Type application}
Polymorphic types can be instantiated with particular type parameters if
desired. For example, sets of integers could be defined as predicates on
integers as shown in Listing~\ref{lst:type_app}. Using the polymorphic type of
predicates defined in Listing~\ref{lst:functions}, we simply apply it to
\ec{int} to instatiate \rawec{'a} with \rawec{int}.

\begin{easycrypt}[label={lst:type_app}]{A simple type application.}
type intSet = int cpred.
\end{easycrypt}

\subsubsection*{On currying}
It is standard practice to write functions of several arguments in curried form,
where applying the function symbol to the first argument returns a function that
takes the second argument, and so on. This very easily allows the user to
partially apply functions to fix some of their arguments. This does not restrict
or increase the expressivity of the language in any way, although it may
initially cause some syntax issues for users unfamiliar with the concept. When
applying a curried function to several arguments, the arguments should be
separated by spaces rather than the more standard commas.
Listing~\ref{lst:currying} illustrates this.

\begin{easycrypt}[label={lst:currying}]{Currying and partial function application.}
op log: int -> int -> int.
op log2: int-> int = log_curried 2.

op log_pair: (int * int) -> int.
op log2_pair (n:int): int = log (2,n).
\end{easycrypt}

The \rawec{log} is declared as an integer function of two (curried) integer
arguments, and the log base 2, \rawec{log2}, is defined as its partial
application to 2. This definition could have been equivalently written as
follows.
\begin{easycrypt}[frame=none]{}
op log2 (n:int): int = log 2 n.
\end{easycrypt}
Contrasting with the curried form described above, \rawec{log_pair} is declared
as an integer function of a single argument (whose type is a pair type).
Partial application is slightly more verbose, but the syntax of application may
look more familiar to imperative programmers.

\subsection{Building Terms}
Under the assumption that all abstract types are inhabited, all types that can
be constructed using the operations above are also inhabited. We now show how
terms of each of the types can be constructed, serving as witnesses. The partial
applications in Listing~\ref{lst:currying} are small examples of building terms
of type \rawec{int -> int} given terms of other types.
%The reader eager to
%perform proofs can start drawing parallels between this subsection and the term
%reduction (computation) rules at the beginning of Chapter~\ref{chap:tactics}.
%TODO: More precise cross-reference

We provide almost all standard constructions of the polymophic
$\lambda$-calculus, with the notable exception of pattern-matching and fixpoint
constructions, since we do not currently support inductive types or recursion
directly. This limitation is easily worked around by axiomatically defining the
recursion and induction principle whenever declaring a type that is meant to be
inductive, and using them to provide realizations of axiomatically specified
theories. For example, the \textsf{List} standard theory of inductive lists
currently uses this two-stage architecture.

\warningbox{This version of the documentation is incomplete on this subject,
and only provides pointers to examples that may not be sufficient to fully
document the language. Please contact us as needed.}

\subsection{Predicates and Ambient Lemmas\label{sec:ec-specifics}}
In \EasyCrypt, booleans are used not only to represent binary program variables, but
also to type logical formulas. In addition to the operator symbols already
introduced, the \rawec{bool} type is therefore equipped with
symbols for universal and existential quantification, equivalence and
implication. However, since the program logic assumes that expressions are
terminating, we need to restrict the class of boolean formulas that can be used
when defining operators and writing programs. To this effect, we consider two
distinct classes of functional symbols: \emph{predicates}, that may not be
computable in general, and cannot be used to write programs (although they can
be used to specify them), and \emph{operators}, that are computable and can be
used for defining predicates and other operators, and to write programs.

Universal and existential quantification (\rawec{forall} and \rawec{exists}),
implication (\rawec{=>}) and equivalence (\rawec{<=>}) are internally defined as
predicate symbols, and therefore cannot be used to define operators. All other
boolean function symbols (in particular those discussed previously) are defined
as operators and can be used in any context.


% TODO: redistribute the following
%
%The following \EasyCrypt code (Listing~\ref{lst:bitstrings}) declares an abstract type of bitstrings, equipped
%with a length operator (whose result is never negative) and an infix addition
%operator that enjoys interesting properties both with respect to length and by
%itself.
%
%\begin{easycrypt}[label={lst:bitstrings}]{[A type for bitstrings]A core type for bitstrings}
%type bitstring.
%
%op length: bitstring -> int.
%axiom length_pos: forall bs, 0 <= length bs.
%
%op (+): bitstring -> bitstring -> bitstring.
%axiom length_xor: forall b0 b1,
%  length b0 = length b1 => length (b0 + b1) = length b0.
%axiom xor_associative: forall b0 b1 b2,
%  (b0 + b1) + b2 = b0 + (b1 + b2).
%axiom xor_commutative: forall b0 b1, b0 + b1 = b1 + b0.
%\end{easycrypt}
%
%\subsubsection{Types}
%\EasyCrypt expressions are equipped with polymorphic, higher-order types, that can be declared abstractly (as in Listing~\ref{lst:bitstrings}), or given concrete definitions, depending on the needs of the proof.
%
%
%\warningbox{One could expect the type annotations in the \rawec{extensionality}
%axiom to be optional. However, they are currently required so that the type
%variable can be instantiated.}
%
%\subsubsection{Operators}
%The language of operators and predicates is functional in style, and
%comma-separated lists of parameters are in fact currified, in this context, to
%yield functional symbols that can be partially applied. (For example, the infix
%extensional equality predicate \ec{(==)} defined in Listing~\ref{lst:arrays} has
%type \ec{'x array -> 'x array -> bool}.)
%
%\paragraph{Mixfix operators}
%Special syntax is used to introduce two hard-coded mixfix operators (generally
%considered to correspond to set and get operations on various types). The
%``\rawec[literate={\_}{{$\cdot$}}1]{_.[_]}'' operator (named \emph{get}
%throughout this manual) is hard-coded as a binary operator. The
%``\rawec[literate={_}{{$\cdot$}}1]{_.[_<-_]}'' operator (not yet encountered,
%and named \emph{set} in the rest of this manual) is hard-coded as a ternary
%operator. In addition, a constant operator ``\rawec{[]}'' can be defined, and
%will often be referred to as the \emph{empty} constant, depending on its type
%and the current scope.
%
%\paragraph{Infix operators}
%Infix operators are declared between parentheses, and can be used either in
%infix syntax using the symbol itself (for example, \rawec{x == y}), or in prefix
%syntax using their parenthesized form (for example, \rawec{(==) x y}).
%
%\subsection{Higher-Order Operators}
%Operators can be higher-order. For example, the code sample shown in
%Listing~\ref{lst:init_arrays} illustrates how one can axiomatically specify an
%operator which creates a fresh array and initializes its elements with
%index-dependent values. We illustrate its usage by defining an operator
%\ec{init} that, given an integer $n$, produces the array containing integers $0$
%through $n$, in ascending order.
%
%\begin{easycrypt}[label={lst:init_arrays}]{[Index-dependent array initializer]An operator initializing an array with index-dependent values}
%op init: (int -> 'x) -> int -> 'x array.
%
%axiom init_length: forall (f:int -> 'x) l,
%  0 <= l => length (init f l) = l.
%
%axiom init_get: forall (f:int -> 'x) l i,
%  0 <= l => 0 <= i => i < l =>
%  (init f l).[i] = f i.
%
%op first_ints n = init (lambda i, i) n.
%\end{easycrypt}
%
%The \rawec{lambda} notation works as expected. The argument types can be
%specified where necessary. Equality on lambda terms is extensional.

\section{Probabilistic Expressions\label{sec:distributions}}

%%%% From Gilles:
\EasyCrypt features a polymorphic type \rawec{'a distr} of \emph{discrete
sub-distributions} over a base type \rawec{'a}. The primary operation over
sub-distributions is the function \rawec{op mu: 'a distr -> ('a -> bool) ->
real.} which measures the probability of an event. The function is assumed to
satisfy the basic axioms of probability sub-distributions:
\begin{itemize}\itemsep-.5em
\item probabilities lie in the unit interval;
\item the probability of the union of events is equal to the sum of
  their probabilities, minus the sum of their intersection;
\item probabilities are monotonic with respect to event inclusion;
\item two sub-distributions over the same base type are equal if and only if
  they are equal on each of the elements in the base type.
\end{itemize}

Sub-distributions are introduced axiomatically, and no well-formedness checks
are performed. For example, the uniform distribution on booleans is defined, in
the standard library, as displayed in Listing~\ref{lst:dbool}, where
\rawec{caract P x} is \rawec{1\%r} if \rawec{P} holds on \rawec{x} and
\rawec{0\%r} otherwise.

\begin{easycrypt}[label={lst:dbool}]{[Uniform Boolean distribution]Defining the uniform distribution on booleans}
op dbool: bool distr.

axiom mu_def: forall (p:bool -> bool), 
  mu dbool p =
    (1%r/2%r) * caract p true +
    (1%r/2%r) * caract p false.
\end{easycrypt}

The standard library on distributions, discussed further in
Chapter~\ref{chap:libraries}, introduces several auxiliary quantities on
distributions that are useful in probability computations and for defining more
complex distributions that are difficult, or impossible, to describe purely
using \rawec{mu}. The definitions of these quantities is shown in
Listing~\ref{lst:mu_aux}, where \rawec{cPtrue} is the constantly true predicate.
Note the use of partial application to the equality operator when defining
\rawec{mu_x}.

\begin{easycrypt}[label={lst:mu_aux}]{Auxiliary operators on distributions}
op mu_x(d:'a distr, x:'a) = mu d ((=) x).
op mu_weight(d:'a distr) = mu d cPtrue.
op in_supp(x:'a, d:'a distr) = 0%r < mu_x d x.
\end{easycrypt}

With these auxiliary operators, it is in fact fairly easy to check that the simple axiom from Listing~\ref{lst:dbool} does indeed define a distribution, by proving the following lemma, which is discharged automatically by the SMT solvers.

\begin{easycrypt}[]{}
lemma lossless : weight dbool = 1%r.
\end{easycrypt}

\section{Modules and Functors\label{sec:modules}}

\subsection{\pWHILE: Schemes, Oracles and Experiments}
So far, we have only considered features whose goal is to extend the language of
expressions and the semantic domain of values. Specifications of schemes,
oracles, assumptions and game-based security properties all use modules (and
module signatures) and functors, which we now discuss, using a simple modular
definition of a random oracle from bitstrings to bitstrings as an example.

\subsection{Module Signatures}
We first formally define the functionalities a random oracle is expected to
provide, as a \emph{module type}, or \emph{signature}
(Listing~\ref{lst:modulesig}). Any module implementation \rawec{M} that provides
\emph{at least} the functions from a module type \rawec{Mt} is said to be of
type \rawec{Mt} (denoted \rawec{M :> Mt}). Module types cannot specify state
directly. Instead, if a global variable of the module should be exposed to the
outside, getter and setter functions can be added to the module signature.

\begin{easycrypt}[label={lst:modulesig}]{[A signature for random oracles]A signature for random oracles from bitstrings to bitstrings}
module type RO = {
  fun init():unit
  fun h(x:bitstring):bitstring }.
\end{easycrypt}

Such a module signature can then be given various realizations. For example, Listing~\ref{lst:modules} shows two possible realizations of a random oracle, both of which assume a positive integer constant \rawec{qH}, used to bound the number of calls to the oracle, and two positive integer constants \rawec{inLen} and \rawec{outLen} representing the input and output lengths of the random oracles.

%% The following is probably the single most disgusting thing I've ever typeset in Latex... and I've done ugly things.
\begin{minipage}{\textwidth}
\hrule
\begin{multicols}{2}
\begin{easycrypt}[frame=none,xleftmargin=0pt,xrightmargin=0pt]{}
module RO_L: RO = {
  var cG: int
  var mG: (bitstring,bitstring) map

  fun init() = {
    mG = empty;
    cG = 0;
  }

  fun h(x:bitstring) = {
    var r:bitstring;
    var res:bitstring = empty;
    if (length x = inLen && cG < qG)
    {
      cG = cG + 1;
      r = $dbitstring(outLen);
      if (!mem(x,dom mG)) mG[x] = r;
      res = mG[x];
    }
    return res;
  }
}.
\end{easycrypt}
\columnbreak
\begin{easycrypt}[frame=none,xleftmargin=0pt,xrightmargin=0pt]{}
module RO_E: RO = {
  var tape: bitstring list
  var mG: (bitstring,bitstring) map

  fun init() = {
    var r:bitstring;
    tape = [];
    mG = empty;
    while (length tape < qG)
    {
      r = $dbitstring(outLen);
      tape = r :: tape;
    }
  }

  fun h(x:bitstring) = {
    var r:bitstring;
    var res:bitstring = empty;
    if (length x = inLen &&
       length tape <> 0)
    {
      r = hd tape;
      if (!mem(x,dom mG))
        mG[x] = r;
      res = mG[x];
      tape = tl tape;
    }
    return res;
  }
}.
\end{easycrypt}
\end{multicols}
\hrule
\begin{easycrypt}[frame=non,xleftmargin=0pt,xrightmargin=0pt,label={lst:modules}]{[Two random oracles]Two possible implementations of module type \rawec{RO}}
\end{easycrypt}
\end{minipage}

\subsection{Functors and Oracle Annotations}
Modules can in fact be specified as functors, parameterized by other modules
whose functions can be accessed as oracles. For example, the signature shown in
Listing~\ref{lst:functorsig} describes a class of modules that are given access
to a random oracle but whose unique function can in fact only call the \rawec{h}
function provided by the parameter.

\begin{easycrypt}[label={lst:functorsig}]{[A functor signature]A functor signature}
module type RO_adv(O:RO) = {
  fun a(): bool { O.h }
}.
\end{easycrypt}

\section{Lemmas and Judgements}

\warningbox{Note that we use different implication symbols. $\Rightarrow$ is implication in the ambient logic (corresponding to the \rawec{=>} predicate), whereas $\Longrightarrow$ is used to express contracts on functions.\footnote{From Francois: it may be worth considering the use of another arrow symbol (for example $\Rrightarrow$) for contracts.}}

\subsection{(Possibilistic) Hoare Judgements}

Interpretation:
\begin{displaymath}
\Hoare{c}{\pre}{\post}
\quad\doteq\quad
\forall m.~ \pre\,m \Rightarrow \range Q ([\![ c ]\!]\,m)
\end{displaymath}
%
where $\range \varphi \mu \doteq \forall f.~ (\forall x.~ \varphi\,x
\Rightarrow f\,x=0) \Rightarrow \mu\,f=0$




\subsection{Probabilistic Hoare Judgements}

Interpretation:
\begin{displaymath}
\HoareLe{c}{\pre}{\post}{\delta} 
\quad\doteq\quad
\forall m.~ \pre\,m \Rightarrow [\![ c ]\!]\,m\,\charfun_\post \leq
\delta\,m
\end{displaymath}

Property?:
\begin{displaymath}
\Hoare{c}{\pre}{\post}
\Leftrightarrow
\HoareEq{c}{\pre}{\neg\post}{0}
\end{displaymath}

Property?:
\begin{displaymath}
\Hoare{c}{\pre}{\post} \land \HoareEq{c}{\pre}{\true}{1}
\Leftrightarrow
\HoareEq{c}{\pre}{\post}{1}
\end{displaymath}

\subsection{Relational Hoare Judgements}

\section{Sections}

\section{Working with Theories\label{sec:cloning}}




%%% Local Variables: 
%%% mode: latex
%%% TeX-master: "easycrypt"
%%% End: 


% Tactics: First-Order and pRHL Tactics
% !TeX root = easycrypt.tex

%% TODO (Francois): For index, rather than \texttt, use \rawec and make a class of keywords for tactics and tacticals

\chapter{Writing Proofs}

\EC comes with a proof engine that allows to state, in the \EC
underneath formalism, properties about the user defined programs
and to prove them.
%
Proofs are built interactively, starting from final goal, by applying
\emph{tactics} that transform a goal (the property we want to prove)
to a set of subsequent goals (the subgoals) s.t. the latter logical
implies the former.
%
This process is repeated iteratively up to the point where all the
subgoals are trivial and can be solved by the system.

This chapter is about the description of this proof engine, and is
structured as follow. We first define the notion of goals and show
how it relates to the \EC formalism. We then introduce the notion
of tactics as logically valid goal transformers. Finally, a listing
of all the existing tactics, along with their detailed descriptions,
is given.

\section{The proof engine}

The proof engine deals with \emph{judgments} or \emph{goals} of the form
$\Env; \Gamma \vdash \phi$ where $\Env$ is the (global) environments,
$\Gamma$ is a set of local facts and $\phi$ is the formula we want
to prove. Here is an example of such a judgment:

\begin{center}
$\Int; x, y , z: \tint, x \le y \vdash x + z \le y + z$.
\end{center}

It states that in the \emph{environment} ($\Env$) solely composed of the
theory $\Int$, having three local variables $x, y, z$ of type $\tint$ along
with the fact $x \le y$ (the \emph{context} $\Gamma$), we are interested
in proving $x + z \le y + z$.

\medskip

On top on this, a set of \emph{deduction rules} is given. They
describe how one can derive a judgment $\Env; \Gamma \vdash \phi$ given
that a set of prerequisites (or \emph{premises}) are fulfilled. The general
form of such a rule is given as follow:

\begin{displaymath}
 \infrule{A_1 \cdots A_n}{\Env; \Gamma \vdash \phi}
\end{displaymath}

It has to be read as: \emph{given that $A_1 \cdots A_n$ are derivable, then
so is $\Env, \Gamma \vdash \phi$}. We give three examples of such deduction
rules:

\begin{displaymath}
 \infrule
         {\Env; \Gamma \vdash \phi_1 \quad
          \Env; \Gamma \vdash \phi_1 \Rightarrow \phi_2}
         {\Env; \Gamma \vdash \phi_2}
         {\rname{MP}}
 \quad\quad
 \infrule
         {\Env; \Gamma, \phi_1 \vdash \phi_2}
         {\Env; \Gamma \vdash \phi_1 \Rightarrow \phi_2}
         {\rname{$\Rightarrow$-I}}
 \quad\quad
 \infrule{ }{\Env; \Gamma, \phi, \Delta \vdash \phi}{\rname{Ax}}
\end{displaymath}

The first, the \emph{modus ponens}, states that one can derive
$\Env; \Gamma \vdash \phi_2$ given that $\Env; \Gamma \vdash \phi_1
\Rightarrow \phi_2$ and $\Env; \Gamma \vdash \phi_1$ are derivable.
%
The next provides a way for deriving $\phi_1 \Rightarrow \phi_2$ from
a derivation of $\phi_2$, but with a context augmented by $\phi_1$.
%
The last states that $\Env; \Gamma, \phi, \Delta \vdash \phi$ is derivable as-is.

\medskip

Combining these deduction rules, it is possible to build a tree rooted by
a judgment $\Env; \Gamma \vdash \phi$ and with leaves composed of deduction
rules with no premises (as the third one in the previous example). Such a
tree forms a \emph{proof} of $\Env; \Gamma \vdash \phi$.
%
For instance, Figure~\ref{fig:LJproof} gives a proof of
%
\begin{center}
 $\Env; b_1, b_2 : \tbool \vdash (b_1 \Rightarrow b_2) \Rightarrow b_1 \Rightarrow b_2$
\end{center}

\begin{figure}
  \begin{displaymath}
    \infrule
      {\infrule{ }{\Env; b_1, b_2 : \tbool, b_1 \Rightarrow b_2, b_1 \vdash b_1 \Rightarrow b_2} \quad
       \infrule{ }{\Env; b_1, b_2 : \tbool, b_1 \Rightarrow b_2, b_1 \vdash b_1}}
      {\infrule
        {\Env; b_1, b_2 : \tbool, b_1 \Rightarrow b_2, b_1 \vdash b_2}
        {\infrule
           {\Env; b_1, b_2 : \tbool, b_1 \Rightarrow b_2 \vdash b_1 \Rightarrow b_2}
           {\Env; b_1, b_2 : \tbool \vdash (b_1 \Rightarrow b_2) \Rightarrow b_1 \Rightarrow b_2}}}
  \end{displaymath}

  \caption{\label{fig:LJproof} Proof tree of
    $\Env; b_1, b_2 : \tbool \vdash
        (b_1 \Rightarrow b_2) \Rightarrow b_1 \Rightarrow b_2$}
\end{figure}

\bigskip

The \EC proof engine helps the user building such proof. At each step
of the proof building, the system presents to the user the set of goals
that has to be proved. The user can then \emph{apply} a tactic to one of
them, each tactic corresponding to a deduction rule. If the conclusion
of the rule corresponding to the applied tactic matches the goal to witch
it is applied, the proof engine replaces it with the set of the
premises of the applied rule - the subgoals. This application may generate
no, one or several subgoals depending on the rule. This process is repeated
iteratively, up to the point where no goals remain.

\section{Ambient Logic (Guillaume)}

\begin{center}
\begin{tabular}{l@{$\quad$}l@{$\quad$}ll}
{\rawec{(lambda (x : t), phi1)\ phi2}} & $\rightarrow_\beta$ &
  \multicolumn{2}{@{}l}{{\rawec{phi2} \{\rawec{x} $\leftarrow$ \rawec{phi1}\}}}\\
{\rawec{if (true) \{ phi1 \} else \{ phi2 \}}} & $\rightarrow_\iota$ &
  {\rawec{phi1}}\\
{\rawec{if (false) \{ phi1 \} else \{ phi2 \}}} & $\rightarrow_\iota$ &
  \multicolumn{2}{@{}l}{{\rawec{phi2}}}\\
{\rawec{let (x1, ..., xn) = (phi1, ..., phin) in phi}} & $\rightarrow_\iota$ &
  \multicolumn{2}{@{}l}{{\rawec{phi} \{ \rawec{x1, ..., xn} $\leftarrow$ \rawec{phi1, ..., phin} \}}}\\
{\rawec{let x = phi1 in phi2}} & $\rightarrow_\zeta$ &
  \multicolumn{2}{@{}l}{{\rawec{phi2} \{ \rawec{x} $\leftarrow$ \rawec{phi1} \}}}\\
{\rawec{o}} & $\rightarrow_\delta^{\Env,\Gamma}$ &
  {\rawec{e}} & if {\rawec{op o := e}} $\in \Env$\\
{\rawec{x}} & $\rightarrow_\delta^{\Env,\Gamma}$ &
  {\rawec{phi}} & if {\rawec{x := phi}} $\in \Gamma$\\
\end{tabular}
\end{center}

\ambientDesc

\section{Program Transformation Tactics}

TODO: fun, inline, swap, unroll, splitwhile, fusion, fission, condt, condf, 

\section{Program Logics Tactics}

\subsection{: the \rawec{skip} tactic}

\Syntax \rawec{skip}

\Description Reduces logical program judgements with empty statements
to a first-order logical goal, as in the following rule for relational
Hoare Logic.
%
\begin{displaymath}
\infrule{
  \pre \Rightarrow \post
}{
  \equiv{}{}{\pre}{\post}
}
\end{displaymath}
%
Similar rules apply for Hoare judgements and probabilistic Hoare
judgements.

\subsection{Reasoning about random samplings: the \rawec{rnd} tactic}
%
\subsubsection{Hoare Logic}
\index{hoare}{Program Reasoning!rnd@\rawec{rnd}}

\Description

Assume $d:A\, \verb+distr+$...

\begin{displaymath}
\infrule{
  \Hoare{c}{\pre}{\forall z:A,in\_supp\,z\,d \Rightarrow \post\subst{x}{z}}
}{
  \Hoare{c;\Rand{x}{d}}{\pre}{\post}
}
\end{displaymath}

\subsubsection{Probabilistic Hoare Logic}
\index{phl}{Program Reasoning!rnd@\rawec{rnd}}
\Syntax 
\verb+rnd+ (\textit{formula} $|$ \_ ) (\textit{formula} $|$ \_ )

\Description
the first optional parameter $p$ is a computable predicate (i.e., \verb+'a cPred+)
(i.e., \verb+'a -> bool+ ). Assume $d$ of type \verb+A Distr.distr+. 
\begin{displaymath}
\begin{array}{c}
  \infrule{
    \Hoare{c}{\pre}{\mu\, d\, p \leq f \land 
      (\forall v\in \mathsf{support}(d).~ \post\subst{x}{v} \Rightarrow p\, v)}
  }{
    \HoareLe{c;\Rand{x}{d}}{\pre}{\post}{f}
  }\left[\verb+rnd+\ p\right]
\\[4ex]
\end{array}
\end{displaymath}
If $p$ is not given then the tool attempts to build it from $\post$
(not implemented yet).

For lower-bounded and exact probabilistic judgments the tactic
additionally accepts an optional parameter $g$ of type \verb+real+
representing a bound:
\begin{displaymath}
  \infrule{
    \HoareGe{c}{\pre}{\mu\, d\, p \geq g \land 
      (\forall v\in \mathsf{support}(d).~ p\, v \Rightarrow \post\subst{x}{v} )}{\frac{f}{g}} 
  }{
    \HoareGe{c;\Rand{x}{d}}{\pre}{\post}{f}
  }\left[\verb+rnd+\ p\ g\right]
\end{displaymath}
%
\begin{displaymath}
  \infrule{
    \HoareEq{c}{\pre}{\mu\, d\, p = g \land 
      (\forall v\in \mathsf{support}(d).~ p\, v \Leftrightarrow \post\subst{x}{v} )}{\frac{f}{g}} 
  }{
    \HoareEq{c;\Rand{x}{d}}{\pre}{\post}{f}
  }\left[\verb+rnd+\ p\ g\right]
\end{displaymath}
%
If $g$ is not given then $g=f$ in the rule.

\subsubsection{Relational Hoare Logic}
\index{prhl}{Program Reasoning!rnd@\rawec{rnd}}

\Syntax \verb+rnd+[\textit{side}] [\textit{bij\_info}]
where
\textit{bij\_info} is either
\begin{itemize}
  \item \textit{form} \textit{form}
  \item \textit{form} \_
\end{itemize}


\Description

The logical rule implemented by the \verb+rnd+ tactic depends on the
the optional parameter \textit{side}. If a left/right side flag is
provided then the one-sided logical rule for random sampling is
applied. If missing, then the two-sided rule for random assignment is
considered.
%

\paragraph*{Two-sided application.} 
In this case case, the \verb+rnd+ tactic takes as parameter a
representation of a bijective function. 

When two formulae are provided as the \textit{bij\_info} parameter,
they are verified to be a bijective function and its inverse. If only
one function is given then it is required to be an involution, and
lastly if no argument is given then the identity function is assumed.

The description of the rule below assumes that a bijective function
$f$ and its inverse is provided and generates according verification
conditions. Furthermore, it requires the following type constraints
for some types \verb+'a+ and \verb+'b+: 
\begin{itemize}
\item $d_1:\verb+'a distr+$,
\item $d_2:\verb+'b distr+$, $f:\verb+'a+\to\verb+'b+$,
\item $f^{-1}:\verb+'b+\to\verb+'a+$, 
\item ...
\end{itemize}

\begin{displaymath}
\infrule{
  \Equiv{c_1}{c_2}{\pre} 
  { \forall z,z',in\_supp \,z\,d_1\Rightarrow in\_supp \,z'\,d_2\Rightarrow
    \begin{array}{l}
      (\mu\,d_1\,\charfun_{\{z\}}=\mu\,d_2\,\charfun_{\{f\,z\}} ) 
      \land \\
      (in\_supp\,d_1\,(f^{-1}\,z'))
      \land \\
      (f^{-1}\,(f\,z)=z)
      \land \\
      (f\,(f^{-1}\,z')=z')
      \land \\
      (\post\subst{x_1}{z}\subst{x_2}{f\,z})
    \end{array}
  }
}{
  \Equiv{c_1;\Rand{x_1}{d_1}}{c_2;\Rand{x_2}{d_2}}{\pre}{\post}
}
\end{displaymath}

\paragraph*{Two-sided application.} 
The logical rule implemented when the optional parameter \textit{side}
is used is similar to the random sampling rule for Hoare judgements:


\begin{displaymath}
\infrule{
  \Equiv{c}{c'}{\pre}{\forall z:A,in\_supp\,z\,d \Rightarrow \post\subst{x}{z}}
}{
  \Equiv{c;\Rand{x}{d}}{c'}{\pre}{\post}
}
\end{displaymath}


\subsection{Reasoning about sequential composition: the \rawec{seq} tactic}
%
\subsubsection{Hoare Logic}
\index{hoare}{Program Reasoning!seq@\rawec{seq}}

\Syntax 
\verb+app+ \textit{codepos} \textit{formula} 

\Description

\Description
Applies the Hoare Logic rule for sequential composition:
$$
\infrule{\Hoare{c}{\post}{\post'} \quad
         \Hoare{c'}{\post'}{\post''}}
        {\Hoare{c;c'}{\post}{\post''}}
$$
The application of tactic \verb+app k p+ defines $c$ as the first
\verb+k+ instructions of the statement $c;c'$ and $\post'$ as
\verb+p+.


\subsubsection{Probabilistic Hoare Logic}
\index{phl}{Program Reasoning!seq@\rawec{seq}}
\Syntax 
\verb+app+ \verb+[>>|<<]+ \textit{codepos} \textit{formula} (
[\textit{formula} \verb+|+ \textit{formula} \textit{formula}
\textit{formula} \textit{formula}]

\Description
The application of the \verb+seq+ tactic is more complicated when
dealing with Probabilistic Hoare Logic judgements. 

The direction parameter is accepted for \emph{lower-bounded} and \emph{exact}
judgments. The direction \verb+<<+ is assumed by default (as it is globally).
The first formula represents the intermediate predicate that must hold
at the splitting program point.

In the following, the rule descriptions assume that $n$ indicates the
program position of statement $s_2$.

\paragraph*{Upper bounded judgements.}
For upper bounded judgments, the most general variant of the
\verb+app+ rule (i.e., when four bounds are given as parameters) implements the following rule:
\begin{displaymath}
  \infrule{
    \begin{array}{c}
      \HoareLe{s1}{P}{R}{f_1} \qquad \HoareLe{s2}{R}{Q}{f_2}
      \\
      \HoareLe{s1}{P}{R}{g_1} \qquad \HoareLe{s2}{R}{Q}{g_2}
      \\
      f_1 f_2 + g_1 g_2 \leq f 
    \end{array}
  }{
    \HoareLe{s1;s2}{P}{Q}{f}
  }
\end{displaymath}
%
If no argument is given then the following rule is applied:
\begin{displaymath}
  \infrule{
    \Hoare{s1}{P}{R} \qquad \HoareLe{s2}{R}{Q}{f}
  }{
    \HoareLe{s1;s2}{P}{Q}{f}
  }\left[\verb+app+\ n\ R\right]
\end{displaymath}
%
% \warningbox{Which, if preferred, can be rewritten to:}
% \begin{displaymath}
%   \infrule{
%     \Hoare{s1}{P}{\lambda m. \Prm{s_2}{m}{Q}\leq f} \qquad 
%   }{
%     \HoareLe{s1;s2}{P}{Q}{f}
%   }\left[\verb+app+\ n\ R\right]
% \end{displaymath}

Single bound parameters are not accepted for upper-bounded judgements.

\paragraph*{Lower-bounded and exact  judgements.}

The application of the \verb+seq+ tactic have similar regardless of
lower or exact bounds. 

The second optional parameter of type $\verb+real+$ represents a
probability bound (only supported for $=$ and $\geq$), and the
optional direction parameter indicates whether this bound is to be
applied to the first or second half of the split statement.

\begin{displaymath}
\begin{array}{c}
  \infrule{
    \HoareGe{s1}{P}{R}{f/g} \qquad \HoareGe{s2}{R}{Q}{g}
  }{
    \HoareGe{s1;s2}{P}{Q}{f}
  }\left[\verb+app+\ n\ R\ g\right]
\\[4ex]
  \infrule{
    \HoareGe{s1}{P}{R}{g} \qquad \HoareGe{s2}{R}{Q}{f/g}
  }{
    \HoareGe{s1;s2}{P}{Q}{f}
  }\left[\verb+app>>+\ n\ R\ g\right]
\end{array}
\end{displaymath}
%
%
Similar rules hold for $=$. If the bound parameter $g$ is not given then
$g$ is defined as $f$ in the above rule description.

\subsubsection{Relational Hoare Logic}
\index{prhl}{Program Reasoning!seq@\rawec{seq}}

\Syntax
\verb+app+ \textit{codepos} \textit{form}

\Description
Applies the RHL rule for sequential composition:
$$
\infrule{\Equiv{c_1}{c_2}{\post}{\post'} \quad
         \Equiv{c_1'}{c_2'}{\post'}{\post''}}
        {\Equiv{c_1;c_1'}{c_2;c_2'}{\post}{\post''}}[\textrm{R-Seq}]
$$
The application of tactic \verb+app m n p+ defines $c_1$ as the first
\verb+m+ instructions of the program on the left-hand side and $c_2$ as
the first \verb+n+ instructions of the program on the right-hand side
and $\post'$ as \verb+p+.



\subsection{Reasoning about conditionals: the \rawec{if} tactic}
%

\subsubsection{Hoare Logic}
\index{hoare}{Program Reasoning!if@\rawec{if}}
\index{phl}{Program Reasoning!if@\rawec{if}}

Applies the following rule for conditional statements. It expects a
conditional statement at the first program position.
\begin{displaymath}
\begin{array}{c}
  \infrule{
    \Hoare{c_1}{\pre \land b}{\post}\qquad
    \Hoare{c_2}{\pre \land \neg b}{\post}
  }{
    \Hoare{\Cond{b}{c_1}{c_2}}{\pre}{\post}
  }\left[\verb+if+ \right] 
\\[4ex]
\end{array}
\end{displaymath}


\subsubsection{Hoare and Probabilistic Hoare Logic}
\index{hoare}{Program Reasoning!if@\rawec{if}}
\index{phl}{Program Reasoning!if@\rawec{if}}

Applies the following rule for conditional statements. It expects a
conditional statement at the first program position.
\begin{displaymath}
\begin{array}{c}
  \infrule{
    \HoareLe{c_1}{\pre \land b}{\post}{f}\qquad
    \HoareLe{c_2}{\pre \land \neg b}{\post}{f}
  }{
    \HoareLe{\Cond{b}{c_1}{c_2}}{\pre}{\post}{f}
  }\left[\verb+if+ \right] 
\\[4ex]
\end{array}
\end{displaymath}
Similar rules hold for $=,\geq$.

\subsubsection{Relational Hoare Logic}
\index{prhl}{Program Reasoning!if@\rawec{if}}

\Syntax \verb+if+ [\textit{side}]

\Description Applies the pRHL rule for conditional.
If the \textit{side} argument is given then the corresponding
one side rule is used, else the two side rule is used.
The \verb+if+ tactic expects a conditional as first instruction. 
\begin{center}
\begin{tabular}{c|c}
Syntax & Rule \\
\hline\\
\verb+if{1}+ &
$
\infrule{\Equiv{c_1;c}{c'}{\pre \land e\sidel}{\post}
        \quad \Equiv{c_2;c}{c'}{\pre \land \neg e\sidel}{\post}}
        {\Equiv{\Cond{e}{c_1}{c_2};c}{c'}{\pre}{\post}}
$\\
\\\hline\\
\verb+if{2}+ &
$
\infrule{\Equiv{c'}{c_1;c}{\pre \land e\sider}{\post}
        \quad \Equiv{c'}{c_2;c}{\pre \land \neg e\sider}{\post}}
        {\Equiv{c'}{\Cond{e}{c_1}{c_2};c}{\pre}{\post}}
$\\
\\\hline\\
\verb+if+ &
$
\infrule{
 \begin{array}{c}
   \vdash \pre \Rightarrow e\sidel = e'\sider \\
   \Equiv{c_1;c}{c'_1;c'}{\pre \land e\sidel \land e'\sider}{\post}\\
   \Equiv{c_2;c}{c'_2;c'}{\pre \land \neg e\sidel \land \neg e'\sider}{\post}
 \end{array}
}{\Equiv{\Cond{e}{c_1}{c_2};c}
        {\Cond{e'}{c'_1}{c'_2};c'}
        {\pre}{\post}}
$\\
\end{tabular}
\end{center}


\subsection{Computing weakest preconditions: the \rawec{wp} tactic}
%

\Syntax \verb+wp+ [\textit{codepos}]

\Description The \verb+wp+ tactic computes the weakest-precondition of
deterministic, loop and procedure-call free program fragments
(i.e. deterministic assignments and conditionals).  If the op code
position parameter is not provided, The tactic processes instructions
bottom-up until a random sampling, a loop or a function call is
reached. The computation of the weakest precondition over a
conditional instruction is only possible if its branches do not
contain random samplings, while loops nor function calls.

The optional code position parameter \textit{pos} restricts the range
of instructions that may be affected by the tactic invocation. 
See \ref{???} for a description of its syntax.


\Example


\subsubsection{Hoare Logic}
\index{hoare}{Program Reasoning!wp@\rawec{wp}}

\subsubsection{Probabilistic Hoare Logic}
\index{phl}{Program Reasoning!wp@\rawec{wp}}

\begin{displaymath}
  \infrule{
    \HoareLe{c_1}{\pre }{\mathsf{wp}(c_2,\post)}{f}
  }{
    \HoareLe{c_1;c_2}{\pre}{\post}{f}
  }\left[\verb+wp+ \right] 
\end{displaymath}
Similar rules hold for $=,\geq$.

\subsubsection{Relational Hoare Logic}
\index{prhl}{Program Reasoning!wp@\rawec{wp}}

\subsection{Concluding proofs of programs: the \rawec{skip} tactic}
\index{hoare}{Program Reasoning!skip@\rawec{skip}}
\index{phl}{Program Reasoning!skip@\rawec{skip}}
\index{prhl}{Program Reasoning!skip@\rawec{skip}}
%

\subsection{Simplifying conditionals: the \rawec{condt,condf} tactic}
%
\subsubsection{Probabilistic Hoare Logic}
\index{tactics}{probabilistic Hoare logic!condt@\rawec{condt}}
\index{tactics}{probabilistic Hoare logic!condf@\rawec{condf}}

\subsection{Reasoning about abstract adversaries: the \rawec{fun} tactic}

\subsubsection{Relational Hoare Logic}

\Syntax \verb+fun+ formula

\Description
The formula given as parameter represents the general oracle
invariant. 

The tactic implements the following rule:
\begin{displaymath}
\infrule{
  \begin{array}{c}
    \pre \Rightarrow \chi \land \glob_A = \glob_B \land \vec{p}_A=\vec{p}_B
    \\[.5ex]
    \chi\land\glob_A=\glob_B\land\result_A=\result_B\Rightarrow\post
    \\ 
    \Equiv{O_i}{O_i'}{\chi\land
      \vec{p}_{O_i}=\vec{p}_{O'_i}}{\chi\land \result_{o_i}=\result_{o'_i}}
  \end{array}
}{
  \Equiv{A}{B}{\pre}{\post}
} [\verb+fun+~\chi]
\end{displaymath}
%
where $\vec{p}_f$ represent the formal parameters of a function
(abstract adversary or oracle) $f$, $\result_f$ represents the result of
a function (abstract adversary or oracle) $f$, $\left\{O_i\right\}_{i=0}^k$ and
$\left\{O'_i\right\}_{i=0}^k$ are the oracles of the abstract adversaries $A$ and
$B$, $\glob_A$ and $\glob_B$ represent the global state of the abstract
adversaries $A$ and $B$, ...

\subsubsection{Probabilistic Hoare Logic}
\begin{displaymath}
\infrule{
  \begin{array}{c}
    \pre \Rightarrow \chi  \qquad 
    \chi \Leftrightarrow\post
    \\[.5ex]
    \HoareEq{O_i}{\chi}{\chi}{1}
  \end{array}
}{
  \HoareEq{A}{\pre}{\post}{1}
} [\verb+fun+~\chi]
\end{displaymath}

\subsubsection{Hoare Logic}
\begin{displaymath}
\infrule{
  \begin{array}{c}
    \pre \Rightarrow \chi  \qquad 
    \chi \Rightarrow\post
    \\[.5ex]
    \Hoare{O_i}{\chi}{\chi}
  \end{array}
}{
  \Hoare{A}{\pre}{\post}
} [\verb+fun+~\chi]
\end{displaymath}

\subsection{??????: The \rawec{exfalso} rule}

\subsection{Frame rules ?? : The \rawec{eqobsin} rule}

\subsection{Weakening judgements: The \rawec{conseq} rule}

\Syntax \verb+conseq+ \textit{formula} \textit{formula}
\subsubsection{Hoare Logic}

\begin{displaymath}
\infrule{
  \Hoare{c}{\pre'}{\post'} \qquad \pre\Rightarrow\pre' \qquad  \post'\Rightarrow\post
}{
  \Hoare{c}{\pre}{\post}
}\left[\verb+conseq+~ \pre'~ \post' \right]
\end{displaymath}

\subsubsection{Probabilistic Hoare Logic}
\begin{displaymath}
\infrule{
  \HoareLe{c}{\pre'}{\post'}{\delta} \qquad \pre\Rightarrow\pre' \qquad  \post\Rightarrow\post'
}{
  \HoareLe{c}{\pre}{\post}{\delta}
}\left[\verb+conseq+~ \pre'~ \post' \right]
\end{displaymath}

\begin{displaymath}
\infrule{
  \HoareEq{c}{\pre'}{\post'}{\delta} \qquad \pre\Rightarrow\pre' \qquad  \post\Leftrightarrow\post'
}{
  \HoareEq{c}{\pre}{\post}{\delta}
}\left[\verb+conseq+~ \pre'~ \post' \right]
\end{displaymath}

\begin{displaymath}
\infrule{
  \HoareGe{c}{\pre'}{\post'}{\delta} \qquad \pre\Rightarrow\pre' \qquad  \post'\Rightarrow\post
}{
  \HoareGe{c}{\pre}{\post}{\delta}
}\left[\verb+conseq+~ \pre'~ \post' \right]
\end{displaymath}

\warningbox{(changing the bound is not yet implemented)}

\subsubsection{Relational Hoare Logic}

\begin{displaymath}
\infrule{
  \Equiv{c_1}{c_2}{\pre'}{\post'} \qquad \pre\Rightarrow\pre' \qquad  \post'\Rightarrow\post
}{
  \Equiv{c_1}{c_2}{\pre}{\post}
}\left[\verb+conseq+~ \pre'~ \post' \right]
\end{displaymath}


\subsection{Reasoning about function calls: the \rawec{call} tactic}
%
\subsubsection{Hoare Logic}
\index{hoare}{Program Reasoning!call@\rawec{call}}
\Syntax \verb+call+ formula formula
\Description

Let $p$ stand for the formal parameters of function $f$, $\result_f$
the result variable of function $f$, and $\vec{m}$ the set of
variables modifiable by $f$.
\begin{displaymath}
  \infrule{
    \begin{array}{c}
      \Hoare{c}{\pre}{\pre_f\subst{\vec{p}}{\vec{y}} \land
        \forall v.~ \forall \vec{z}.~ 
        \post_f\subst{\result_f}{v}\subst{\vec{m}}{\vec{z}}
        \Rightarrow \post\subst{x}{v}\subst{\vec{m}}{\vec{z}}
      }
      \\[.5ex]
      \Hoare{f}{\pre_f}{\post_f}
    \end{array}
  }{
    \Hoare{c;\Call{x}{f}{\vec{y}}}{\pre}{\post}
  }\left[\verb+call+~ \pre_f~ \post_f \right]
\end{displaymath}



\subsubsection{Probabilistic Hoare Logic}
\index{phl}{probabilistic Hoare logic!call@\rawec{call}}

\Syntax \verb+call+ formula formula [formula]

\Description

Let $p$ stand for the formal parameters of function $f$, $\result_f$
the result variable of function $f$, and $\vec{m}$ the set of
variables modifiable by $f$.
\begin{displaymath}
  \infrule{
    \begin{array}{c}
      \Hoare{c}{\pre}{\pre_f\subst{\vec{p}}{\vec{y}} \land
        \forall v.~ \forall \vec{z}.~ 
        \post_f\subst{\result_f}{v}\subst{\vec{m}}{\vec{z}}
        \Rightarrow \post\subst{x}{v}\subst{\vec{m}}{\vec{z}}
      }
      \\[.5ex]
      \HoareLe{f}{\pre_f}{\post_f}{\delta}
    \end{array}
  }{
    \HoareLe{c;\Call{x}{f}{\vec{y}}}{\pre}{\post}{\delta}
  } \left[\verb+call+~ \pre_f~ \post_f \right]
\end{displaymath}

\begin{displaymath}
  \infrule{
    \begin{array}{c}
      \HoareEq{c}{\pre}{\pre_f\subst{\vec{p}}{\vec{y}} \land
        \forall v.~ \forall \vec{z}.~ 
        \post_f\subst{\result_f}{v}\subst{\vec{m}}{\vec{z}}
        \Rightarrow \post\subst{x}{v}\subst{\vec{m}}{\vec{z}}}{\frac{\delta}{\delta'}}
    \\[.5ex]
    \HoareEq{f}{\pre_f}{\post_f}{\delta'}
  \end{array}
  }{
    \HoareEq{c;\Call{x}{f}{\vec{y}}}{\pre}{\post}{\delta}
  } \left[\verb+call+~ \pre_f~ \post_f~ \delta' \right]
\end{displaymath}

\begin{displaymath}
  \infrule{
    \begin{array}{c}
      \HoareGe{c}{\pre}{\pre_f\subst{\vec{p}}{\vec{y}} \land
        \forall v.~ \forall \vec{z}.~ 
        \post_f\subst{\result_f}{v}\subst{\vec{m}}{\vec{z}}
        \Rightarrow \post\subst{x}{v}\subst{\vec{m}}{\vec{z}}}
      {\frac{\delta}{\delta'}}
    \\[.5ex]
    \HoareGe{f}{\pre_f}{\post_f}{\delta'}
  \end{array}
  }{
    \HoareGe{c;\Call{x}{f}{\vec{y}}}{\pre}{\post}{\delta}
  } \left[\verb+call+~ \pre_f~ \post_f ~\delta' \right]
\end{displaymath}

If no parameter is given for the lower-bounded and exact judgements
then $\delta'=1$.

\warningbox{New tactics, needs structuring.}

\subsection{: the \rawec{hoare,hoare\_bd,pr\_bounded,bd\_eq}}

\subsubsection{Possibilistic and probabilistic Hoare Logic}
\Syntax \verb+hoare+, \verb+hoare_bd+
allows to switch between possibilistic and probabilistic logics
according to these rules:
\begin{displaymath}
\begin{array}{cc}
\infrule{
  \Hoare{c}{\pre}{\neg \post} \quad f = 0
}{
  \HoareEq{c}{\pre}{\post}{f}
}
&
\infrule{
  \HoareEq{c}{\pre}{\neg\post}{0}
}{
  \Hoare{c}{\pre}{\post}
}
\end{array}
\end{displaymath}

\Syntax \verb+pr_bounded+
discharges goals by applying trivial probability properties:
\begin{displaymath}
\begin{array}{cc}
\infrule{
}{
  \HoareLe{c}{\pre}{\post}{1}
}
&
\infrule{
}{
  \HoareGe{c}{\pre}{\post}{0}
}
% \\[3ex]
% \infrule{
% }{
%   \Prm{c}{m}{\post} \leq 1
% }
% &
% \infrule{
% }{
%   \Prm{c}{m}{\post} \geq 0
% }
\end{array}
\end{displaymath}

\Syntax \verb+bd_eq+
\begin{displaymath}
\begin{array}{cc}
\infrule{
  \HoareEq{c}{\pre}{\post}{f}
}{
  \HoareLe{c}{\pre}{\post}{f}
}
&
\infrule{
  \HoareEq{c}{\pre}{\post}{f}
}{
  \HoareGe{c}{\pre}{\post}{f}
}
\end{array}
\end{displaymath}


\subsection{\rawec{Denot} tactics}
%
\subsubsection{Probabilistic Hoare Logic}

\begin{displaymath}
\infrule{
    \pre 
    \qquad 
    \chi\Rightarrow\post 
    \qquad 
    \HoareLe{f}{\pre}{\post}{\delta}
}{
  \Prm{c}{m}{\chi} \leq \delta
}\left[\verb+hoare_deno+\ \pre\ \post\right]
\end{displaymath}

\begin{displaymath}
\infrule{
    \pre 
    \qquad 
    \post\Leftrightarrow \chi 
    \qquad 
    \HoareEq{f}{\pre}{\post}{\delta}
}{
  \Prm{c}{m}{\chi} = \delta
}\left[\verb+hoare_deno+\ \pre\ \post\right]
\end{displaymath}

\begin{displaymath}
\infrule{
    \pre 
    \qquad 
    \post\Rightarrow\chi
    \qquad 
    \HoareGe{f}{\pre}{\post}{\delta}
}{
  \delta \leq \Prm{c}{m}{\chi}
}\left[\verb+hoare_deno+\ \pre\ \post\right]
\end{displaymath}


\subsubsection{Relational Hoare Logic}

\begin{displaymath}
\infrule{
  \Equiv{c_1}{c_2}{\pre}{\post} 
  \qquad
  \pre
  \qquad
  \post \Rightarrow \chi_1 \Rightarrow \chi_2
}{
  \Prm{c_1}{m_1}{\chi_1} \leq \Prm{c_2}{m_2}{\chi_2}
}\left[\verb+deno+\ \pre\ \post\right]
\end{displaymath}

\begin{displaymath}
\infrule{
  \Equiv{c_1}{c_2}{\pre}{\post} 
  \qquad
  \pre
  \qquad
  \post \Rightarrow (\chi_1 \Leftrightarrow \chi_2)
}{
  \Prm{c_1}{m_1}{\chi_1} = \Prm{c_2}{m_2}{\chi_2}
}\left[\verb+deno+\ \pre\ \post\right]
\end{displaymath}


\subsection{Some \textsf{Pr} tactics: \rawec{pr\_false},
  \rawec{pr\_or}}

\begin{displaymath}
\infrule{
  \false \Rightarrow \post
}{
  \Prm{c}{m}{\post} = 0
}
\end{displaymath}

\begin{displaymath}
\infrule{
\Prm{c}{m}{\pre} \land
  \Prm{c}{m}{\post} \land \Prm{c}{m}{\pre \wedge \post} = \delta
}{
  \Prm{c}{m}{\pre \vee \post} = \delta
}
\end{displaymath}


\subsection{The \rawec{inline} tactic}
%

\subsection{The \rawec{swap} tactic}
%
\Syntax \verb+swap+ [\textit{side}] \textit{swap\_pos}

\textbf{where:} 
\begin{tabular}[t]{l}
  \textit{swap\_pos} ::= 
  \textit{n} \textit{n} \textit{n} $\mid$ \textit{n} \textit{z} $\mid$ [\textit{n}:\textit{n}] \textit{z}
  \\
  $n$ a natural number
  \\
  $z$ an integer number
\end{tabular}
  

The tactic [\verb+swap+ $p_1$ $p_2$ $p_3$] swaps the code between
positions $p_1$ and $p_2$ with the code between positions $p_2$ and
$p_3$. That is, assuming that $c_1$ and $c_2$ are syntactically
independent, that $c_1$ is between positions $p_1$ and $p_2$ and that
$c_2$ is between positions $p_2$ and $p_3$, the tactic implements the
following rule:
\begin{displaymath}
\infrule{
  \Hoare{c;c_2;c_1;c_3}{\pre}{\post}
}{
  \Hoare{c;c_1;c_2;c_3}{\pre}{\post}
} [\verb+swap+\ p_1\ p_2\ p_3]
\end{displaymath}

If $k$ is positive (negative) then [\verb+swap+ $k$] moves the first
(last) instruction $k$ positions forwards (backwards). Similarly,
[\verb+swap+ $i$ $k$] moves the $i^{th}$ instruction forwards or
backwards, and [\verb+swap+ $[i_1:i_2]$ $k$] moves the instructions
between positions $i_1$ and $i_2$.


\subsection{Reasoning about loops: the \rawec{while} tactic}
%
\subsubsection{Hoare Logic}

\Syntax

\Description


\subsubsection{Probabilistic Hoare Logic}
\index{phl}{Program Reasoning!while@\rawec{while}}

\Syntax \verb+while+ \textit{formula} \textit{formula} 
[\textit{formula} \textit{formula}]
%

\Description
%
The first formula is the loop invariant.
%
The second one is a variant expression. 
%
The third one is a real expression bound $g$ and the fourth one an
integer expression $n$.
%
If $g$ is not given then it is interpreted as $g=1$, and the fourth
formula is ignored, otherwise required. $M$ stands for the variables
that may be modified by $c$.

\begin{displaymath}
  \infrule{
    \begin{array}{c}
    \HoareGe{c'}{\pre }{\chi \land 
      \forall M.~ (\chi \land 0 \leq e \Rightarrow \neg b)  \land
      \chi \land \neg b \Rightarrow \post}{f} 
    \\[.5ex]
    \forall k.~ \HoareEq{c}{\chi \land b \land e = k}{\chi \land e
      < k}{1}
  \end{array}
}{
    \HoareGe{c';\While{b}{c}}{\pre}{\post}{f}
  }\left[\verb+while+\ \chi\ e \right] 
\end{displaymath}
Similarly for (=).

\warningbox{The following variants are not implemented}

For an arbitrary bound $g$ the following rule generalizes the one
above for lower bounded judgments:
\begin{displaymath}
  \infrule{
    \begin{array}{c}
    \HoareGe{c'}{\pre }{\chi \land e \leq n \land 
      \forall M.~ (\chi \land 0 \leq e \Rightarrow \neg b) 
      \land (\chi \land \neg b \Rightarrow \post)}{\frac{f}{g^n}} 
    \\[.5ex]
    \HoareGe{c}{\chi \land b}{\chi}{g}
    \\[.5ex]
    \forall k.~ \HoareEq{c}{\chi \land b \land e = k}{e<k}{1}
  \end{array}
}{
    \HoareGe{c';\While{b}{c}}{\pre}{\post}{f}
  }\left[\verb+while+\ \chi\ e\ g\ n \right] 
\end{displaymath}

and the folowing one for exact judgments (=):
\begin{displaymath}
  \infrule{
    \begin{array}{c}
    \HoareGe{c'}{\pre }{\chi \land e = n \land 
      \forall M.~ (\chi\Rightarrow (0\leq e \Leftrightarrow \neg b)) 
        \land (\chi \land \neg b \Rightarrow \post)}
      {\frac{f}{g^n}}
    \\[.5ex]
    \HoareGe{c}{\chi \land b}{\chi}{g}
    \\[.5ex]
    \forall k.~ \HoareEq{c}{\chi \land b \land e = k}{e=k-1}{1}
  \end{array}
}{
    \HoareGe{c';\While{b}{c}}{\pre}{\post}{f}
  }\left[\verb+while+\ \chi\ e\ g\ n \right] 
\end{displaymath}

There is no appropriate rule for $(\leq)$.


\subsubsection{Relational Hoare Logic}

\Syntax  \verb+while+ [\textit{side}] \textit{form} [\textit{form}]

\Description This tactic applies the pRHL verification rules for
loops:
\begin{itemize}
\item the optional argument \textit{side} can be either \verb+{1}+ or
  \verb+{2}+ to indicate the application of one-sided versions of the
  rule. If missing, the two-sided rule for loops is considered.
\item the first \textit{form} argument is mandatory and is used as
  loop invariant. It can refer to variables in both the left and right
  programs.
\item the optional parameter \textit{form} is required (and accepted
  only) in the one-sided application of the rule. This parameter
  corresponds to the decreasing variant expression used to prove loop
  termination.
\end{itemize}



\paragraph{Two-sided version.}
%
\Syntax \verb+while+ \textit{form} 
%
\Description Applies the two-sided RHL rule for while loops, using the
\textit{form} parameter as loop invariant. This tactic requires that
the last instruction of both left and right statements are while loops.
In the rule, $M$ refers to the variables that may be modified by the
loop bodies.

\begin{displaymath}
\infrule{ 
  \begin{array}{c}
    \Equiv{c_2}{c'_2}{I \land e\sidel \land e'\sider}{I \land  e\sidel = e'\sider}\\
    \Equiv{c_1}{c'_1}{\pre}{ I \land e\sidel = e'\sider \land 
      \forall M, (I \land \neg e\sidel \land \neg e'\sider \Rightarrow \post)}
  \end{array}
}{
  \Equiv{c_1;\While{e}{c_2}}{c'_1;\While{e'}{c'_1}}{\pre}{\post}
}
\end{displaymath}

\paragraph{One-sided version.}

\Syntax \verb+while+ \textit{side} \textit{form} \textit{form} 

\Description Applies the one-sided pRHL rule for while loops, using
the first parameter \textit{form} as loop invariant and the second
parameter \textit{form} as a decreasing \textit{variant}
expression. The variant is used to verify the loop termination. The
one-sided rule are described below. Only the left (\verb+{1}+) variant
is shown; the right (\verb+{2}+) variant is symmetric. The expressions
$\forall X,~\varphi$ and $\exists X,~\varphi$ denote, respectively,
universal and existential quantification over the set of variables $X$
modified in the loop body $c$.

\begin{displaymath}
\infrule{
  \begin{array}{c}
    \vdash I \land v \leq b \Rightarrow \neg e  \\
    \Equiv{c}{\Skip}{b=B \land v=C \land e \land I }{b=B \land v<C \land I} \\
    \Equiv{c_1}{c_2}{\pre}{I \land \forall X, (I \land \neg e
      \Rightarrow \post)}
  \end{array}
}{
  \Equiv{c_1;\While{e}{c}}{c_2}{\pre}{\post}
}
\end{displaymath}

\subsection{Reasoning on function invocation: the \rawec{call}
  tactic}

\subsubsection{Hoare Logic}

\subsubsection{Probabilistic Hoare Logic}

\subsubsection{Relational Hoare Logic}


\subsection{Reasoning with \emph{failure events}: the \rawec{fel} tactic}
%
The following rule describes the application of the tactic
$\verb+fel+\ k\ q\ c\ \delta\ F\ P$.  Assume $f$ is defined and
$c_1,c_2$ stands for the splitting of its body at position $n$. Let
$\left\{O_i\right\}_{i=0}^k$ stand for all oracles accessed by any
adversary called at $c_2$. Assume that variables in $F$ can at most be
modified by $\left\{O_i\right\}_{i=0}^k$.
 
\begin{displaymath}
\infrule{
  \begin{array}{c}
    \left\{
    \begin{array}{l}
      \HoareLe{O_i}{\neg F}{F}{c \delta} \\
      \forall c_0,\ \Hoare{O_i}{P\land c=c_0}{c_0 < c} \\
      \forall c_0,\ \forall f_0,\ \Hoare{O_i}{\neg P\land F=f_0 \land c=c_0}{F=f_0 \land c=c_0} \\
    \end{array}\right\}_{i=0}^k\\[5ex]
    \forall m', (\varphi \Rightarrow F \land c\leq q) 
    \qquad 
    q (q-1) \delta \leq \epsilon 
    \qquad
    \Hoare{c_1}{\true}{\neg F \land c=0}
  \end{array}
}{
  \Prm{f}{m}{\varphi} \leq \epsilon  
} \left[\verb+fel+\ n\ q\ c\ \delta\ F\ P\right]
\end{displaymath}

\subsection{Proving equivalences by probability computation: 
  the \rawec{bypr} tactic}
%
\subsubsection{Relational Hoare Logic}

\begin{displaymath}
\infrule{
  \forall m_1,\ \forall m_2,\ 
  \Prm{f_1}{m_1}{\varphi_1} = \Prm{f_2}{m_2}{\varphi_2}
}{
  \Equiv{f_1}{f_2}{\pre}{\post}
}
\end{displaymath}


\subsection{Loop reordering}

\Syntax 

\Description 
An invocation of 
$$\left[\verb+reordering+\ i\ (p_{w_1},p_{\mathsf{incr}_1})\
  (p_{w_2},p_{\mathsf{incr}_2})\ \mathcal{I}\ \varolessthan\ (f,f^{-1})\right]$$
%
where $i$ is the iteration expression, $p_{w_1},p_{w_2}$ indicates the
position of the while loops and
$p_{\mathsf{incr}_1},p_{\mathsf{incr}_2}$ the position of the $d$
statement increasing $i$, $b$ only depends on $i$, $c$ does not modify
$i$, $\mathcal{I}:\verb+'a+\to \verb+Bool+$, $\varolessthan:
\verb+'a+\to\verb+'a+\to\verb+Bool+$, $f:\verb+'a+\to\verb+'a+$

 implements the following rule
%
\begin{displaymath}
\infrule{
\begin{array}{c}
  \Equiv{c_1}{c_2}{\pre}{\varphi \land \left| 
      \begin{array}{l}
        (\mathcal{I}\,i \Rightarrow\forall j,~\mathcal{I}\,j
        \Rightarrow i<j) \land  
      \\
      \varphi \land
      \\
      \forall j.~\neg\mathcal{I}\,j\Rightarrow \neg\,j \land
      \\
      \forall M.~ (\varphi\Rightarrow Q) \land
    \end{array}
    \right.
  }
  \\
  \mbox{$f$ bijection on $\mathcal{I}$}
  \\
  \forall z_1\,z_2\Equiv
  {c\subst{i}{f\,z_1};c\subst{i}{f\,z_2}}
  {c\subst{i}{f\,z_2};c\subst{i}{f\,z_1}}
  {
    \begin{array}{l}
      \mathcal{I}\,z_1 \land \mathcal{I}\,z_2 \land z_1\varolessthan
      z_2 
      \\ 
      \land f\, z_2\varolessthan f\, z_1\land \varphi
    \end{array}
  }
  {\varphi}
  \\
  \Equiv{c}{c}
  {\mathcal{I}\,i \land i\sidel=i\sider \land \varphi}
  {\varphi}
\end{array}
}{
\Equiv{c_1;\While{b}{(c(i);d)}}{c_2;\While{b}{(c(f\,i);d)}}{\pre}{\post}
}
\end{displaymath}


\section{Tacticals}


\section{Automated Tactics}


%%% Local Variables: 
%%% mode: latex
%%% TeX-master: "easycrypt"
%%% End: 

% Libraries
% !TeX root = easycrypt.tex

\chapter{Standard Library\label{chap:libraries} (P-Y + Davide + others)}
\section{Theories}

\section{Cryptographic Assumptions and Properties}

%%% Local Variables: 
%%% mode: latex
%%% TeX-master: "easycrypt"
%%% End: 

% Examples
% !TeX root = easycrypt.tex

\chapter{Advanced Examples}

\section{Public Key Encryption Schemes}
\subsection{Another Look at \citet{br93}}
In this Section, we return on the example proof discussed in
Section~\ref{sec:tutorial} and rewrite it making use of the standard library and
advanced features of \EC presented in this manual.

\subsection{El Gamal}

\subsection{Hashed El Gamal}

\subsection{OAEP}

\section{Public Key Signature Schemes}
\subsection{FDH}
\subsection{PSS}

%%% Local Variables: 
%%% mode: latex
%%% TeX-master: "easycrypt"
%%% End: 


\part{Language Reference (Pierre-Yves)}

\printindex

\end{document}

\lstset{language=easycrypt}

\def\ls{\lstinline}
\def\ec#1{\lstinline[language=easycrypt]"#1"}

\def\Arg{\ensuretext{\ec{arg}}}

% --------------------------------------------------------------------
% Typesetting judgments
\newcommand{\pRHL}[4]{\{#1\}\; #2 \mathrel{\sim} #3\; \{#4\}}
\newcommand{\pHL}[5]{\{#1\}\; #2\; \{#3\} \mathrel{#4} #5}
\newcommand{\HL}[3]{\{#1\}\; #2\; \{#3\}}
\newcommand{\PR}[4]{\mathbf{Pr} [#3, #1(#2) : #4]}

\newcommand{\pRHLs}[4]{\ec{equiv [#2 ~ #3: #1 ==> #4]}}
\newcommand{\pHLs}[5]{\ec{phoare [#2: #1 ==> #3] #4 #5}}
\newcommand{\HLs}[3]{\ec{hoare [#2: #1 ==> #3]}}
\newcommand{\PRs}[4]{\ec{Pr[#1(#2) @ &#3: #4]}}

% --------------------------------------------------------------------
\newcommand{\mem}[1]{#1}
\newcommand{\inmem}[2]{#1\langle{#2}\rangle}
