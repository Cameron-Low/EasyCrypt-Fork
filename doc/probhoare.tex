

\subsection{Reasoning on the program structure}
\subsubsection*{Empty statements: the \rawec{skip} tactic}
\index{phl}{\rawec{skip}}

\Syntax \rawec{skip}

\Description Reduces logical program judgements with empty statements
to a higher-order logical goal. The generated subgoals can then be
processed using the ambient logic. The behaviour of the \rawec{skip}
tactic can be briefly described by the following rules:
%
\begin{displaymath}
\infrule{
  \pre \Rightarrow \post \qquad \bound = 1
}{
  \HoareLe{\Skip}{\pre}{\post}{\bound}
}
\end{displaymath}
%


\subsubsection*{Random samplings: the \rawec{rnd} tactic}
\index{phl}{\rawec{rnd}}

\Syntax 
\rawec{rnd} [\textit{formula} | \textit{formula} \textit{formula}
\textit{formula} \textit{formula} \textit{formula} \textit{formula} ] 

\Description

The invocation of this tactic expects a goal of any of the following
forms:
\begin{itemize}
\item $\HoareLe{\pre}{c;\Rand{x}{d}}{\post}{\bound}$
\item $\HoareEq{\pre}{c;\Rand{x}{d}}{\post}{\bound}$
\item $\HoareGe{\pre}{c;\Rand{x}{d}}{\post}{\bound}$
\end{itemize}

The following rule, corresponding to the invocation of
$
\left[\mbox{\rawec{rnd}}\ \varphi\ \bound_1\ \bound_2\ \bound_3\ \bound_4\ p\right],
$
describes the most general variant implemented by
the \rawec{rnd} tactic for probabilistic Hoare judgements:
%
\begin{displaymath}
\infrule{
  \begin{array}{c}
    \bound_1 \bound_2 + \bound_3 \bound_4 \leq \bound \\
    \HoareLe{c}{\pre}{\varphi}{\bound_1} \\
    \varphi \Rightarrow \mu\, d\, p \leq \bound_2 \land (\forall v,~\insupp{v}{d}\Rightarrow
    \post\subst{x}{v} \Rightarrow p\, v) \\
    \HoareLe{c}{\pre}{\neg \varphi}{\bound_3} \\
    \neg\varphi \Rightarrow \mu\, d\, p \leq \bound_4 \land (\forall v,~\insupp{v}{d}\Rightarrow
    \post\subst{x}{v} \Rightarrow p\, v) \\
  \end{array}
}{
  \HoareLe{c;\Rand{x}{d}}{\pre}{\post}{\bound}
}
\end{displaymath}
%
where $\insupp{z}{d}$ stands for the predicate $\ECinsupp{z}{d}$.
%
Similar rules hold by substituting every occurrence of the
comparison operator $\leq$ by the operator $=$, or by the operator $\geq$.

If a single parameter is given, then the \rawec{rnd} tactic implements
the following simpler rules:
\begin{displaymath}
  \infrule{
    \Hoare{c}{\pre}{\mu\, d\, p \leq f \land 
      (\forall v,~\insupp{v}{d}\Rightarrow \post\subst{x}{v} \Rightarrow p\, v)}
  }{
    \HoareLe{c;\Rand{x}{d}}{\pre}{\post}{f}
  }\left[\mathec{rnd}\ p\right]
\end{displaymath}
%
\begin{displaymath}
  \infrule{
    \HoareEq{c}{\pre}{\mu\, d\, p \geq \delta \land 
      (\forall v,~\insupp{v}{d} \Rightarrow p\, v \Rightarrow \post\subst{x}{v} )}{1} 
  }{
    \HoareGe{c;\Rand{x}{d}}{\pre}{\post}{\delta}
  }\left[\mathec{rnd}\ p\right]
\end{displaymath}
%
\begin{displaymath}
  \infrule{
    \HoareEq{c}{\pre}{\mu\, d\, p = \delta \land 
      \forall v,~ (\insupp{v}{d} \Rightarrow (p\, v \Leftrightarrow \post\subst{x}{v} ))}{1} 
  }{
    \HoareEq{c;\Rand{x}{d}}{\pre}{\post}{\delta}
  }\left[\mathec{rnd}\ p\right]
\end{displaymath}

If the \rawec{rnd} tactic is invoked with no parameters, and the
assigned variable does not occur in the postcondition, then the
following rule variants are applied:
%
\begin{displaymath}
  \infrule{
    \HoareLe{c}{\pre}{\post}{\bound}
  }{
    \HoareLe{c;\Rand{x}{d}}{\pre}{\post}{\bound}
  }\left[\mathec{rnd}\ p\right]
\end{displaymath}
%
\begin{displaymath}
  \infrule{
    \HoareEq{c}{\pre}{\post}{\bound} \qquad
    \mu\, d\, \mathec{cpTrue} = 1
  }{
    \HoareEq{c;\Rand{x}{d}}{\pre}{\post}{\delta}
  }\left[\mathec{rnd}\ p\right]
\end{displaymath}
%
\begin{displaymath}
  \infrule{
    \HoareGe{c}{\pre}{\post}{\bound} \qquad
    \mu\, d\, {cpTrue} = 1
  }{
    \HoareGe{c;\Rand{x}{d}}{\pre}{\post}{\bound}
  }\left[\mathec{rnd}\ p\right]
\end{displaymath}


If the judgement bound expression contains variables that are modified
by the judgement statement, then the same rules are applied after a
renaming of the bound expression enabled by the following equivalence: 
\begin{displaymath}
  \HoareLe{c;\Rand{x}{d}}{\pre}{\post}{\bound}
  \Longleftrightarrow
  \forall \bound', \HoareLe{c;\Rand{x}{d}}{\pre \land \bound=\bound'}{\post}{\bound'}
\end{displaymath}



\subsubsection*{Sequential composition: the \rawec{seq} tactic}
\index{phl}{\rawec{seq}}

\Syntax 
\rawec{seq}  \textit{codepos} \textit{formula}
[\textit{formula} | \textit{formula} \textit{formula}
\textit{formula} \textit{formula}]

\Description

\begin{displaymath}
  \infrule{
    \begin{array}{c}
      \Hoare{c_1}{\pre}{\varphi}
      \\
      \HoareLe{c_1}{\pre}{\chi}{\bound_1} \qquad 
          \HoareLe{c_2}{\varphi \land \chi }{\post}{\bound_2}
      \\
      \HoareLe{c_1}{\pre}{\neg \chi}{\bound_3} \qquad 
             \HoareLe{c_2}{\varphi \land \neg\chi}{\post}{\bound_4}
      \\
      \bound_1 \bound_2 + \bound_3 \bound_4 \leq \bound
    \end{array}
  }{
    \HoareLe{c_1;c_2}{\pre}{\post}{\bound}
  }\left[\mathec{seq}\ \varphi\ \chi\ f_1\ f_2\ f_3\ f_4 \right]
\end{displaymath}

(idem with the comparison operators ($=$) and ($\geq$)).



\subsubsection*{Conditional statements: the \rawec{condt,condf} tactic}
\index{phl}{\rawec{condt,condf}}
%
\NotDocumented


\subsubsection*{Conditional statements: the \rawec{if} tactic}
\index{phl}{\rawec{if}}
Applies the following rule for conditional statements. It expects a
conditional statement at the first program position.
\begin{displaymath}
\begin{array}{c}
  \infrule{
    \HoareLe{c_1;c}{\pre \land b}{\post}{f}\qquad
    \HoareLe{c_2;c}{\pre \land \neg b}{\post}{f}
  }{
    \HoareLe{\Cond{b}{c_1}{c_2};c}{\pre}{\post}{f}
  }\left[\mathec{if} \right] 
\\[4ex]
\end{array}
\end{displaymath}
Similar rules hold for the other comparison operators $=,\geq$.

\subsubsection*{Deterministic straight-line code: the \rawec{wp} tactic}
\index{phl}{\rawec{wp}}


\Syntax \rawec{wp} [\textit{codepos}]

\Description The \rawec{wp} tactic computes the weakest-precondition of
deterministic, loop and procedure-call free program fragments
(i.e. deterministic assignments and conditionals).   
The computation of the weakest precondition over a
conditional statement is only possible if its branches do not
contain random samplings, while-loops nor function calls.

The optional code position parameter \textit{pos} restricts the range
of instructions that may be affected by the tactic invocation. 
%
If the optional code position parameter is not provided, The tactic
processes instructions bottom-up until a random sampling, a loop or a
function call is reached.

\begin{displaymath}
  \infrule{
    \HoareLe{c_1}{\pre }{\mathsf{wp}(c_2,\post)}{f}
  }{
    \HoareLe{c_1;c_2}{\pre}{\post}{f}
  }\left[\mathec{wp} \right] 
\end{displaymath}
Similar rules hold for $=,\geq$.

\subsubsection*{Function call statements: the \rawec{call} tactic}
\index{phl}{\rawec{call}}

\Syntax \mathec{call} \textit{formula} \textit{formula} [\textit{formula}]

\Description

Let $p$ stand for the formal parameters of function $f$, $\result_f$
the result variable of function $f$, and $\vec{m}$ the set of
variables modifiable by $f$.
\begin{displaymath}
  \infrule{
    \begin{array}{c}
      \Hoare{c}{\pre}{\pre_f\subst{\vec{p}}{\vec{y}} \land
        \forall v.~ \forall \vec{z}.~ 
        \post_f\subst{\result_f}{v}\subst{\vec{m}}{\vec{z}}
        \Rightarrow \post\subst{x}{v}\subst{\vec{m}}{\vec{z}}
      }
      \\[.5ex]
      \HoareLe{f}{\pre_f}{\post_f}{\delta}
    \end{array}
  }{
    \HoareLe{c;\Call{x}{f}{\vec{y}}}{\pre}{\post}{\delta}
  } \left[\mathec{call}~ \pre_f~ \post_f \right]
\end{displaymath}

\begin{displaymath}
  \infrule{
    \begin{array}{c}
      \HoareEq{c}{\pre}{\pre_f\subst{\vec{p}}{\vec{y}} \land
        \forall v.~ \forall \vec{z}.~ 
        \post_f\subst{\result_f}{v}\subst{\vec{m}}{\vec{z}}
        \Rightarrow \post\subst{x}{v}\subst{\vec{m}}{\vec{z}}}{1}
    \\[.5ex]
    \HoareEq{f}{\pre_f}{\post_f}{\delta}
  \end{array}
  }{
    \HoareEq{c;\Call{x}{f}{\vec{y}}}{\pre}{\post}{\delta}
  } \left[\mathec{call}~ \pre_f~ \post_f \right]
\end{displaymath}

\begin{displaymath}
  \infrule{
    \begin{array}{c}
      \HoareEq{c}{\pre}{\pre_f\subst{\vec{p}}{\vec{y}} \land
        \forall v.~ \forall \vec{z}.~ 
        \post_f\subst{\result_f}{v}\subst{\vec{m}}{\vec{z}}
        \Rightarrow \post\subst{x}{v}\subst{\vec{m}}{\vec{z}}}
      {1}
    \\[.5ex]
    \HoareGe{f}{\pre_f}{\post_f}{\delta}
  \end{array}
  }{
    \HoareGe{c;\Call{x}{f}{\vec{y}}}{\pre}{\post}{\delta}
  } \left[\mathec{call}~ \pre_f~ \post_f \right]
\end{displaymath}


\subsubsection*{While loop statements: the \rawec{while} tactic}
\index{phl}{\rawec{while}}

\Syntax \rawec{while} \textit{formula} \textit{formula} 
%

\Description
%
The argument is the loop invariant, and the second one is a variant
expression used to prove termination. 
%
$M$ stands for the variables that may be modified by $c$.

\begin{displaymath}
  \infrule{
    \begin{array}{c}
    \HoareLe{c'}{\pre }{\chi \land 
      \forall M.~ (\chi \land 0 \leq e \Rightarrow \neg b)  \land
      \chi \land \neg b \Rightarrow \post}{f} 
    \\[.5ex]
    \forall k.~ \HoareEq{c}{\chi \land b \land e = k}{\chi \land e
      < k}{1}
  \end{array}
}{
    \HoareLe{c';\While{b}{c}}{\pre}{\post}{f}
  }\left[\mathec{while}\ \chi\ e \right] 
\end{displaymath}
Similarly for the comparison operators $=$ and $\leq$.


\subsection{Abstract adversaries: the \rawec{fun} tactic}
\index{phl}{\rawec{fun}}


\begin{displaymath}
\infrule{
    \pre \Rightarrow \chi  \qquad \chi \Leftrightarrow\post \qquad
    \HoareEq{O_i}{\chi}{\chi}{1}
}{
  \HoareEq{A}{\pre}{\post}{1}
} [\mathec{fun}~\chi]
\end{displaymath}


\subsection{Trivial judgements: the \rawec{pr_bounded} tactic}
\index{phl}{\rawec{pr_bounded}}

The \rawec{pr_bounded} tactic is only applicable to probabilistic
Hoare judgements. It discharges proof goals by trivial probability axioms:
\begin{displaymath}
\begin{array}{cc}
\infrule{
}{
  \HoareLe{c}{\pre}{\post}{1}
}
&
\infrule{
}{
  \HoareGe{c}{\pre}{\post}{0}
}
\end{array}
\end{displaymath}


\subsection{Strengthening goals}

\subsubsection*{The \rawec{conseq} tactic}
\index{phl}{\rawec{conseq}}


\begin{displaymath}
\infrule{
  \HoareLe{c}{\pre'}{\post'}{\delta} \qquad \pre\Rightarrow\pre' \qquad  \post\Rightarrow\post'
}{
  \HoareLe{c}{\pre}{\post}{\delta}
}\left[\mathec{conseq}~ (\_ : \pre'~ \mathec{==>}~ \post') \right]
\end{displaymath}

\begin{displaymath}
\infrule{
  \HoareEq{c}{\pre'}{\post'}{\delta} \qquad \pre\Rightarrow\pre' \qquad  \post\Leftrightarrow\post'
}{
  \HoareEq{c}{\pre}{\post}{\delta}
}\left[\mathec{conseq}~ (\_ : \pre'~ \mathec{==>}~ \post') \right]
\end{displaymath}

\begin{displaymath}
\infrule{
  \HoareGe{c}{\pre'}{\post'}{\delta} \qquad \pre\Rightarrow\pre' \qquad  \post'\Rightarrow\post
}{
  \HoareGe{c}{\pre}{\post}{\delta}
}\left[\mathec{conseq}~ (\_ : \pre'~ \mathec{==>}~ \post') \right]
\end{displaymath}


One can additionally strengthen the bounding condition with the syntax:

\rawec{conseq} (\_ : \_ \rawec{==>} \_ : [\textit{cmp}]
\textit{formula} ) 

that implements strengthening rules like the following:

\begin{displaymath}
\infrule{
  \HoareLe{c}{\pre}{\post}{\delta'} \qquad \delta'\leq\delta 
}{
  \HoareLe{c}{\pre}{\post}{\delta}
}\left[\mathec{conseq}~ (\_ : \_ ~\mathec{==>}~ \_) : \delta' \right]
\end{displaymath}

\begin{displaymath}
\infrule{
  \HoareEq{c}{\pre}{\post}{\delta'} \qquad \delta\leq\delta' 
}{
  \HoareGe{c}{\pre}{\post}{\delta}
}\left[\mathec{conseq}~ (\_ : \_ ~\mathec{==>}~ \_) : = \delta' \right]
\end{displaymath}



\subsubsection*{The \rawec{bd\_eq}}
\index{phl}{\rawec{bd_eq}}
%
\Syntax \rawec{bd_eq}
\begin{displaymath}
\begin{array}{cc}
\infrule{
  \HoareEq{c}{\pre}{\post}{f}
}{
  \HoareLe{c}{\pre}{\post}{f}
}
&
\infrule{
  \HoareEq{c}{\pre}{\post}{f}
}{
  \HoareGe{c}{\pre}{\post}{f}
}
\end{array}
\end{displaymath}


\subsection{Possibilistic or probabilistic: the \rawec{hoare,hoare\_bd} tactic}
\index{phl}{\rawec{hoare,hoare_db}}

\Syntax \mathec{hoare}, \mathec{hoare_bd}
allows to switch between possibilistic and probabilistic logics
according to these rules:
\begin{displaymath}
\infrule{
  \Hoare{c}{\pre}{\neg \post} \quad f = 0
}{
  \HoareEq{c}{\pre}{\post}{f}
}
\end{displaymath}

\subsection{Probability expressions: \rawec{deno} tactics}

\index{phl}{\rawec{deno,hoare_deno}}

\begin{displaymath}
\infrule{
    \pre 
    \qquad 
    \chi\Rightarrow\post 
    \qquad 
    \HoareLe{f}{\pre}{\post}{\delta}
}{
  \Prm{c}{m}{\chi} \leq \delta
}\left[\mathec{hoare_deno}\ \pre\ \post\right]
\end{displaymath}

\begin{displaymath}
\infrule{
    \pre 
    \qquad 
    \post\Leftrightarrow \chi 
    \qquad 
    \HoareEq{f}{\pre}{\post}{\delta}
}{
  \Prm{c}{m}{\chi} = \delta
}\left[\mathec{hoare_deno}\ \pre\ \post\right]
\end{displaymath}

\begin{displaymath}
\infrule{
    \pre 
    \qquad 
    \post\Rightarrow\chi
    \qquad 
    \HoareGe{f}{\pre}{\post}{\delta}
}{
  \delta \leq \Prm{c}{m}{\chi}
}\left[\mathec{hoare_deno}\ \pre\ \post\right]
\end{displaymath}



\subsection{Failure events: the \rawec{fel} tactic}
\index{phl}{\rawec{fel}}
%
The following rule describes the application of the tactic
$\mathec{fel}\ k\ q\ c\ \delta\ F\ P$.  Assume $f$ is defined and
$c_1,c_2$ stands for the splitting of its body at position $n$. Let
$\left\{O_i\right\}_{i=0}^k$ stand for all oracles accessed by any
adversary called at $c_2$. Assume that variables in $F$ can at most be
modified by $\left\{O_i\right\}_{i=0}^k$.
 
\begin{displaymath}
\infrule{
  \begin{array}{c}
    \left\{
    \begin{array}{l}
      \HoareLe{O_i}{\neg F}{F}{h(c)} \\
      \forall c_0,\ \Hoare{O_i}{P\land c=c_0}{c_0 < c} \\
      \forall c_0,\ \forall f_0,\ \Hoare{O_i}{\neg P\land F=f_0 \land c=c_0}{F=f_0 \land c=c_0} \\
    \end{array}\right\}_{i=0}^k\\[5ex]
    \forall m', (\varphi \Rightarrow F \land c\leq q) 
    \qquad 
    \sum_{i=0}^{q-1} h(i) \leq \epsilon 
    \qquad
    \Hoare{c_1}{\true}{\neg F \land c=0}
  \end{array}
}{
  \Prm{f}{m}{\varphi} \leq \epsilon  
} \left[\mathec{fel}\ n\ q\ h\ c\ F\ P\right]
\end{displaymath}


%%% Local Variables: 
%%% mode: latex
%%% TeX-master: "easycrypt"
%%% End: 
